\documentclass[a4paper,11pt,openany]{memoir}
\def\MyFileVersion{Version 1.7e, 2012/04/11}
\setlrmarginsandblock{3cm}{3cm}{1} 
\setulmarginsandblock{2.5cm}{2.5cm}{*}
\setmarginnotes{2.5mm}{2cm}{1em}
\checkandfixthelayout
\usepackage[latin1]{inputenc}
\usepackage[english]{babel}
\usepackage[T1]{fontenc}
\usepackage{
  calc,
  graphicx,
  url,
  fancyvrb,
  multicol,
  kvsetkeys
}
\usepackage[usenames]{color}
%\usepackage[usenames,dvipsnames,svgnames,table]{xcolor}
%\definecolor{felinesrcbgcolor}{rgb}{1,1,0.85}
%\definecolor{felinesrcbgcolor}{rgb}{0.94,0.97,1}
%\definecolor{felineframe}{rgb}{0.79,0.88,1}
%\definecolor{myorange}{rgb}{1,0.375,0}
\definecolor{lightgray}{gray}{0.5}

\usepackage[draft]{fixme}
\usepackage{fourier}
\usepackage[scaled]{luximono}


\usepackage[scaled]{berasans}

\chapterstyle{ell}
\renewcommand\tocheadstart{}
\renewcommand\printtoctitle[1]{}

%\raggedbottom
%\fvset{frame=lines,
%  framesep=3mm,
%  framerule=3pt,
%  fontsize=\small,
%  rulecolor=\color{myorange},
%  formatcom=\color{DarkGreen},
%}


\usepackage[unicode=true,pdfusetitle,
 bookmarks=true,bookmarksnumbered=true,bookmarksopen=true,bookmarksopenlevel=2,
 breaklinks=false,pdfborder={0 0 0},backref=false,colorlinks=false]{hyperref}

\setlength{\parindent}{0pt}
\begin{document}





\title{Matlab Toolbox Envelope: Reference Manual}
%\author{Dennis Cook\thanks{Email: \protect\url{dennis@stat.umn.edu}} \and Zhihua Su\thanks{Email: \protect\url{suzhihua@stat.umn.edu}} \and Yi Yang\thanks{Email: \protect\url{yiyang@umn.edu}}}

\maketitle



\setcounter{tocdepth}{2}
\tableofcontents


\newpage


\chapter{tools}

\rmfamily
\color{black}\section{bootstrapse}

\begin{par}
Perform bootstrap to estimate actual standard errors for models in the envelope family.
\end{par} \vspace{1em}

\subsection*{Contents}

\begin{itemize}
\setlength{\itemsep}{-1ex}
   \item Syntax
   \item Input
   \item Output
   \item Description
   \item Example
\end{itemize}


\subsection*{Syntax}


\begin{verbatim}       bootse = bootstrapse(X, Y, u, B, modelType)
       bootse = bootstrapse(X, Y, u, B, modelType, Opts)\end{verbatim}
    

\subsection*{Input}

\begin{par}
\textbf{X}: Predictors.   The predictors can be univariate or multivariate, discrete or continuous.
\end{par} \vspace{1em}
\begin{par}
For model type for method 'env', 'henv', 'ienv', ' senv', and 'xenv'. X is an n by p matrix, p is the number of predictors.
\end{par} \vspace{1em}
\begin{par}
For model type 'penv', X is  A list containing the value of X1 and X2.
\end{par} \vspace{1em}
\begin{itemize}
\setlength{\itemsep}{-1ex}
   \item X.X1 (only for 'penv'): Predictors of main interest. An n by p1 matrix, n is the number of observations, and p1 is the number of main predictors. The predictors can be univariate or multivariate, discrete or continuous.
   \item X.X2 (only for 'penv'): Covariates, or predictors not of main interest.  An n by p2 matrix, p2 is the number of covariates.
\end{itemize}
\begin{par}
\textbf{Y}: Multivariate responses. An n by r matrix, r is the number of responses and n is number of observations. The responses must be continuous variables.
\end{par} \vspace{1em}
\begin{par}
\textbf{u}: Dimension of the envelope subspace. The legitimate range of u depends on the model specified.
\end{par} \vspace{1em}
\begin{par}
\textbf{B}: Number of bootstrap samples.  A positive integer.
\end{par} \vspace{1em}
\begin{par}
\textbf{modelType}: A string characters indicting the model, choices can be 'env', 'henv', 'ienv', 'penv', 'senv' and 'xenv'.
\end{par} \vspace{1em}
\begin{par}
\textbf{Opts}: A list containing the optional input parameter, to control the iterations in sg\_min. If one or several (even all) fields are not defined, the default settings are used.
\end{par} \vspace{1em}
\begin{itemize}
\setlength{\itemsep}{-1ex}
   \item Opts.maxIter: Maximum number of iterations.  Default value: 300.
   \item Opts.ftol: Tolerance parameter for F.  Default value: 1e-10.
   \item Opts.gradtol: Tolerance parameter for dF.  Default value: 1e-7.
   \item Opts.verbose: Flag for print out the number of bootstrap samples, logical 0 or 1. Default value: 0.
\end{itemize}


\subsection*{Output}

\begin{par}
bootse: For 'env', 'henv', 'ienv', 'senv' and 'xenv', an r by p matrix containing the standard errors for elements in $\beta$ computed by bootstrap.  For 'penv', an r by p1 matrix containing the standard errors for $\beta_1$ computed by bootstrap.
\end{par} \vspace{1em}


\subsection*{Description}

\begin{par}
This function computes the bootstrap standard errors for the regression coefficients or for partial envelope model, the main regression coefficients in the specified model by bootstrapping the residuals.
\end{par} \vspace{1em}


\subsection*{Example}


\begin{verbatim}       load wheatprotein.txt
       X = wheatprotein(:, 8);
       Y = wheatprotein(:, 1:6);
       alpha = 0.01;
       u = lrt_env(X, Y, alpha);
       B = 100;
       modelType = 'env';
       bootse = bootstrapse(X, Y, u, B, modelType)\end{verbatim}
        \color{lightgray}\ttfamily 
        \begin{verbatim}
bootse =

    0.2896
    0.4352
    0.3189
    0.5735
    0.2543
    0.5840
\end{verbatim} 
\rmfamily
\color{black}
    
\begin{verbatim}       load fiberpaper.dat
       Y = fiberpaper(:, 1 : 4);
       Xtemp = fiberpaper(:, 5 : 7);
       X.X1 = Xtemp(:, 3);
       X.X2 = Xtemp(:, 1 : 2);
       alpha = 0.01;
       u = lrt_penv(X, Y, alpha);
       B = 100;
       modelType = 'penv';
       bootse = bootstrapse(X, Y, u, B, modelType)\end{verbatim}
    

        \color{lightgray}\ttfamily \begin{verbatim}


bootse =

    0.0027
    0.0012
    0.0020
    0.0009

\end{verbatim} \rmfamily
\color{black}
    

\newpage

\rmfamily
\color{black}\section{bootstrapse\_OLS}

\begin{par}
Compute bootstrap standard error for ordinary least squares.
\end{par} \vspace{1em}

\subsection*{Contents}

\begin{itemize}
\setlength{\itemsep}{-1ex}
   \item Syntax
   \item Input
   \item Output
   \item Description
   \item Example
\end{itemize}


\subsection*{Syntax}


\begin{verbatim}       bootse = bootstrapse_OLS(X, Y, B)\end{verbatim}
    

\subsection*{Input}

\begin{par}
\textbf{X}: Predictors, an n by p matrix, p is the number of predictors.  The predictors can be univariate or multivariate, discrete or continuous.
\end{par} \vspace{1em}
\begin{par}
\textbf{Y}: Multivariate responses, an n by r matrix, r is the number of responses and n is number of observations.  The responses must be continuous variables.
\end{par} \vspace{1em}
\begin{par}
\textbf{B}: Number of bootstrap samples.  A positive integer.
\end{par} \vspace{1em}
\begin{par}
\textbf{Opts}: A list containing the optional input parameter. If not defined, the default setting is used.
\end{par} \vspace{1em}
\begin{itemize}
\setlength{\itemsep}{-1ex}
   \item Opts.verbose: Flag for print out the number of bootstrap samples, logical 0 or 1. Default value: 0.
\end{itemize}


\subsection*{Output}

\begin{par}
\textbf{bootse}: The standard error for elements in $\beta$ computed by bootstrap.  An r by p matrix.
\end{par} \vspace{1em}


\subsection*{Description}

\begin{par}
This function computes the bootstrap standard errors for the regression coefficients in ordinary least squares by bootstrapping the residuals.
\end{par} \vspace{1em}


\subsection*{Example}


\begin{verbatim}       load wheatprotein.txt
       X = wheatprotein(:, 8);
       Y = wheatprotein(:, 1 : 6);
       bootse = bootstrapse_OLS(X, Y, 200)\end{verbatim}
    
        \color{lightgray}\ttfamily \begin{verbatim}
bootse =

   10.2168
    8.3940
    9.0503
    9.9677
   14.5822
    5.5874

\end{verbatim} \rmfamily
\color{black}

\newpage
    
\rmfamily
\color{black}\section{modelselectaic}

\begin{par}
Select the dimension for the envelope family using Akaike information criteria.
\end{par} \vspace{1em}

\subsection*{Contents}

\begin{itemize}
\setlength{\itemsep}{-1ex}
   \item Syntax
   \item Input
   \item Output
   \item Description
   \item Example
\end{itemize}


\subsection*{Syntax}


\begin{verbatim}       u = modelselectaic(X, Y, modelType)
       u = modelselectaic(X, Y, modelType, Opts)\end{verbatim}
    

\subsection*{Input}

\begin{par}
\textbf{X}: Predictors.   The predictors can be univariate or multivariate, discrete or continuous.
\end{par} \vspace{1em}
\begin{par}
For model type for method 'env', 'henv', 'ienv', ' senv', and 'xenv'. X is an n by p matrix, p is the number of predictors.
\end{par} \vspace{1em}
\begin{par}
For model type 'penv', X is  A list containing the value of X1 and X2.
\end{par} \vspace{1em}
\begin{itemize}
\setlength{\itemsep}{-1ex}
   \item X.X1 (only for 'penv'): Predictors of main interest. An n by p1 matrix, n is the number of observations, and p1 is the number of main predictors. The predictors can be univariate or multivariate, discrete or continuous.
   \item X.X2 (only for 'penv'): Covariates, or predictors not of main interest.  An n by p2 matrix, p2 is the number of covariates.
\end{itemize}
\begin{par}
\textbf{Y}: Multivariate responses. An n by r matrix, r is the number of responses and n is number of observations. The responses must be continuous variables.
\end{par} \vspace{1em}
\begin{par}
\textbf{modelType}: A string characters indicting the model, choices can be 'env', 'henv', 'ienv', 'penv', 'senv' and 'xenv'.
\end{par} \vspace{1em}
\begin{par}
\textbf{Opts}: A list containing the optional input parameter, to control the iterations in sg\_min. If one or several (even all) fields are not defined, the default settings are used.
\end{par} \vspace{1em}
\begin{itemize}
\setlength{\itemsep}{-1ex}
   \item Opts.maxIter: Maximum number of iterations.  Default value: 300.
   \item Opts.ftol: Tolerance parameter for F.  Default value: 1e-10.
   \item Opts.gradtol: Tolerance parameter for dF.  Default value: 1e-7.
   \item Opts.verbose: Flag for print out dimension selection process, logical 0 or 1. Default value: 0.
\end{itemize}


\subsection*{Output}

\begin{par}
\textbf{u}: Dimension of the envelope. An integer between 0 and r.
\end{par} \vspace{1em}


\subsection*{Description}

\begin{par}
This function implements the Akaike information criteria (AIC) to select the dimension of the envelope subspace for method 'env', 'henv', 'ienv', 'penv', ' senv', and 'xenv'.
\end{par} \vspace{1em}


\subsection*{Example}


\begin{verbatim}       load wheatprotein.txt
       X = wheatprotein(:, 8);
       Y = wheatprotein(:, 1 : 6);
       modelType = 'env';
       u = modelselectaic(X, Y, modelType)\end{verbatim}
\color{lightgray}\ttfamily\begin{verbatim}
u =

     1

\end{verbatim}  \rmfamily
\color{black}
\begin{verbatim}       load fiberpaper.dat
       Y = fiberpaper(:, 1 : 4);
       Xtemp = fiberpaper(:, 5 : 7);
       X.X1 = Xtemp(:, 3);
       X.X2 = Xtemp(:, 1 : 2);
       modelType = 'penv';
       u = modelselectaic(X, Y, modelType)\end{verbatim}
    

\color{lightgray}\ttfamily\begin{verbatim}
u =

     3

\end{verbatim} \rmfamily
\color{black}
    
\newpage


\rmfamily
\color{black}\section{modelselectbic}

\begin{par}
Select the dimension for the envelope family using Bayesian information criteria.
\end{par} \vspace{1em}

\subsection*{Contents}

\begin{itemize}
\setlength{\itemsep}{-1ex}
   \item Syntax
   \item Input
   \item Output
   \item Description
   \item Example
\end{itemize}


\subsection*{Syntax}


\begin{verbatim}       u = modelselectbic(X, Y, modelType)
       u = modelselectbic(X, Y, modelType, Opts)\end{verbatim}
    

\subsection*{Input}

\begin{par}
\textbf{X}: Predictors.  The predictors can be univariate or multivariate, discrete or continuous.
\end{par} \vspace{1em}
\begin{par}
For model type for method 'env', 'henv', 'ienv', ' senv', and 'xenv'. X is an n by p matrix, p is the number of predictors.
\end{par} \vspace{1em}
\begin{par}
For model type 'penv', X is  A list containing the value of X1 and X2.
\end{par} \vspace{1em}
\begin{itemize}
\setlength{\itemsep}{-1ex}
   \item X.X1 (only for 'penv'): Predictors of main interest. An n by p1 matrix, n is the number of observations, and p1 is the number of main predictors. The predictors can be univariate or multivariate, discrete or continuous.
   \item X.X2 (only for 'penv'): Covariates, or predictors not of main interest.  An n by p2 matrix, p2 is the number of covariates.
\end{itemize}
\begin{par}
\textbf{Y}: Multivariate responses. An n by r matrix, r is the number of responses and n is number of observations. The responses must be continuous variables.
\end{par} \vspace{1em}
\begin{par}
\textbf{modelType}: A string characters indicting the model, choices can be 'env', 'henv', 'ienv', 'penv', 'senv' and 'xenv'.
\end{par} \vspace{1em}
\begin{par}
\textbf{Opts}: A list containing the optional input parameter, to control the iterations in sg\_min. If one or several (even all) fields are not defined, the default settings are used.
\end{par} \vspace{1em}
\begin{itemize}
\setlength{\itemsep}{-1ex}
   \item Opts.maxIter: Maximum number of iterations.  Default value: 300.
   \item Opts.ftol: Tolerance parameter for F.  Default value: 1e-10.
   \item Opts.gradtol: Tolerance parameter for dF.  Default value: 1e-7.
   \item Opts.verbose: Flag for print out dimension selection process, logical 0 or 1. Default value: 0.
\end{itemize}


\subsection*{Output}

\begin{par}
\textbf{u}: Dimension of the envelope. An integer between 0 and r.
\end{par} \vspace{1em}


\subsection*{Description}

\begin{par}
This function implements the Bayesian information criteria (BIC) to select the dimension of the envelope subspace for method 'env', 'henv', 'ienv', 'penv', ' senv', and 'xenv'.
\end{par} \vspace{1em}


\subsection*{Example}


\begin{verbatim}       load wheatprotein.txt
       X = wheatprotein(:, 8);
       Y = wheatprotein(:, 1 : 6);
       modelType = 'env';
       u = modelselectbic(X, Y, modelType)\end{verbatim}
        \color{lightgray}\ttfamily \begin{verbatim}
u =

     1

\end{verbatim} \rmfamily
\color{black}    
\begin{verbatim}       load fiberpaper.dat
       Y = fiberpaper(:, 1 : 4);
       Xtemp = fiberpaper(:, 5 : 7);
       X.X1 = Xtemp(:, 3);
       X.X2 = Xtemp(:, 1 : 2);
       modelType = 'penv';
       u = modelselectbic(X, Y, modelType)\end{verbatim}
    

   \color{lightgray}\ttfamily \begin{verbatim}

u =

     1

\end{verbatim} \rmfamily
\color{black}
    

\newpage


    
\rmfamily
\color{black}\section{modelselectlrt}

\begin{par}
Select the dimension for the envelope family using likelihood ratio testing procedure.
\end{par} \vspace{1em}

\subsection*{Contents}

\begin{itemize}
\setlength{\itemsep}{-1ex}
   \item Syntax
   \item Input
   \item Output
   \item Description
   \item Example
\end{itemize}


\subsection*{Syntax}


\begin{verbatim}       u = modelselectlrt(X, Y, alpha, modelType)
       u = modelselectlrt(X, Y, alpha, modelType, Opts)\end{verbatim}
    

\subsection*{Input}

\begin{par}
\textbf{X}: Predictors.   The predictors can be univariate or multivariate, discrete or continuous.
\end{par} \vspace{1em}
\begin{par}
For model type for method 'env', 'henv', 'ienv', and 'xenv'. X is an n by p matrix, p is the number of predictors.
\end{par} \vspace{1em}
\begin{par}
For model type 'penv', X is  A list containing the value of X1 and X2.
\end{par} \vspace{1em}
\begin{itemize}
\setlength{\itemsep}{-1ex}
   \item X.X1 (only for 'penv'): Predictors of main interest. An n by p1 matrix, n is the number of observations, and p1 is the number of main predictors. The predictors can be univariate or multivariate, discrete or continuous.
   \item X.X2 (only for 'penv'): Covariates, or predictors not of main interest.  An n by p2 matrix, p2 is the number of covariates.
\end{itemize}
\begin{par}
\textbf{Y}: Multivariate responses. An n by r matrix, r is the number of responses and n is number of observations. The responses must be continuous variables.
\end{par} \vspace{1em}
\begin{par}
\textbf{alpha}: Significance level for testing.  A real number between 0 and 1, often taken at 0.05 or 0.01.
\end{par} \vspace{1em}
\begin{par}
\textbf{modelType}: A string characters indicting the model, choices can be 'env', 'henv', 'ienv', 'penv' and 'xenv'.
\end{par} \vspace{1em}
\begin{par}
\textbf{Opts}: A list containing the optional input parameter, to control the iterations in sg\_min. If one or several (even all) fields are not defined, the default settings are used.
\end{par} \vspace{1em}
\begin{itemize}
\setlength{\itemsep}{-1ex}
   \item Opts.maxIter: Maximum number of iterations.  Default value: 300.
   \item Opts.ftol: Tolerance parameter for F.  Default value: 1e-10.
   \item Opts.gradtol: Tolerance parameter for dF.  Default value: 1e-7.
   \item Opts.verbose: Flag for print out dimension selection process, logical 0 or 1. Default value: 0.
\end{itemize}


\subsection*{Output}

\begin{par}
\textbf{u}: Dimension of the envelope. An integer between 0 and r.
\end{par} \vspace{1em}


\subsection*{Description}

\begin{par}
This function implements the likelihood ratio testing procedure to select the dimension of the envelope subspace for method 'env', 'henv', 'ienv', 'penv', and 'xenv'.  The likelihood ratio resting procedure does not support 'senv', because the scaled envelope models are not nested with the standard model.
\end{par} \vspace{1em}


\subsection*{Example}


\begin{verbatim}       load wheatprotein.txt
       X = wheatprotein(:, 8);
       Y = wheatprotein(:, 1 : 6);
       alpha = 0.01;
       modelType = 'env';
       u = modelselectlrt(X, Y, alpha, modelType)\end{verbatim}
            \color{lightgray}\ttfamily \begin{verbatim}
u =

     1
\end{verbatim} \rmfamily
\color{black}
\begin{verbatim}       load fiberpaper.dat
       Y = fiberpaper(:, 1 : 4);
       Xtemp = fiberpaper(:, 5 : 7);
       X.X1 = Xtemp(:, 3);
       X.X2 = Xtemp(:, 1 : 2);
       alpha = 0.01;
       modelType = 'penv';
       u = modelselectlrt(X, Y, alpha, modelType)\end{verbatim}
    


        \color{lightgray}\ttfamily \begin{verbatim}


u =

     1

\end{verbatim} \rmfamily
\color{black}
    




\newpage

\rmfamily
\color{black}\section{prediction}

\begin{par}
Perform estimation or prediction for models in the envelope family.
\end{par} \vspace{1em}

\subsection*{Contents}

\begin{itemize}
\setlength{\itemsep}{-1ex}
   \item Syntax
   \item Input
   \item Output
   \item Description
   \item Example
\end{itemize}


\subsection*{Syntax}


\begin{verbatim}       PredictOutput = prediction(ModelOutput, Xnew, infType, modelType)\end{verbatim}
    

\subsection*{Input}

\begin{par}
\textbf{ModelOutput}: A list containing the model outputs from fitting the models.
\end{par} \vspace{1em}
\begin{par}
\textbf{Xnew}: The value of X with which to estimate or predict Y.
\end{par} \vspace{1em}
\begin{par}
For 'env', 'henv', 'ienv', 'senv' and 'xenv', it is a p by 1 vector.
\end{par} \vspace{1em}
\begin{par}
For 'penv', it is a list containing the value of X1 and X2.
\end{par} \vspace{1em}

\begin{verbatim}* Xnew.X1 (only for 'penv'): A p1 by 1 vector containing the value of X1.
* Xnew.X2 (only for 'penv'): A p2 by 1 vector containing the value of X2.\end{verbatim}
    \begin{par}
\textbf{infType}: A string of characters indicting the inference type, the choices can be 'estimation' or 'prediction'.
\end{par} \vspace{1em}
\begin{par}
\textbf{modelType}: A string characters indicting the model, choices can be 'env', 'henv', 'ienv', 'penv', 'senv' and 'xenv'.
\end{par} \vspace{1em}


\subsection*{Output}

\begin{par}
\textbf{PredictOutput}: A list containing the results of the inference.
\end{par} \vspace{1em}
\begin{itemize}
\setlength{\itemsep}{-1ex}
   \item PredictOutput.value: The fitted value or the prediction value evaluated at Xnew. An r by 1 vector.
   \item PredictOutput.covMatrix: The covariance matrix of PredictOutput.value. An r by r matrix.
   \item PredictOutput.SE: The standard error of elements in PredictOutput.value. An r by 1 vector.
\end{itemize}


\subsection*{Description}

\begin{par}
This function evaluates the user-specified model, could be 'env', 'henv', 'ienv', 'penv', 'senv' or 'xenv', at new value Xnew.  It can perform estimation: find the fitted value when X = Xnew, or prediction: predict Y when X = Xnew.  The covariance matrix and the standard errors are also provided.
\end{par} \vspace{1em}


\subsection*{Example}


\begin{verbatim}       load wheatprotein.txt
       X = wheatprotein(:, 8);
       Y = wheatprotein(:, 1:6);
       modelType = 'env';
       u =  modelselectbic(X, Y, modelType);
       ModelOutput = env(X, Y, u);
       Xnew = X(2, :)';
       PredictOutput = predict_env(ModelOutput, Xnew, 'estimation')
       [PredictOutput.value, Y(2, :)'] % Compare the fitted value with
       the observed value\end{verbatim}
             \color{lightgray}\ttfamily \begin{verbatim}

PredictOutput = 

        value: [6x1 double]
    covMatrix: [6x6 double]
           SE: [6x1 double]


ans =

  474.7135  458.0000
  127.4740  112.0000
  251.2044  236.0000
  380.8280  368.0000
  380.9473  383.0000
   -6.3287  -15.0000

\end{verbatim} \rmfamily
\color{black}
    
   
\begin{verbatim}       load fiberpaper.dat
       Y = fiberpaper(:, 1 : 4);
       Xtemp = fiberpaper(:, 5 : 7);
       X.X1 = Xtemp(:, 3);
       X.X2 = Xtemp(:, 1 : 2);
       modelType = 'penv';
       u =  modelselectbic(X, Y, modelType);
       ModelOutput = penv(X, Y, u);
       Xnew.X1 = X.X1(1, :)';
       Xnew.X2 = X.X2(1, :)';
       PredictOutput = predict_penv(ModelOutput, Xnew, 'estimation')
       PredictOutput.SE\end{verbatim}
    

        \color{lightgray}\ttfamily \begin{verbatim}


PredictOutput = 

        value: [4x1 double]
    covMatrix: [4x4 double]
           SE: [4x1 double]


ans =

    1.4680
    0.4234
    0.7145
    0.3161

\end{verbatim} \rmfamily
\color{black}
 
    
\newpage

\rmfamily
\color{black}\section{testcoefficient}


\begin{verbatim}This function tests the null hypothesis L * beta * R = A versus the
alternative hypothesis L * beta * R ~= A, where beta is estimated under
the model in the envelope family.\end{verbatim}
    
\subsection*{Contents}

\begin{itemize}
\setlength{\itemsep}{-1ex}
   \item Syntax
   \item Input
   \item Output
   \item Description
   \item Example
\end{itemize}


\subsection*{Syntax}


\begin{verbatim}       TestOutput = testcoefficient(ModelOutput, modelType)
       TestOutput = testcoefficient(ModelOutput, modelType, TestInput)\end{verbatim}
    

\subsection*{Input}

\begin{par}
\textbf{ModelOutput}: A list containing the model outputs from fitting the models.
\end{par} \vspace{1em}
\begin{par}
\textbf{modelType}: A string characters indicting the model, choices can be 'env', 'henv', 'ienv', 'penv', 'senv' and 'xenv'.
\end{par} \vspace{1em}
\begin{par}
\textbf{TestInput}: A list that specifies the null hypothesis, including L, R, and A.  If not provided by the user, default values will be used.
\end{par} \vspace{1em}
\begin{itemize}
\setlength{\itemsep}{-1ex}
   \item TestInput.L: The matrix multiplied to $\beta$ on the left.  According to different model, it has different size requirement.  Default value will be set if the user does not specify.
\end{itemize}
\begin{itemize}
\setlength{\itemsep}{-1ex}
   \item TestInput.R: The matrix multiplied to $\beta$ on the right.  According to different model, it has different size requirement.  Default value will be set if the user does not specify.
\end{itemize}
\begin{itemize}
\setlength{\itemsep}{-1ex}
   \item TestInput.A: The matrix on the right hand side of the equation.  Default value will be set if the user does not specify.
\end{itemize}


\subsection*{Output}

\begin{par}
\textbf{TestOutput}: A list containing test statistics, degrees of freedom for the reference chi-squared distribution, the p-value, and the covariance matrix of vec($L\beta R$).  At the same time, a table is printed out.
\end{par} \vspace{1em}
\begin{itemize}
\setlength{\itemsep}{-1ex}
   \item TestOutput.chisqStatistic: The test statistics. A real number.
   \item TestOutput.df: The degrees of freedom of the reference chi-squared distribution.  A positive integer.
   \item TestOutput.pValue: p-value of the test.  A real number in [0, 1].
   \item TestOutput.covMatrix: The covariance matrix of vec($L\beta R$). A d1 * d2 by d1 * d2 matrix.
\end{itemize}


\subsection*{Description}

\begin{par}
This function tests for hypothesis $H_0: L\beta R = A$, versus $H_\alpha: L\beta R\neq A$.  The $\beta$ is estimated by a model in the envelope model.  If the user does not specify the values for L, R and A, then the test is equivalent to the standard F test on if $\beta = 0$ (for 'env', 'ienv', 'penv', 'senv' and 'xenv'), or if the group main effects are all zeros (for 'henv').  The test statistics used is vec $(L\beta R - A)$ $\hat{\Sigma}^{-1}$ vec $(L\beta R - A)^{T}$, and the reference distribution is chi-squared distribution with degrees of freedom the same as the length of vec(A).
\end{par} \vspace{1em}


\subsection*{Example}


\begin{verbatim}       load wheatprotein.txt
       X = wheatprotein(:, 8);
       Y = wheatprotein(:, 1:6);
       alpha = 0.01;
       u = lrt_env(X, Y, alpha);
       ModelOutput = env(X, Y, u);
       modelType = 'env';
       TestOutout = testcoefficient(ModelOutput, modelType);\end{verbatim}
         \color{lightgray}\ttfamily \begin{verbatim}
 Test Hypothesis     Chisq Statistic    DF     P-value
------------------------------------------------------------------------
L * beta * R = A         116.230         6        0.0000
------------------------------------------------------------------------
\end{verbatim} \rmfamily
\color{black}   
\begin{verbatim}       load fiberpaper.dat
       Y = fiberpaper(:, 1 : 4);
       Xtemp = fiberpaper(:, 5 : 7);
       X.X1 = Xtemp(:, 3);
       X.X2 = Xtemp(:, 1 : 2);
       alpha = 0.01;
       u = lrt_penv(X, Y, alpha);
       ModelOutput = penv(X, Y, u);
       r = size(Y, 2);
       p1 = size(X.X1, 2);
       TestInput.L = rand(2, r);
       TestInput.R = rand(p1, 1);
       TestInput.A = zeros(2, 1);
       TestOutout = testcoefficient_penv(ModelOutput, TestInput);\end{verbatim}
    
        \color{lightgray}\ttfamily \begin{verbatim}

 Test Hypothesis     Chisq Statistic    DF     P-value
------------------------------------------------------------------------
L * beta * R = A          12.598          2        0.0018
------------------------------------------------------------------------
\end{verbatim} \rmfamily
\color{black}
    


\newpage


\chapter{env}
    
\rmfamily
\color{black}\section{aic\_env}

\begin{par}
Select the dimension of the envelope subspace using Akaike information criterion.
\end{par} \vspace{1em}

\subsection*{Contents}

\begin{itemize}
\setlength{\itemsep}{-1ex}
   \item Syntax
   \item Input
   \item Output
   \item Description
   \item Example
\end{itemize}


\subsection*{Syntax}


\begin{verbatim}       u = aic_env(X, Y)
       u = aic_env(X, Y, Opts)\end{verbatim}
    

\subsection*{Input}

\begin{par}
\textbf{X}: Predictors. An n by p matrix, p is the number of predictors. The predictors can be univariate or multivariate, discrete or continuous.
\end{par} \vspace{1em}
\begin{par}
\textbf{Y}: Multivariate responses. An n by r matrix, r is the number of responses and n is number of observations. The responses must be continuous variables.
\end{par} \vspace{1em}
\begin{par}
\textbf{Opts}: A list containing the optional input parameter, to control the iterations in sg\_min. If one or several (even all) fields are not defined, the default settings are used.
\end{par} \vspace{1em}
\begin{itemize}
\setlength{\itemsep}{-1ex}
   \item Opts.maxIter: Maximum number of iterations.  Default value: 300.
   \item Opts.ftol: Tolerance parameter for F.  Default value: 1e-10.
   \item Opts.gradtol: Tolerance parameter for dF.  Default value: 1e-7.
   \item Opts.verbose: Flag for print out dimension selection process, logical 0 or 1. Default value: 0.
\end{itemize}


\subsection*{Output}

\begin{par}
\textbf{u}: Dimension of the envelope. An integer between 0 and r.
\end{par} \vspace{1em}


\subsection*{Description}

\begin{par}
This function implements the Akaike information criteria (AIC) to select the dimension of the envelope subspace.
\end{par} \vspace{1em}


\subsection*{Example}


\begin{verbatim}       load wheatprotein.txt
       X = wheatprotein(:, 8);
       Y = wheatprotein(:, 1:6);
       u = aic_env(X, Y)\end{verbatim}
    
        \color{lightgray}\ttfamily \begin{verbatim}
u =

     1

\end{verbatim} \rmfamily
\color{black}
    
 

\newpage

\rmfamily
\color{black}\section{bic\_env}

\begin{par}
Select the dimension of the envelope subspace using Bayesian information criterion.
\end{par} \vspace{1em}

\subsection*{Contents}

\begin{itemize}
\setlength{\itemsep}{-1ex}
   \item Syntax
   \item Input
   \item Output
   \item Description
   \item Example
\end{itemize}


\subsection*{Syntax}


\begin{verbatim}       u = bic_env(X, Y)
       u = bic_env(X, Y, Opts)\end{verbatim}
    

\subsection*{Input}

\begin{par}
\textbf{X}: Predictors. An n by p matrix, p is the number of predictors and n is the number of observations. The predictors can be univariate or multivariate, discrete or continuous.
\end{par} \vspace{1em}
\begin{par}
\textbf{Y}: Multivariate responses. An n by r matrix, r is the number of responses. The responses must be continuous variables.
\end{par} \vspace{1em}
\begin{par}
\textbf{Opts}: A list containing the optional input parameter, to control the iterations in sg\_min. If one or several (even all) fields are not defined, the default settings are used.
\end{par} \vspace{1em}
\begin{itemize}
\setlength{\itemsep}{-1ex}
   \item Opts.maxIter: Maximum number of iterations.  Default value: 300.
   \item Opts.ftol: Tolerance parameter for F.  Default value: 1e-10.
   \item Opts.gradtol: Tolerance parameter for dF.  Default value: 1e-7.
   \item Opts.verbose: Flag for print out dimension selection process, logical 0 or 1. Default value: 0.
\end{itemize}


\subsection*{Output}

\begin{par}
\textbf{u}: Dimension of the envelope. An integer between 0 and r.
\end{par} \vspace{1em}


\subsection*{Description}

\begin{par}
This function implements the Bayesian information criteria (BIC) to select the dimension of the envelope subspace.
\end{par} \vspace{1em}


\subsection*{Example}


\begin{verbatim}       load wheatprotein.txt
       X = wheatprotein(:, 8);
       Y = wheatprotein(:, 1:6);
       u = bic_env(X, Y)\end{verbatim}
    
        \color{lightgray}\ttfamily \begin{verbatim}
u =

     1

\end{verbatim} \rmfamily
\color{black}
 
 
 \newpage

\rmfamily
\color{black}\section{bstrp\_env}

\begin{par}
Compute bootstrap standard error for the envelope model.
\end{par} \vspace{1em}

\subsection*{Contents}

\begin{itemize}
\setlength{\itemsep}{-1ex}
   \item Syntax
   \item Input
   \item Output
   \item Description
   \item Example
\end{itemize}


\subsection*{Syntax}


\begin{verbatim}       bootse = bstrp_env(X, Y, u, B)
       bootse = bstrp_env(X, Y, u, B, Opts)\end{verbatim}
    

\subsection*{Input}

\begin{par}
\textbf{X}: Predictors, an n by p matrix, p is the number of predictors.  The predictors can be univariate or multivariate, discrete or continuous.
\end{par} \vspace{1em}
\begin{par}
\textbf{Y}: Multivariate responses, an n by r matrix, r is the number of responses and n is number of observations.  The responses must be continuous variables.
\end{par} \vspace{1em}
\begin{par}
\textbf{u}: Dimension of the envelope subspace.  A positive integer between 0 and r.
\end{par} \vspace{1em}
\begin{par}
\textbf{B}: Number of bootstrap samples.  A positive integer.
\end{par} \vspace{1em}
\begin{par}
\textbf{Opts}: A list containing the optional input parameter, to control the iterations in sg\_min. If one or several (even all) fields are not defined, the default settings are used.
\end{par} \vspace{1em}
\begin{itemize}
\setlength{\itemsep}{-1ex}
   \item Opts.maxIter: Maximum number of iterations.  Default value: 300.
   \item Opts.ftol: Tolerance parameter for F.  Default value: 1e-10.
   \item Opts.gradtol: Tolerance parameter for dF.  Default value: 1e-7.
   \item Opts.verbose: Flag for print out the number of bootstrap samples, logical 0 or 1. Default value: 0.
\end{itemize}


\subsection*{Output}

\begin{par}
\textbf{bootse}: The standard error for elements in $\beta$ computed by bootstrap.  An r by p matrix.
\end{par} \vspace{1em}


\subsection*{Description}

\begin{par}
This function computes the bootstrap standard errors for the regression coefficients in the envelope model by bootstrapping the residuals.
\end{par} \vspace{1em}


\subsection*{Example}


\begin{verbatim}       load wheatprotein.txt
       X = wheatprotein(:, 8);
       Y = wheatprotein(:, 1:6);
       alpha = 0.01;
       u = lrt_env(X, Y, alpha)
\end{verbatim} \rmfamily
\color{black}
\color{lightgray}\ttfamily \begin{verbatim}
u =

     1
\end{verbatim} \rmfamily
\color{black}
        \begin{verbatim}
       B = 100;
       bootse = bstrp_env(X, Y, u, B)\end{verbatim}
    
        \color{lightgray}\ttfamily \begin{verbatim}

bootse =

    0.2893
    0.4260
    0.3523
    0.5628
    0.1675
    0.6192

\end{verbatim} \rmfamily
\color{black}
    


\newpage


\rmfamily
\color{black}\section{dF4env}

\begin{par}
The first derivative of the objective function for computing the envelope subspace.
\end{par} \vspace{1em}

\subsection*{Contents}

\begin{itemize}
\setlength{\itemsep}{-1ex}
   \item Syntax
   \item Input
   \item Output
   \item Description
\end{itemize}


\subsection*{Syntax}


\begin{verbatim}       df = dF4env(R, DataParameter)\end{verbatim}
    

\subsection*{Input}

\begin{par}
\textbf{R}: An r by u semi orthogonal matrix, 0\ensuremath{<}u\ensuremath{<}=r.
\end{par} \vspace{1em}
\begin{par}
\textbf{DataParameter}: A structure that contains the statistics calculated from the data.
\end{par} \vspace{1em}


\subsection*{Output}

\begin{par}
\textbf{df}: An r by u matrix containing the value of the derivative function evaluated at R.
\end{par} \vspace{1em}


\subsection*{Description}

\begin{par}
The objective function is derived in Section 4.3 in Cook et al. (2010) by  using maximum likelihood estimation. This function is the derivative of  the objective function.
\end{par} \vspace{1em}


\newpage



\rmfamily
\color{black}\section{env}

\begin{par}
Fit the envelope model.
\end{par} \vspace{1em}

\subsection*{Contents}

\begin{itemize}
\setlength{\itemsep}{-1ex}
   \item Syntax
   \item Input
   \item Output
   \item Description
   \item References
   \item Example
\end{itemize}


\subsection*{Syntax}


\begin{verbatim}       ModelOutput = env(X, Y, u)
       ModelOutput = env(X, Y, u, Opts)\end{verbatim}
    

\subsection*{Input}

\begin{par}
\textbf{X}: Predictors. An n by p matrix, p is the number of predictors. The predictors can be univariate or multivariate, discrete or continuous.
\end{par} \vspace{1em}
\begin{par}
\textbf{Y}: Multivariate responses. An n by r matrix, r is the number of responses and n is number of observations. The responses must be continuous variables, and r should be strictly greater than p.
\end{par} \vspace{1em}
\begin{par}
\textbf{u}: Dimension of the envelope. An integer between 0 and r.
\end{par} \vspace{1em}
\begin{par}
\textbf{Opts}: A list containing the optional input parameter, to control the iterations in sg\_min. If one or several (even all) fields are not defined, the default settings are used.
\end{par} \vspace{1em}
\begin{itemize}
\setlength{\itemsep}{-1ex}
   \item Opts.maxIter: Maximum number of iterations.  Default value: 300.
   \item Opts.ftol: Tolerance parameter for F.  Default value: 1e-10.
   \item Opts.gradtol: Tolerance parameter for dF.  Default value: 1e-7.
   \item Opts.verbose: Flag for print out Grassmann manifold optimization process, logical 0 or 1. Default value: 0.
   \item Opts.init: The initial value for the envelope subspace. An r by u matrix. Default value is the one generated by function get\_init.

\end{itemize}


\subsection*{Output}

\begin{par}
\textbf{ModelOutput}: A list that contains the maximum likelihood estimators and some statistics.
\end{par} \vspace{1em}
\begin{itemize}
\setlength{\itemsep}{-1ex}
   \item ModelOutput.beta: The envelope estimator of the regression coefficients $\beta$. An r by p matrix.
   \item ModelOutput.Sigma: The envelope estimator of the error covariance matrix.  An r by r matrix.
   \item ModelOutput.Gamma: The orthogonal basis of the envelope subspace. An r by u semi-orthogonal matrix.
   \item ModelOutput.Gamma0: The orthogonal basis of the complement of the envelope subspace.  An r by r-u semi-orthogonal matrix.
   \item ModelOutput.eta: The coordinates of $\beta$ with respect to Gamma. An u by p matrix.
   \item ModelOutput.Omega: The coordinates of Sigma with respect to Gamma. An u by u matrix.
   \item ModelOutput.Omega0: The coordinates of Sigma with respect to Gamma0. An r-u by r-u matrix.
   \item ModelOutput.alpha: The estimated intercept in the envelope model.  An r by 1 vector.
   \item ModelOutput.l: The maximized log likelihood function.  A real number.
   \item ModelOutput.covMatrix: The asymptotic covariance of vec($\beta$).  An rp by rp matrix.  The covariance matrix returned are asymptotic.  For the actual standard errors, multiply by 1/n.
   \item ModelOutput.asyEnv: The asymptotic standard error for elements in $\beta$ under the envelope model.  An r by p matrix.  The standard errors returned are asymptotic, for actual standard errors, multiply by 1/sqrt(n).
   \item ModelOutput.ratio: The asymptotic standard error ratio of the standard multivariate linear regression estimator over the envelope estimator, for each element in $\beta$.  An r by p matrix.
   \item ModelOutput.np: The number of parameters in the envelope model.  A positive integer.
   \item ModelOutput.n: The number of observations in the data.  A positive integer.
\end{itemize}


\subsection*{Description}

\begin{par}
This function fits the envelope model to the responses and predictors, using the maximum likelihood estimation.  When the dimension of the envelope is between 1 and r-1, we implemented the algorithm in Cook et al. (2010).  When the dimension is r, then the envelope model degenerates to the standard multivariate linear regression.  When the dimension is 0, it means that X and Y are uncorrelated, and the fitting is different.
\end{par} \vspace{1em}


\subsection*{References}

\begin{enumerate}
\setlength{\itemsep}{-1ex}
   \item The codes is implemented based on the algorithm in Section 4.3 of Cook et al (2010).
   \item The Grassmann manifold optimization step calls the package sg\_min 2.4.1 by Ross Lippert (http://web.mit.edu/$\sim$ripper/www.sgmin.html).
\end{enumerate}


\subsection*{Example}

\begin{par}
The following codes will reconstruct the results in the wheat protein data example in Cook et al. (2010).
\end{par} \vspace{1em}

\begin{verbatim}       load wheatprotein.txt
       X = wheatprotein(:, 8);
       Y = wheatprotein(:, 1:6);
       alpha = 0.01;
       u = lrt_env(X, Y, alpha)
\end{verbatim}    
        \color{lightgray}\ttfamily \begin{verbatim}
u =

     1
\end{verbatim} \rmfamily
\color{black}
       \begin{verbatim}
       ModelOutput = env(X, Y, u)
\end{verbatim}    

        \color{lightgray}\ttfamily \begin{verbatim}

ModelOutput = 

         beta: [6x1 double]
        Sigma: [6x6 double]
        Gamma: [6x1 double]
       Gamma0: [6x5 double]
          eta: 8.5647
        Omega: 7.8762
       Omega0: [5x5 double]
        alpha: [6x1 double]
            l: -850.7592
    covMatrix: [6x6 double]
       asyEnv: [6x1 double]
        ratio: [6x1 double]
           np: 28
            n: 50
\end{verbatim} \rmfamily
\color{black}
\begin{verbatim}
       ModelOutput.Omega
       \end{verbatim}

        \color{lightgray}\ttfamily \begin{verbatim}

ans =

    7.8762
\end{verbatim} \rmfamily
\color{black}
\begin{verbatim}
       eig(ModelOutput.Omega0)
       \end{verbatim}
        \color{lightgray}\ttfamily \begin{verbatim}

ans =

   1.0e+03 *

    6.5166
    0.2083
    0.0201
    0.0004
    0.0003

\end{verbatim} \rmfamily
\color{black}
\begin{verbatim}
       ModelOutput.ratio
       \end{verbatim}
        \color{lightgray}\ttfamily \begin{verbatim}

ans =

   28.0945
   18.4326
   23.6384
   16.3211
   65.8245
    6.4668

\end{verbatim} \rmfamily
\color{black}

\newpage


\rmfamily
\color{black}\section{F4env}

\begin{par}
Objective function for computing the envelope subspace.
\end{par} \vspace{1em}

\subsection*{Contents}

\begin{itemize}
\setlength{\itemsep}{-1ex}
   \item Syntax
   \item Input
   \item Output
   \item Description
\end{itemize}


\subsection*{Syntax}


\begin{verbatim}       f = F4env(R, DataParameter)\end{verbatim}
    

\subsection*{Input}

\begin{par}
\textbf{R}: An r by u semi orthogonal matrix, 0 \ensuremath{<} u \ensuremath{<}= r.
\end{par} \vspace{1em}
\begin{par}
\textbf{DataParameter}: A structure that contains the statistics calculated from the data.
\end{par} \vspace{1em}


\subsection*{Output}

\begin{par}
\textbf{f}: A scalar containing the value of the objective function evaluated at R.
\end{par} \vspace{1em}


\subsection*{Description}

\begin{par}
The objective function is derived in Section 4.3 of Cook et al. (2010)  using maximum likelihood estimation. The columns of the semi-orthogonal matrix that minimizes this function span the estimated envelope subspace.
\end{par} \vspace{1em}

\newpage

\rmfamily
\color{black}\section{lrt\_env}

\begin{par}
Select the dimension of the envelope subspace using likelihood ratio testing.
\end{par} \vspace{1em}

\subsection*{Contents}

\begin{itemize}
\setlength{\itemsep}{-1ex}
   \item Syntax
   \item Input
   \item Output
   \item Description
   \item Example
\end{itemize}


\subsection*{Syntax}


\begin{verbatim}       u = lrt_env(X, Y, alpha)
       u = lrt_env(X, Y, alpha, Opts)\end{verbatim}
    

\subsection*{Input}

\begin{par}
\textbf{X}: Predictors. An n by p matrix, p is the number of predictors. The predictors can be univariate or multivariate, discrete or continuous.
\end{par} \vspace{1em}
\begin{par}
\textbf{Y}: Multivariate responses. An n by r matrix, r is the number of responses and n is number of observations. The responses must be continuous variables.
\end{par} \vspace{1em}
\begin{par}
\textbf{alpha}: Significance level for testing.  A real number between 0 and 1, often taken at 0.05 or 0.01.
\end{par} \vspace{1em}
\begin{par}
\textbf{Opts}: A list containing the optional input parameter, to control the iterations in sg\_min. If one or several (even all) fields are not defined, the default settings are used.
\end{par} \vspace{1em}
\begin{itemize}
\setlength{\itemsep}{-1ex}
   \item Opts.maxIter: Maximum number of iterations.  Default value: 300.
   \item Opts.ftol: Tolerance parameter for F.  Default value: 1e-10.
   \item Opts.gradtol: Tolerance parameter for dF.  Default value: 1e-7.
   \item Opts.verbose: Flag for print out dimension selection process, logical 0 or 1. Default value: 0.
\end{itemize}


\subsection*{Output}

\begin{par}
\textbf{u}: Dimension of the envelope. An integer between 0 and r.
\end{par} \vspace{1em}


\subsection*{Description}

\begin{par}
This function implements the likelihood ratio testing procedure to select the dimension of the envelope subspace, with pre-specified significance level $\alpha$.
\end{par} \vspace{1em}


\subsection*{Example}


\begin{verbatim}       load wheatprotein.txt
       X = wheatprotein(:, 8);
       Y = wheatprotein(:, 1:6);
       alpha = 0.01;
       u = lrt_env(X, Y, alpha)\end{verbatim}
    
        \color{lightgray}\ttfamily \begin{verbatim}
u =

     1

\end{verbatim} \rmfamily
\color{black}
    
\newpage


\rmfamily
\color{black}\section{predict\_env}

\begin{par}
Perform estimation or prediction under the envelope model.
\end{par} \vspace{1em}

\subsection*{Contents}

\begin{itemize}
\setlength{\itemsep}{-1ex}
   \item Syntax
   \item Input
   \item Output
   \item Description
   \item Example
\end{itemize}


\subsection*{Syntax}


\begin{verbatim}       PredictOutput = predict_env(ModelOutput, Xnew, infType)\end{verbatim}
    

\subsection*{Input}

\begin{par}
\textbf{ModelOutput}: A list containing the maximum likelihood estimators and other statistics inherited from env.
\end{par} \vspace{1em}
\begin{par}
\textbf{Xnew}: The value of X with which to estimate or predict Y.  A p by 1 vector.
\end{par} \vspace{1em}
\begin{par}
\textbf{infType}: A string of characters indicting the inference type, the choices can be 'estimation' or 'prediction'.
\end{par} \vspace{1em}


\subsection*{Output}

\begin{par}
\textbf{PredictOutput}: A list containing the results of the inference.
\end{par} \vspace{1em}
\begin{itemize}
\setlength{\itemsep}{-1ex}
   \item PredictOutput.value: The fitted value or the prediction value evaluated at Xnew. An r by 1 vector.
   \item PredictOutput.covMatrix: The covariance matrix of PredictOutput.value. An r by r matrix.
   \item PredictOutput.SE: The standard error of elements in PredictOutput.value.  An r by 1 vector.
\end{itemize}


\subsection*{Description}

\begin{par}
This function evaluates the envelope model at new value Xnew.  It can perform estimation: find the fitted value when X = Xnew, or prediction: predict Y when X = Xnew.  The covariance matrix and the standard errors are also provided.
\end{par} \vspace{1em}


\subsection*{Example}


\begin{verbatim}       load wheatprotein.txt
       X = wheatprotein(:, 8);
       Y = wheatprotein(:, 1:6);
       alpha = 0.01;
       u = lrt_env(X, Y, alpha);
       ModelOutput = env(X, Y, u);
       Xnew = X(2, :)';
       PredictOutput = predict_env(ModelOutput, Xnew, 'estimation')
       [PredictOutput.value, Y(1, :)']  % Compare the fitted value with the data
       PredictOutput.SE
       \end{verbatim}
        \color{lightgray}\ttfamily \begin{verbatim}
PredictOutput = 

        value: [6x1 double]
    covMatrix: [6x6 double]
           SE: [6x1 double]


ans =

  474.7135  468.0000
  127.4740  123.0000
  251.2044  246.0000
  380.8280  374.0000
  380.9473  386.0000
   -6.3287  -11.0000


ans =

    4.8892
    4.0227
    4.3237
    4.7470
    6.8186
    2.6948
\end{verbatim} \rmfamily
\color{black}
       \begin{verbatim}
       PredictOutput = predict_env(ModelOutput, Xnew, 'prediction')
       PredictOutput.SE\end{verbatim}
            \color{lightgray}\ttfamily \begin{verbatim}
PredictOutput = 

        value: [6x1 double]
    covMatrix: [6x6 double]
           SE: [6x1 double]


ans =

  474.7135
  127.4740
  251.2044
  380.8280
  380.9473
   -6.3287


ans =

   34.9161
   28.7280
   30.8775
   33.9006
   48.6945
   19.2448
\end{verbatim} \rmfamily
\color{black}
 
\newpage


\rmfamily
\color{black}\section{testcoefficient\_env}


\begin{verbatim}This function tests the null hypothesis L * beta * R = A versus the
alternative hypothesis L * beta * R ~= A, where beta is estimated under
the envelope model.\end{verbatim}
    
\subsection*{Contents}

\begin{itemize}
\setlength{\itemsep}{-1ex}
   \item Syntax
   \item Input
   \item Output
   \item Description
   \item Example
\end{itemize}


\subsection*{Syntax}


\begin{verbatim}       TestOutput = testcoefficient_env(ModelOutput)
       TestOutput = testcoefficient_env(ModelOutput, TestInput)\end{verbatim}
    

\subsection*{Input}

\begin{par}
\textbf{ModelOutput}: A list containing the maximum likelihood estimators and other statistics inherited from env.
\end{par} \vspace{1em}
\begin{par}
\textbf{TestInput}: A list that specifies the null hypothesis, including L, R, and A.  If not provided by the user, default values will be used.
\end{par} \vspace{1em}
\begin{itemize}
\setlength{\itemsep}{-1ex}
   \item TestInput.L: The matrix multiplied to $\beta$ on the left.  It is a d1 by r matrix, while d1 is less than or equal to r.  Default value: identity matrix $I_r$.
   \item TestInput.R: The matrix multiplied to $\beta$ on the right.  It is a p by d2 matrix, while d2 is less than or equal to p.  Default value: identity matrix $I_p$.
   \item TestInput.A: The matrix on the right hand side of the equation.  It is a d1 by d2 matrix.  Default value: d1 by d2 zero matrix.
\end{itemize}


\subsection*{Output}

\begin{par}
\textbf{TestOutput}: A list containing test statistics, degrees of freedom for the reference chi-squared distribution, the p-value, and the covariance matrix of vec($L\beta R$).  At the same time, a table is printed out.
\end{par} \vspace{1em}
\begin{itemize}
\setlength{\itemsep}{-1ex}
   \item TestOutput.chisqStatistic: The test statistics. A real number.
   \item TestOutput.df: The degrees of freedom of the reference chi-squared distribution.  A positive integer.
   \item TestOutput.pValue: p-value of the test.  A real number in [0, 1].
   \item TestOutput.covMatrix: The covariance matrix of vec($L\beta R$). A d1 * d2 by d1 * d2 matrix.
\end{itemize}

\subsection*{Description}

\begin{par}
This function tests for hypothesis $H_0: L\beta R = A$, versus $H_\alpha: L\beta R\neq A$.  The $\beta$ is estimated by the envelope model.  If the user does not specify the values for L, R and A, then the test is equivalent to the standard F test on if $\beta = 0$.  The test statistics used is vec $(L\beta R - A)$ $\hat{\Sigma}^{-1}$ vec $(L\beta R - A)^{T}$, and the reference distribution is chi-squared distribution with degrees of freedom d1 * d2.
\end{par} \vspace{1em}


\subsection*{Example}


\begin{verbatim}       load wheatprotein.txt
       X = wheatprotein(:, 8);
       Y = wheatprotein(:, 1:6);
       alpha = 0.01;
       u = lrt_env(X, Y, alpha);
       ModelOutput = env(X, Y, u);
       TestOutout = testcoefficient_env(ModelOutput);
       \end{verbatim}
        \color{lightgray}\ttfamily \begin{verbatim}
 Test Hypothesis     Chisq Statistic    DF     P-value
------------------------------------------------------------------------
L * beta * R = A         116.230         6        0.0000
------------------------------------------------------------------------
\end{verbatim} \rmfamily
\color{black}       
\begin{verbatim}
       r = size(Y, 2);
       p = size(X, 2);
       TestInput.L = rand(2, r);
       TestInput.R = rand(p, 1);
       TestInput.A = zeros(2, 1);
       TestOutout = testcoefficient_senv(ModelOutput, TestInput);\end{verbatim}
    
        \color{lightgray}\ttfamily \begin{verbatim}
 Test Hypothesis     Chisq Statistic    DF     P-value
------------------------------------------------------------------------
L * beta * R = A          61.344         2        0.0000
------------------------------------------------------------------------
\end{verbatim} \rmfamily
\color{black}
    
\newpage

\chapter{henv}

\rmfamily
\color{black}\section{aic\_henv}

\begin{par}
Select the dimension of the envelope subspace using Akaike information criterion for the heteroscedastic envelope model.
\end{par} \vspace{1em}

\subsection*{Contents}

\begin{itemize}
\setlength{\itemsep}{-1ex}
   \item Syntax
   \item Input
   \item Output
   \item Description
   \item Example
\end{itemize}


\subsection*{Syntax}


\begin{verbatim}       u = aic_henv(X, Y)
       u = aic_henv(X, Y, Opts)\end{verbatim}
    

\subsection*{Input}

\begin{par}
\textbf{X}: Group indicators. A matrix with n rows.  X can only have p unique  rows, where p is the number of groups. For example, if there are two groups, X can only have 2 different kinds of rows, such as (0, 1) and (1, 0), or (1, 0, 10) and (0, 5, 6).  The number of columns is not restricted, as long as X only has p unique rows.
\end{par} \vspace{1em}
\begin{par}
\textbf{Y}: Multivariate responses. An n by r matrix, r is the number of responses and n is number of observations. The responses must be continuous variables, and r should be greater than p.
\end{par} \vspace{1em}
\begin{par}
\textbf{Opts}: A list containing the optional input parameter, to control the iterations in sg\_min. If one or several (even all) fields are not defined, the default settings are used.
\end{par} \vspace{1em}
\begin{itemize}
\setlength{\itemsep}{-1ex}
   \item Opts.maxIter: Maximum number of iterations.  Default value: 300.
   \item Opts.ftol: Tolerance parameter for F.  Default value: 1e-10.
   \item Opts.gradtol: Tolerance parameter for dF.  Default value: 1e-7.
   \item Opts.verbose: Flag for print out dimension selection process, logical 0 or 1. Default value: 0.
\end{itemize}


\subsection*{Output}

\begin{par}
\textbf{u}: Dimension of the envelope. An integer between 0 and r.
\end{par} \vspace{1em}


\subsection*{Description}

\begin{par}
This function implements the Akaike information criteria (AIC) to select the dimension of the envelope subspace for the heteroscedastic envelope model.
\end{par} \vspace{1em}


\subsection*{Example}


\begin{verbatim}       load waterstrider.mat
       u = aic_henv(X, Y)\end{verbatim}
    
        \color{lightgray}\ttfamily \begin{verbatim}
u =

     6

\end{verbatim} \rmfamily
\color{black}
    \newpage

\rmfamily
\color{black}\section{bic\_henv}

\begin{par}
Select the dimension of the envelope subspace using Bayesian information criterion for the heteroscedastic envelope model.
\end{par} \vspace{1em}

\subsection*{Contents}

\begin{itemize}
\setlength{\itemsep}{-1ex}
   \item Syntax
   \item Input
   \item Output
   \item Description
   \item Example
\end{itemize}


\subsection*{Syntax}


\begin{verbatim}       u = bic_henv(X, Y)
       u = bic_henv(X, Y, Opts)\end{verbatim}
    

\subsection*{Input}

\begin{par}
\textbf{X}: Group indicators. A matrix with n rows.  X can only have p unique  rows, where p is the number of groups. For example, if there are two groups, X can only have 2 different kinds of rows, such as (0, 1) and (1, 0), or (1, 0, 10) and (0, 5, 6).  The number of columns is not restricted, as long as X only has p unique rows.
\end{par} \vspace{1em}
\begin{par}
\textbf{Y}: Multivariate responses. An n by r matrix, r is the number of responses and n is number of observations. The responses must be continuous variables, and r should be greater than p.
\end{par} \vspace{1em}
\begin{par}
\textbf{Opts}: A list containing the optional input parameter, to control the iterations in sg\_min. If one or several (even all) fields are not defined, the default settings are used.
\end{par} \vspace{1em}
\begin{itemize}
\setlength{\itemsep}{-1ex}
   \item Opts.maxIter: Maximum number of iterations.  Default value: 300.
   \item Opts.ftol: Tolerance parameter for F.  Default value: 1e-10.
   \item Opts.gradtol: Tolerance parameter for dF.  Default value: 1e-7.
   \item Opts.verbose: Flag for print out dimension selection process, logical 0 or 1. Default value: 0.
\end{itemize}


\subsection*{Output}

\begin{par}
\textbf{u}: Dimension of the envelope. An integer between 0 and r.
\end{par} \vspace{1em}


\subsection*{Description}

\begin{par}
This function implements the Bayesian information criteria (BIC) to select the dimension of the envelope subspace for the heteroscedastic envelope model.
\end{par} \vspace{1em}


\subsection*{Example}


\begin{verbatim}       load waterstrider.mat
       u = bic_henv(X, Y)\end{verbatim}
    
        \color{lightgray}\ttfamily \begin{verbatim}
u =

     4

\end{verbatim} \rmfamily
\color{black}

\newpage

\rmfamily
\color{black}\section{bstrp\_henv}

\begin{par}
Compute bootstrap standard error for the heteroscedastic envelope model.
\end{par} \vspace{1em}

\subsection*{Contents}

\begin{itemize}
\setlength{\itemsep}{-1ex}
   \item Syntax
   \item Input
   \item Output
   \item Description
   \item Example
\end{itemize}


\subsection*{Syntax}


\begin{verbatim}       bootse = bstrp_henv(X, Y, u, B)
       bootse = bstrp_henv(X, Y, u, B, Opts)\end{verbatim}
    

\subsection*{Input}

\begin{par}
\textbf{X}: Group indicators. A matrix with n rows.  X can only have p unique  rows, where p is the number of groups. For example, if there are two groups, X can only have 2 different kinds of rows, such as (0, 1) and (1, 0), or (1, 0, 10) and (0, 5, 6).  The number of columns is not restricted, as long as X only has p unique rows.
\end{par} \vspace{1em}
\begin{par}
\textbf{Y}: Multivariate responses. An n by r matrix, r is the number of responses and n is number of observations. The responses must be continuous variables, and r should be greater than p.
\end{par} \vspace{1em}
\begin{par}
\textbf{u}: Dimension of the envelope subspace.  A positive integer between 0 and r.
\end{par} \vspace{1em}
\begin{par}
\textbf{B}: Number of bootstrap samples.  A positive integer.
\end{par} \vspace{1em}
\begin{par}
\textbf{Opts}: A list containing the optional input parameter, to control the iterations in sg\_min. If one or several (even all) fields are not defined, the default settings are used.
\end{par} \vspace{1em}
\begin{itemize}
\setlength{\itemsep}{-1ex}
   \item Opts.maxIter: Maximum number of iterations.  Default value: 300.
   \item Opts.ftol: Tolerance parameter for F.  Default value: 1e-10.
   \item Opts.gradtol: Tolerance parameter for dF.  Default value: 1e-7.
   \item Opts.verbose: Flag for print out the number of bootstrap samples, logical 0 or 1. Default value: 0.
\end{itemize}


\subsection*{Output}

\begin{par}
\textbf{bootse}: The standard error for elements in $\beta$ computed by bootstrap.  An r by p matrix.
\end{par} \vspace{1em}


\subsection*{Description}

\begin{par}
This function computes the bootstrap standard errors for the regression coefficients in the heteroscedastic envelope model by bootstrapping the residuals.
\end{par} \vspace{1em}


\subsection*{Example}


\begin{verbatim}       load waterstrider.mat\end{verbatim}
    
\begin{verbatim}       u = lrt_henv(X, Y, 0.01)\end{verbatim}
        \color{lightgray}\ttfamily \begin{verbatim}
u =

     6        
\end{verbatim} \rmfamily
\color{black}
\begin{verbatim}       
       B = 100;
       bootse = bstrp_henv(X, Y, u, B)\end{verbatim}
    
        \color{lightgray}\ttfamily \begin{verbatim}
bootse =

    0.0305    0.0466    0.0647
    0.0309    0.0485    0.0682
    0.0305    0.0432    0.0638
    0.0205    0.0289    0.0425
    0.0385    0.0553    0.0799
    0.0295    0.0427    0.0618
    0.0389    0.0567    0.0819
    0.0321    0.0463    0.0665

\end{verbatim} \rmfamily
\color{black}

\newpage


    
\rmfamily
\color{black}\section{dF4henv}

\begin{par}
The first derivative of the objective function for computing the envelope subspace in the heteroscedastic envelope model.
\end{par} \vspace{1em}

\subsection*{Contents}

\begin{itemize}
\setlength{\itemsep}{-1ex}
   \item Syntax
   \item Input
   \item Output
   \item Description
\end{itemize}


\subsection*{Syntax}


\begin{verbatim}       df = dF4henv(R, DataParameter)\end{verbatim}
    

\subsection*{Input}

\begin{par}
\textbf{R}: An r by u semi orthogonal matrix, 0 \ensuremath{<} u \ensuremath{<}= r.
\end{par} \vspace{1em}
\begin{par}
\textbf{DataParameter}: A structure that contains the statistics calculated from the data.
\end{par} \vspace{1em}


\subsection*{Output}

\begin{par}
\textbf{df}: An r by u matrix containing the value of the derivative function evaluated at R.
\end{par} \vspace{1em}


\subsection*{Description}

\begin{par}
The objective function is derived in Section 2.2 in Su and Cook (2012) by  using maximum likelihood estimation. This function is the derivative of  the objective function.
\end{par} \vspace{1em}

\newpage

\rmfamily
\color{black}\section{F4henv}

\begin{par}
Objective function for computing the envelope subspace in heteroscedastic envelope model.
\end{par} \vspace{1em}

\subsection*{Contents}

\begin{itemize}
\setlength{\itemsep}{-1ex}
   \item Syntax
   \item Input
   \item Output
   \item Description
\end{itemize}


\subsection*{Syntax}


\begin{verbatim}       f = F4henv(R, DataParameter)\end{verbatim}
    

\subsection*{Input}

\begin{par}
\textbf{R}: An r by u semi orthogonal matrix, 0 \ensuremath{<} u \ensuremath{<}= r.
\end{par} \vspace{1em}
\begin{par}
\textbf{DataParameter}: A structure that contains the statistics calculated from the data.
\end{par} \vspace{1em}


\subsection*{Output}

\begin{par}
\textbf{f}: A scalar containing the value of the objective function evaluated at R.
\end{par} \vspace{1em}


\subsection*{Description}

\begin{par}
The objective function is derived in Section 2.2 of Su and Cook (2012)  using maximum likelihood estimation. The columns of the semi-orthogonal matrix that minimizes this function span the estimated envelope subspace in the heteroscedastic envelope model.
\end{par} \vspace{1em}
\newpage


    
\rmfamily
\color{black}\section{henv}

\begin{par}
Fit the heteroscedastic envelope model.
\end{par} \vspace{1em}

\subsection*{Contents}

\begin{itemize}
\setlength{\itemsep}{-1ex}
   \item Syntax
   \item Input
   \item Output
   \item Description
   \item References
   \item Example
\end{itemize}


\subsection*{Syntax}


\begin{verbatim}       ModelOutput = henv(X, Y, u)
       ModelOutput = henv(X, Y, u, Opts)\end{verbatim}
    

\subsection*{Input}

\begin{par}
\textbf{X}: Group indicators. A matrix with n rows.  X can only have p unique  rows, where p is the number of groups. For example, if there are two groups, X can only have 2 different kinds of rows, such as (0, 1) and (1, 0), or (1, 0, 10) and (0, 5, 6).  The number of columns is not restricted, as long as X only has p unique rows.
\end{par} \vspace{1em}
\begin{par}
\textbf{Y}: Multivariate responses. An n by r matrix, r is the number of responses and n is number of observations. The responses must be continuous variables, and r should be greater than p.
\end{par} \vspace{1em}
\begin{par}
\textbf{u}: Dimension of the envelope. An integer between 0 and r.
\end{par} \vspace{1em}
\begin{par}
\textbf{Opts}: A list containing the optional input parameter, to control the iterations in sg\_min. If one or several (even all) fields are not defined, the default settings are used.
\end{par} \vspace{1em}
\begin{itemize}
\setlength{\itemsep}{-1ex}
   \item Opts.maxIter: Maximum number of iterations.  Default value: 300.
   \item Opts.ftol: Tolerance parameter for F.  Default value: 1e-10.
   \item Opts.gradtol: Tolerance parameter for dF.  Default value: 1e-7.
   \item Opts.verbose: Flag for print out Grassmann manifold optimization process, logical 0 or 1. Default value: 0.
   \item Opts.init: The initial value for the heteroscedastic envelope subspace. An r by u matrix. Default value is the one generated by function get\_init4henv.
\end{itemize}


\subsection*{Output}

\begin{par}
\textbf{ModelOutput}: A list that contains the maximum likelihood estimators and some statistics.
\end{par} \vspace{1em}
\begin{itemize}
\setlength{\itemsep}{-1ex}
   \item ModelOutput.mu: The heteroscedastic envelope estimator of the grand mean. A r by 1 vector.
   \item ModelOutput.mug: The heteroscedastic envelope estimator of the group mean. A r by p matrix, the ith column of the matrix contains the mean for the ith group.
   \item ModelOutput.Yfit: A n by r matrix, the ith row gives the group mean of the group that the ith observation belongs to.  As X is just a group indicator, and is not ordinal, ModelOutput.mug alone does not tell which group corresponds to which group mean.
   \item ModelOutput.Gamma: The orthogonal basis of the envelope subspace. An r by u semi-orthogonal matrix.
   \item ModelOutput.Gamma0: The orthogonal basis of the complement of the envelope subspace.  An r by r-u semi-orthogonal matrix.
   \item ModelOutput.beta: The heteroscedastic envelope estimator of the group main effect. An r by p matrix, the ith column of the matrix contains the main effect for the ith group.
   \item ModelOutput.groupInd: A matrix containing the unique values of group indicators. The matrix has p rows.  The group mean of the ith row is stored in the ith column of ModelOutput.mug.
   \item ModelOutput.Sigma: The heteroscedastic envelope estimator of the error covariance matrix.  A three dimensional matrix with dimension r, r and p, ModelOutput.Sigma(:, :, i) contains the estimated covariance matrix for the ith group.
   \item ModelOutput.eta: The coordinates of $\beta$ with respect to Gamma. An u by p matrix, the ith column contains the coordinates of the main effect of the ith group with respect to Gamma.
   \item ModelOutput.Omega: The coordinates of Sigma with respect to Gamma. An u by u by p matrix, ModelOutput.Omega(:, :, i) contains the coordinates of the covariance matrix of the ith group with respect to Gamma.
   \item ModelOutput.Omega0: The coordinates of Sigma with respect to Gamma0. An r - u by r - u matrix.
   \item ModelOutput.l: The maximized log likelihood function.  A real number.
   \item ModelOutput.np: The number of parameters in the heteroscedastic envelope model.  A positive integer.
   \item ModelOutput.covMatrix: The asymptotic covariance of ($\mu$', vec($\beta$'))'.  An r(p + 1) by r(p + 1) matrix.  The covariance matrix returned are asymptotic.  For the actual standard errors, multiply by 1 / n.
   \item ModelOutput.asyHenv: The asymptotic standard errors for elements in $\beta$ under the heteroscedastic envelope model. An r by p matrix.  The standard errors returned are asymptotic, for actual standard errors, multiply by 1 / sqrt(n).
   \item ModelOutput.ratio: The asymptotic standard error ratio of the standard multivariate linear regression estimator over the heteroscedastic envelope estimator. An r by p matrix, the (i, j)th element in ModelOutput.ratio is the elementwise standard error ratio for the ith element in the jth group mean effect.
   \item ModelOutput.ng: The number of observations in each group.  A p by 1 vector.
\end{itemize}


\subsection*{Description}

\begin{par}
This function fits the heteroscedastic envelope model to the responses and predictors, using the maximum likelihood estimation.  When the dimension of the envelope is between 1 and r-1, we implemented the algorithm in Su and Cook (2012). When the dimension is r, then the envelope model degenerates to the standard multivariate linear model for comparing group means.  When the dimension is 0, it means there is not any group effect, and the fitting is different.
\end{par} \vspace{1em}


\subsection*{References}

\begin{enumerate}
\setlength{\itemsep}{-1ex}
   \item The codes is implemented based on the algorithm in Section 2.2 of Su and Cook (2012).
   \item The Grassmann manifold optimization step calls the package sg\_min 2.4.1 by Ross Lippert (http://web.mit.edu/$\sim$ripper/www.sgmin.html).
\end{enumerate}


\subsection*{Example}

\begin{par}
The following codes produce the results of the water strider example in Su and Cook (2011).
\end{par} \vspace{1em}

\begin{verbatim}       load waterstrider.mat
       u = lrt_henv(X, Y, 0.01)
       \end{verbatim}
               \color{lightgray}\ttfamily \begin{verbatim}
u =

     6
\end{verbatim} \rmfamily
\color{black}
\begin{verbatim}       ModelOutput = henv(X, Y, u)
       ModelOutput.ratio\end{verbatim}

 \color{lightgray}\ttfamily \begin{verbatim}
ModelOutput = 

           mu: [8x1 double]
          mug: [8x3 double]
         Yfit: [90x8 double]
        Gamma: [8x6 double]
       Gamma0: [8x2 double]
         beta: [8x3 double]
     groupInd: [3x2 double]
        Sigma: [8x8x3 double]
          eta: [6x3 double]
        Omega: [6x6x3 double]
       Omega0: [2x2 double]
           np: 98
            l: 1.0051e+03
    covMatrix: [32x32 double]
      asyHenv: [8x3 double]
        ratio: [8x3 double]
           ng: [3x1 double]


ans =

    6.5439   11.2830    6.4954
    4.6325    5.3226    4.7242
    4.4456    5.0741    4.4198
    4.7338    6.2469    5.1937
    8.0377   12.5386    9.4823
    9.5067   11.5974   11.3444
   11.8632   15.6080   12.5611
    6.9792   11.1559   10.1002

\end{verbatim} \rmfamily
\color{black}
    
\newpage
    
\rmfamily
\color{black}\section{lrt\_henv}

\begin{par}
Select the dimension of the envelope subspace using likelihood ratio testing for the heteroscedastic envelope model.
\end{par} \vspace{1em}

\subsection*{Contents}

\begin{itemize}
\setlength{\itemsep}{-1ex}
   \item Syntax
   \item Input
   \item Output
   \item Description
   \item Example
\end{itemize}


\subsection*{Syntax}


\begin{verbatim}       u = lrt_henv(X, Y, alpha)
       u = lrt_henv(X, Y, alpha, Opts)\end{verbatim}
    

\subsection*{Input}

\begin{par}
\textbf{X}: Group indicators. A matrix with n rows.  X can only have p unique  rows, where p is the number of groups. For example, if there are two groups, X can only have 2 different kinds of rows, such as (0, 1) and (1, 0), or (1, 0, 10) and (0, 5, 6).  The number of columns is not restricted, as long as X only has p unique rows.
\end{par} \vspace{1em}
\begin{par}
\textbf{Y}: Multivariate responses. An n by r matrix, r is the number of responses and n is number of observations. The responses must be continuous variables, and r should be greater than p.
\end{par} \vspace{1em}
\begin{par}
\textbf{alpha}: Significance level for testing.  A real number between 0 and 1, often taken at 0.05 or 0.01.
\end{par} \vspace{1em}
\begin{par}
\textbf{Opts}: A list containing the optional input parameter, to control the iterations in sg\_min. If one or several (even all) fields are not defined, the default settings are used.
\end{par} \vspace{1em}
\begin{itemize}
\setlength{\itemsep}{-1ex}
   \item Opts.maxIter: Maximum number of iterations.  Default value: 300.
   \item Opts.ftol: Tolerance parameter for F.  Default value: 1e-10.
   \item Opts.gradtol: Tolerance parameter for dF.  Default value: 1e-7.
   \item Opts.verbose: Flag for print out dimension selection process, logical 0 or 1. Default value: 0.
\end{itemize}


\subsection*{Output}

\begin{par}
\textbf{u}: Dimension of the envelope. An integer between 0 and r.
\end{par} \vspace{1em}


\subsection*{Description}

\begin{par}
This function implements the likelihood ratio testing procedure to select the dimension of the envelope subspace in heteroscedastic envelope model, with pre-specified significance level $\alpha$.
\end{par} \vspace{1em}


\subsection*{Example}


\begin{verbatim}       load waterstrider.mat
       u = lrt_henv(X, Y, 0.01)\end{verbatim}
    
        \color{lightgray}\ttfamily \begin{verbatim}
u =

     6

\end{verbatim} \rmfamily
\color{black}

\newpage

\rmfamily
\color{black}\section{predict\_henv}

\begin{par}
Perform estimation or prediction under the heteroscedastic envelope model.
\end{par} \vspace{1em}

\subsection*{Contents}

\begin{itemize}
\setlength{\itemsep}{-1ex}
   \item Syntax
   \item Input
   \item Output
   \item Description
   \item Example
\end{itemize}


\subsection*{Syntax}


\begin{verbatim}       PredictOutput = predict_henv(ModelOutput, Xnew, infType)\end{verbatim}
    

\subsection*{Input}

\begin{par}
\textbf{ModelOutput}: A list containing the maximum likelihood estimators and other statistics inherited from henv.
\end{par} \vspace{1em}
\begin{par}
\textbf{Xnew}: A group indicator.  It must be a column vector, whose transpose is the same as one of the group indictors from the original data.
\end{par} \vspace{1em}
\begin{par}
\textbf{infType}: A string of characters indicting the inference type, the choices can be 'estimation' or 'prediction'.
\end{par} \vspace{1em}


\subsection*{Output}

\begin{par}
\textbf{PredictOutput}: A list containing the results of the inference.
\end{par} \vspace{1em}
\begin{itemize}
\setlength{\itemsep}{-1ex}
   \item PredictOutput.value: The fitted value or the prediction value evaluated at Xnew. An r by 1 vector.
   \item PredictOutput.covMatrix: The covariance matrix of PredictOutput.value. An r by r matrix.
   \item PredictOutput.SE: The standard error of elements in PredictOutput.value. An r by 1 vector.
\end{itemize}


\subsection*{Description}

\begin{par}
This function evaluates the inner envelope model at new value Xnew.  It can perform estimation: find the group mean for the group indicated by Xnew, or prediction: predict Y for the group indicated by Xnew.  The covariance matrix and the standard errors are also provided.
\end{par} \vspace{1em}


\subsection*{Example}


\begin{verbatim}       load waterstrider.mat
       u = lrt_henv(X, Y, 0.01);
       ModelOutput = henv(X, Y, u);
       ModelOutput.groupInd
       ModelOutput.mug
       Xnew = X(1, :)'
       \end{verbatim}
        \color{lightgray}\ttfamily \begin{verbatim}
  
  ans =

    -1    -1
     0     1
     1     0


ans =

   -1.1417   -1.1267   -1.0845
   -1.4063   -1.4067   -1.3132
   -1.3314   -1.3336   -1.2152
   -0.3113   -0.1839   -0.1736
    0.4003    0.3847    0.3072
    0.4107    0.3753    0.3735
    0.3467    0.3271    0.3179
   -0.1954   -0.2100   -0.3488


Xnew =

     1
     0
\end{verbatim} \rmfamily
\color{black}
          
       \begin{verbatim}
       PredictOutput = predict_henv(ModelOutput, Xnew, 'estimation')
       PredictOutput.value %This is the 3rd group mean
       PredictOutput.SE
\end{verbatim}
    
        \color{lightgray}\ttfamily \begin{verbatim}
PredictOutput = 

        value: [8x1 double]
    covMatrix: [8x8 double]
           SE: [8x1 double]


ans =

   -1.0845
   -1.3132
   -1.2152
   -0.1736
    0.3072
    0.3735
    0.3179
   -0.3488


ans =

    0.0682
    0.0695
    0.0651
    0.0436
    0.0832
    0.0636
    0.0847
    0.0698
\end{verbatim} \rmfamily
\color{black}
       \begin{verbatim}
       PredictOutput = predict_henv(ModelOutput, Xnew, 'prediction')
       PredictOutput.SE\end{verbatim}

        \color{lightgray}\ttfamily \begin{verbatim}

PredictOutput = 

        value: [8x1 double]
    covMatrix: [8x8 double]
           SE: [8x1 double]


ans =

    0.3720
    0.3812
    0.3581
    0.2398
    0.4612
    0.3519
    0.4710
    0.3854

\end{verbatim} \rmfamily
\color{black}
    
\newpage

\rmfamily
\color{black}\section{testcoefficient\_henv}


\begin{verbatim}This function tests the null hypothesis L * beta * R = A versus the
alternative hypothesis L * beta * R ~= A, where beta is estimated under
the heteroscedastic envelope model.\end{verbatim}
    
\subsection*{Contents}

\begin{itemize}
\setlength{\itemsep}{-1ex}
   \item Syntax
   \item Input
   \item Output
   \item Description
   \item Example
\end{itemize}


\subsection*{Syntax}


\begin{verbatim}       TestOutput = testcoefficient_henv(ModelOutput)
       TestOutput = testcoefficient_henv(ModelOutput, TestInput)\end{verbatim}
    

\subsection*{Input}

\begin{par}
\textbf{ModelOutput}: A list containing the maximum likelihood estimators and other statistics inherited from henv.
\end{par} \vspace{1em}
\begin{par}
\textbf{TestInput}: A list that specifies the null hypothesis, including L, R, and A.  If not provided by the user, default values will be used.
\end{par} \vspace{1em}
\begin{itemize}
\setlength{\itemsep}{-1ex}
   \item TestInput.L: The matrix multiplied to $\beta$ on the left.  It is a d1 by r matrix, while d1 is less than or equal to r - 1.  Default value: identity matrix $I_{r}$.
   \item TestInput.R: The matrix multiplied to $\beta$ on the right.  It is a p by d2 matrix, while d2 is less than or equal to p.  Default value: identity matrix $(I_{p-1}, 0_{(p-1)\times 1})^{T}$.  This is because the columns of $\beta$ sum to 0.  Then we cannot use $I_p$ as default.
   \item TestInput.A: The matrix on the right hand side of the equation.  It is a d1 by d2 matrix.  Default value: d1 by d2 zero matrix.
\end{itemize}


\subsection*{Output}

\begin{par}
\textbf{TestOutput}: A list containing test statistics, degrees of freedom for the reference chi-squared distribution, the p-value, and the covariance matrix of vec($L\beta R$).  At the same time, a table is printed out.
\end{par} \vspace{1em}
\begin{itemize}
\setlength{\itemsep}{-1ex}
   \item TestOutput.chisqStatistic: The test statistics. A real number.
   \item TestOutput.df: The degrees of freedom of the reference chi-squared distribution.  A positive integer.
   \item TestOutput.pValue: p-value of the test.  A real number in [0, 1].
   \item TestOutput.covMatrix: The covariance matrix of vec($L\beta R$). A d1 * d2 by d1 * d2 matrix.
\end{itemize}


\subsection*{Description}

\begin{par}
This function tests for hypothesis $H_0: L\beta R = A$, versus $H_\alpha: L\beta R\neq A$.  The $\beta$ is estimated by the heteroscedastic envelope model.  If the user does not specify the values for L, R and A, then the test is equivalent to the standard F test on if all the main group effects are 0.  The test statistics used is vec $(L\beta R - A)$ $\hat{\Sigma}^{-1}$ vec $(L\beta R - A)^{T}$, and the reference distribution is chi-squared distribution with degrees of freedom d1 * d2.
\end{par} \vspace{1em}


\subsection*{Example}


\begin{verbatim}       load waterstrider.mat
       u = lrt_henv(X, Y, 0.01);
       ModelOutput = henv(X, Y, u);
       TestOutout = testcoefficient_henv(ModelOutput);\end{verbatim}

               \color{lightgray}\ttfamily \begin{verbatim}
Test Hypothesis     Chisq Statistic    DF     P-value
------------------------------------------------------------------------
L * beta * R = A         226.256        16       0.0000
------------------------------------------------------------------------
\end{verbatim} \rmfamily
\color{black}
\begin{verbatim}       r = size(Y, 2);
       p = size(ModelOutput.beta, 2);
       TestInput.L = rand(2, r);
       TestInput.R = rand(p, 1);
       TestInput.A = zeros(2, 1);
       TestOutout = testcoefficient_henv(ModelOutput, TestInput);\end{verbatim}
    

        \color{lightgray}\ttfamily \begin{verbatim}

Test Hypothesis     Chisq Statistic    DF     P-value
------------------------------------------------------------------------
L * beta * R = A          23.429         2        0.0000
------------------------------------------------------------------------
\end{verbatim} \rmfamily
\color{black}
    
\newpage

\chapter{ienv}


\rmfamily
\color{black}\section{aic\_ienv}

\begin{par}
Select the dimension of the inner envelope subspace using Akaike information criterion.
\end{par} \vspace{1em}

\subsection*{Contents}

\begin{itemize}
\setlength{\itemsep}{-1ex}
   \item Syntax
   \item Input
   \item Output
   \item Description
   \item Example
\end{itemize}


\subsection*{Syntax}


\begin{verbatim}       u = aic_ienv(X, Y)
       u = aic_ienv(X, Y, Opts)\end{verbatim}
    

\subsection*{Input}

\begin{par}
\textbf{X}: Predictors. An n by p matrix, p is the number of predictors and n is the number of observations. The predictors can be univariate or multivariate, discrete or continuous.
\end{par} \vspace{1em}
\begin{par}
\textbf{Y}: Multivariate responses. An n by r matrix, r is the number of responses. The responses must be continuous variables.
\end{par} \vspace{1em}
\begin{par}
\textbf{Opts}: A list containing the optional input parameter, to control the iterations in sg\_min. If one or several (even all) fields are not defined, the default settings are used.
\end{par} \vspace{1em}
\begin{itemize}
\setlength{\itemsep}{-1ex}
   \item Opts.maxIter: Maximum number of iterations.  Default value: 300.
   \item Opts.ftol: Tolerance parameter for F.  Default value: 1e-10.
   \item Opts.gradtol: Tolerance parameter for dF.  Default value: 1e-7.
   \item Opts.verbose: Flag for print out dimension selection process, logical 0 or 1. Default value: 0.
\end{itemize}


\subsection*{Output}

\begin{par}
\textbf{u}: Dimension of the inner envelope. An integer between 0 and p or equal to r.
\end{par} \vspace{1em}


\subsection*{Description}

\begin{par}
This function implements the Akaike information criteria (AIC) to select the dimension of the inner envelope subspace.
\end{par} \vspace{1em}


\subsection*{Example}


\begin{verbatim}       load irisf.mat
       u = aic_ienv(X, Y)\end{verbatim}
    
        \color{lightgray}\ttfamily \begin{verbatim}
u =

     1

\end{verbatim} \rmfamily
\color{black}

\newpage

\rmfamily
\color{black}\section{bic\_ienv}

\begin{par}
Select the dimension of the inner envelope subspace using Bayesian information criterion.
\end{par} \vspace{1em}

\subsection*{Contents}

\begin{itemize}
\setlength{\itemsep}{-1ex}
   \item Syntax
   \item Input
   \item Output
   \item Description
   \item Example
\end{itemize}


\subsection*{Syntax}


\begin{verbatim}       u = bic_ienv(X, Y)
       u = bic_ienv(X, Y, Opts)\end{verbatim}
    

\subsection*{Input}

\begin{par}
\textbf{X}: Predictors. An n by p matrix, p is the number of predictors and n is the number of observations. The predictors can be univariate or multivariate, discrete or continuous.
\end{par} \vspace{1em}
\begin{par}
\textbf{Y}: Multivariate responses. An n by r matrix, r is the number of responses. The responses must be continuous variables.
\end{par} \vspace{1em}
\begin{par}
\textbf{Opts}: A list containing the optional input parameter, to control the iterations in sg\_min. If one or several (even all) fields are not defined, the default settings are used.
\end{par} \vspace{1em}
\begin{itemize}
\setlength{\itemsep}{-1ex}
   \item Opts.maxIter: Maximum number of iterations.  Default value: 300.
   \item Opts.ftol: Tolerance parameter for F.  Default value: 1e-10.
   \item Opts.gradtol: Tolerance parameter for dF.  Default value: 1e-7.
   \item Opts.verbose: Flag for print out dimension selection process, logical 0 or 1. Default value: 0.
\end{itemize}


\subsection*{Output}

\begin{par}
\textbf{u}: Dimension of the inner envelope. An integer between 0 and p or equal to r.
\end{par} \vspace{1em}


\subsection*{Description}

\begin{par}
This function implements the Bayesian information criteria (BIC) to select the dimension of the inner envelope subspace.
\end{par} \vspace{1em}


\subsection*{Example}


\begin{verbatim}       load irisf.mat
       u = bic_ienv(X, Y)\end{verbatim}
    
        \color{lightgray}\ttfamily \begin{verbatim}
u =

     1

\end{verbatim} \rmfamily
\color{black}

\newpage

\rmfamily
\color{black}\section{bstrp\_ienv}

\begin{par}
Compute bootstrap standard error for the inner envelope model.
\end{par} \vspace{1em}

\subsection*{Contents}

\begin{itemize}
\setlength{\itemsep}{-1ex}
   \item Syntax
   \item Input
   \item Output
   \item Description
   \item Example
\end{itemize}


\subsection*{Syntax}


\begin{verbatim}       bootse = bstrp_ienv(X, Y, u, B)
       bootse = bstrp_ienv(X, Y, u, B, Opts)\end{verbatim}
    

\subsection*{Input}

\begin{par}
\textbf{X}: Predictors, an n by p matrix, p is the number of predictors.  The predictors can be univariate or multivariate, discrete or continuous.
\end{par} \vspace{1em}
\begin{par}
\textbf{Y}: Multivariate responses, an n by r matrix, r is the number of responses and n is number of observations.  The responses must be continuous variables.
\end{par} \vspace{1em}
\begin{par}
\textbf{u}: Dimension of the inner envelope. An integer between 0 and p or equal to r.
\end{par} \vspace{1em}
\begin{par}
\textbf{B}: Number of bootstrap samples.  A positive integer.
\end{par} \vspace{1em}
\begin{par}
\textbf{Opts}: A list containing the optional input parameter, to control the iterations in sg\_min. If one or several (even all) fields are not defined, the default settings are used.
\end{par} \vspace{1em}
\begin{itemize}
\setlength{\itemsep}{-1ex}
   \item Opts.maxIter: Maximum number of iterations.  Default value: 300.
   \item Opts.ftol: Tolerance parameter for F.  Default value: 1e-10.
   \item Opts.gradtol: Tolerance parameter for dF.  Default value: 1e-7.
   \item Opts.verbose: Flag for print out the number of bootstrap samples, logical 0 or 1. Default value: 0.
\end{itemize}


\subsection*{Output}

\begin{par}
\textbf{bootse}: The standard error for elements in $\beta$ computed by bootstrap.  An r by p matrix.
\end{par} \vspace{1em}


\subsection*{Description}

\begin{par}
This function computes the bootstrap standard errors for the regression coefficients in the inner envelope model by bootstrapping the residuals.
\end{par} \vspace{1em}


\subsection*{Example}


\begin{verbatim}       load irisf.mat\end{verbatim}
    
\begin{verbatim}       u = bic_ienv(X, Y)\end{verbatim}
        \color{lightgray}\ttfamily \begin{verbatim}

u =

     1

\end{verbatim} \rmfamily
\color{black}
\begin{verbatim}       B = 100;
       bootse = bstrp_ienv(X, Y, u, B)\end{verbatim}
    
        \color{lightgray}\ttfamily \begin{verbatim}

bootse =

   13.4695    4.9601
    7.4709    2.7315
   14.9316    5.2913
    8.7597    3.0853

\end{verbatim} \rmfamily
\color{black}

\newpage


\rmfamily
\color{black}\section{dF4ienv}

\begin{par}
First derivative of the objective function for computing the inner envelope subspace.
\end{par} \vspace{1em}

\subsection*{Contents}

\begin{itemize}
\setlength{\itemsep}{-1ex}
   \item Syntax
   \item Input
   \item Output
   \item Description
\end{itemize}


\subsection*{Syntax}


\begin{verbatim}       df = dF4ienv(R, DataParameter)\end{verbatim}
    

\subsection*{Input}

\begin{par}
\textbf{R}: An r by u semi-orthogonal matrix, 0 \ensuremath{<} u \ensuremath{<}= p.
\end{par} \vspace{1em}
\begin{par}
\textbf{DataParameter}: A structure that contains the statistics calculated from the data.
\end{par} \vspace{1em}


\subsection*{Output}

\begin{par}
\textbf{df}: The first derivative of the objective function for computing the inner envelope subspace.  An r by u matrix.
\end{par} \vspace{1em}


\subsection*{Description}

\begin{par}
This first derivative of F4ienv obtained by matrix calculus calculations.
\end{par} \vspace{1em}


\newpage

 \rmfamily
\color{black}\section{F4ienv}

\begin{par}
Objective function for computing the inner envelope subspace.
\end{par} \vspace{1em}

\subsection*{Contents}

\begin{itemize}
\setlength{\itemsep}{-1ex}
   \item Syntax
   \item Input
   \item Output
   \item Description
\end{itemize}


\subsection*{Syntax}


\begin{verbatim}       f = F4ienv(R, DataParameter)\end{verbatim}
    

\subsection*{Input}

\begin{par}
\textbf{R}: An r by u semi orthogonal matrix, 0 \ensuremath{<} u \ensuremath{<}= p.
\end{par} \vspace{1em}
\begin{par}
\textbf{DataParameter}: A structure that contains the statistics calculated from the data.
\end{par} \vspace{1em}


\subsection*{Output}

\begin{par}
\textbf{f}: A scalar containing the value of the objective function evaluated at R.
\end{par} \vspace{1em}


\subsection*{Description}

\begin{par}
The objective function is derived in Section 3.3 in Su and Cook (2012) by  using maximum likelihood estimation. The columns of the semi-orthogonal matrix that minimizes this function span the estimated inner envelope subspace.
\end{par} \vspace{1em}

\newpage

\rmfamily
\color{black}\section{ienv}

\begin{par}
Fit the inner envelope model.
\end{par} \vspace{1em}

\subsection*{Contents}

\begin{itemize}
\setlength{\itemsep}{-1ex}
   \item Syntax
   \item Input
   \item Output
   \item Description
   \item References
   \item Example
\end{itemize}


\subsection*{Syntax}


\begin{verbatim}       ModelOutput = ienv(X, Y, u)
       ModelOutput = ienv(X, Y, u, Opts)\end{verbatim}
    

\subsection*{Input}

\begin{par}
\textbf{X}: Predictors. An n by p matrix, p is the number of predictors. The predictors can be univariate or multivariate, discrete or continuous.
\end{par} \vspace{1em}
\begin{par}
\textbf{Y}: Multivariate responses. An n by r matrix, r is the number of responses and n is number of observations. The responses must be continuous variables, and r should be strictly greater than p.
\end{par} \vspace{1em}
\begin{par}
\textbf{u}: Dimension of the inner envelope. An integer between 0 and p or equal to r.
\end{par} \vspace{1em}
\begin{par}
\textbf{Opts}: A list containing the optional input parameter, to control the iterations in sg\_min. If one or several (even all) fields are not defined, the default settings are used.
\end{par} \vspace{1em}
\begin{itemize}
\setlength{\itemsep}{-1ex}
   \item Opts.maxIter: Maximum number of iterations.  Default value: 300.
   \item Opts.ftol: Tolerance parameter for F.  Default value: 1e-10.
   \item Opts.gradtol: Tolerance parameter for dF.  Default value: 1e-7.
   \item Opts.verbose: Flag for print out Grassmann manifold optimization process, logical 0 or 1. Default value: 0.
   \item Opts.init: The initial value for the inner envelope subspace. An r by u matrix. Default value is the one generated by function get\_init.
\end{itemize}


\subsection*{Output}

\begin{par}
\textbf{ModelOutput}: A list that contains the maximum likelihood estimators and some statistics.
\end{par} \vspace{1em}
\begin{itemize}
\setlength{\itemsep}{-1ex}
   \item ModelOutput.beta: The envelope estimator of the regression coefficients $\beta$. An r by p matrix.
   \item ModelOutput.Sigma: The envelope estimator of the error covariance matrix.  An r by r matrix.
   \item ModelOutput.Gamma1: The orthogonal basis of the inner envelope subspace. An r by u semi-orthogonal matrix.
   \item ModelOutput.Gamma0: The orthogonal basis of the complement of the inner envelope subspace.  An r by r-u semi-orthogonal matrix.
   \item ModelOutput.eta1: The transpose of the coordinates of $\beta$ with respect to Gamma1. An p by u matrix.
   \item ModelOutput.B: An (r - u) by (p - u) semi-orthogonal matrix, so that (Gamma, Gamma0 * B) spans $\beta$.
   \item ModelOutput.eta2: The transpose of the coordinates of $\beta$ with respect to Gamma0. An p by (p-u) matrix.
   \item ModelOutput.Omega1: The coordinates of Sigma with respect to Gamma1. An u by u matrix.
   \item ModelOutput.Omega0: The coordinates of Sigma with respect to Gamma0. An r - u by r - u matrix.
   \item ModelOutput.alpha: The estimated intercept in the inner envelope model.  An r by 1 vector.
   \item ModelOutput.l: The maximized log likelihood function.  A real number.
   \item ModelOutput.covMatrix: The asymptotic covariance of vec($\beta$).  An rp by rp matrix.  The covariance matrix returned are asymptotic.  For the actual standard errors, multiply by 1/n.
   \item ModelOutput.asyIenv: Asymptotic standard error for elements in $\beta$ under the inner envelope model.  An r by p matrix.  The standard errors returned are asymptotic, for actual standard errors, multiply by 1 / sqrt(n).
   \item ModelOutput.ratio: The asymptotic standard error ratio of the standard multivariate linear regression estimator over the inner envelope estimator, for each element in $\beta$.  An r by p matrix.
   \item ModelOutput.np: The number of parameters in the inner envelope model.  A positive integer.
   \item ModelOutput.n: The number of observations in the data.  A positive integer.
\end{itemize}


\subsection*{Description}

\begin{par}
This function fits the inner envelope model to the responses and predictors, using the maximum likelihood estimation.  When the dimension of the envelope is between 1 and p-1, we implemented the algorithm in Su and Cook (2012).  When the dimension is p, then the inner envelope model degenerates to the standard multivariate linear regression.  When the dimension is 0, it means that X and Y are uncorrelated, and the fitting is different.
\end{par} \vspace{1em}


\subsection*{References}

\begin{enumerate}
\setlength{\itemsep}{-1ex}
   \item The codes is implemented based on the algorithm in Su and Cook (2012).
   \item The Grassmann manifold optimization step calls the package sg\_min 2.4.1 by Ross Lippert (http://web.mit.edu/$\sim$ripper/www.sgmin.html).
\end{enumerate}


\subsection*{Example}

\begin{par}
The following codes gives the results of the Fisher's iris data example in Su and Cook (2012).
\end{par} \vspace{1em}

\begin{verbatim}       load irisf.mat\end{verbatim}
    
\begin{verbatim}       d = bic_ienv(X, Y)\end{verbatim}
        \color{lightgray}\ttfamily \begin{verbatim}
d =

     1
\end{verbatim} \rmfamily
\color{black}
\begin{verbatim}       ModelOutput = ienv(X, Y, d)
       1 - 1 ./ ModelOutput.ratio\end{verbatim}
        \color{lightgray}\ttfamily \begin{verbatim}

ModelOutput = 

         beta: [4x2 double]
        Sigma: [4x4 double]
       Gamma1: [4x1 double]
       Gamma0: [4x3 double]
            B: [3x1 double]
         eta1: [2x1 double]
         eta2: [2x1 double]
       Omega1: 8.3751
       Omega0: [3x3 double]
        alpha: [4x1 double]
           np: 16
            l: -1.4805e+03
    covMatrix: [8x8 double]
      asyIenv: [4x2 double]
        ratio: [4x2 double]
            n: 150


ans =

    0.0049    0.2122
    0.0020    0.0806
    0.0033    0.1404
    0.0004    0.0138

\end{verbatim} \rmfamily
\color{black}


\newpage


\rmfamily
\color{black}\section{lrt\_ienv}

\begin{par}
Select the dimension of the inner envelope subspace using likelihood ratio testing.
\end{par} \vspace{1em}

\subsection*{Contents}

\begin{itemize}
\setlength{\itemsep}{-1ex}
   \item Syntax
   \item Input
   \item Output
   \item Description
   \item Example
\end{itemize}


\subsection*{Syntax}


\begin{verbatim}       u = lrt_ienv(X, Y, alpha)
       u = lrt_ienv(X, Y, alpha, Opts)\end{verbatim}
    

\subsection*{Input}

\begin{par}
\textbf{X}: Predictors. An n by p matrix, p is the number of predictors. The predictors can be univariate or multivariate, discrete or continuous.
\end{par} \vspace{1em}
\begin{par}
\textbf{Y}: Multivariate responses. An n by r matrix, r is the number of responses and n is number of observations. The responses must be continuous variables.
\end{par} \vspace{1em}
\begin{par}
\textbf{alpha}: Significance level for testing.  A real number between 0 and 1, often taken at 0.05 or 0.01.
\end{par} \vspace{1em}
\begin{par}
\textbf{Opts}: A list containing the optional input parameter, to control the iterations in sg\_min. If one or several (even all) fields are not defined, the default settings are used.
\end{par} \vspace{1em}
\begin{itemize}
\setlength{\itemsep}{-1ex}
   \item Opts.maxIter: Maximum number of iterations.  Default value: 300.
   \item Opts.ftol: Tolerance parameter for F.  Default value: 1e-10.
   \item Opts.gradtol: Tolerance parameter for dF.  Default value: 1e-7.
   \item Opts.verbose: Flag for print out dimension selection process, logical 0 or 1. Default value: 0.
\end{itemize}


\subsection*{Output}

\begin{par}
\textbf{u}: Dimension of the inner envelope. An integer between 0 and p or equal to r.
\end{par} \vspace{1em}


\subsection*{Description}

\begin{par}
This function implements the likelihood ratio testing procedure to select the dimension of the inner envelope subspace, with pre-specified significance level $\alpha$.
\end{par} \vspace{1em}


\subsection*{Example}


\begin{verbatim}       load irisf.mat\end{verbatim}
    
\begin{verbatim}       alpha = 0.01;
       u = lrt_ienv(X, Y, alpha)\end{verbatim}
    
        \color{lightgray}\ttfamily \begin{verbatim}
u =

     1

\end{verbatim} \rmfamily
\color{black}

\newpage

\rmfamily
\color{black}\section{predict\_ienv}

\begin{par}
Perform estimation or prediction under the inner envelope model.
\end{par} \vspace{1em}

\subsection*{Contents}

\begin{itemize}
\setlength{\itemsep}{-1ex}
   \item Syntax
   \item Input
   \item Output
   \item Description
   \item Example
\end{itemize}


\subsection*{Syntax}


\begin{verbatim}       PredictOutput = predict_ienv(ModelOutput, Xnew, infType)\end{verbatim}
    

\subsection*{Input}

\begin{par}
\textbf{ModelOutput}: A list containing the maximum likelihood estimators and other statistics inherited from ienv.
\end{par} \vspace{1em}
\begin{par}
\textbf{Xnew}: The value of X with which to estimate or predict Y.  A p by 1 vector.
\end{par} \vspace{1em}
\begin{par}
\textbf{infType}: A string of characters indicting the inference type, the choices can be 'estimation' or 'prediction'.
\end{par} \vspace{1em}


\subsection*{Output}

\begin{par}
\textbf{PredictOutput}: A list containing the results of the inference.
\end{par} \vspace{1em}
\begin{itemize}
\setlength{\itemsep}{-1ex}
   \item PredictOutput.value: The fitted value or the prediction value evaluated at Xnew. An r by 1 vector.
   \item PredictOutput.covMatrix: The covariance matrix of PredictOutput.value. An r by r matrix.
   \item PredictOutput.SE: The standard error of elements in PredictOutput.value. An r by 1 vector.
\end{itemize}


\subsection*{Description}

\begin{par}
This function evaluates the inner envelope model at new value Xnew.  It can perform estimation: find the fitted value when X = Xnew, or prediction: predict Y when X = Xnew.  The covariance matrix and the standard errors are also provided.
\end{par} \vspace{1em}


\subsection*{Example}


\begin{verbatim}       load irisf.mat
       d = bic_ienv(X, Y);
       ModelOutput = ienv(X, Y, d);
       Xnew = X(1, :)';
       PredictOutput = predict_ienv(ModelOutput, Xnew, 'estimation')
       [PredictOutput.value, Y(1, :)']  % Compare the fitted value with the data
       PredictOutput.SE\end{verbatim}
        \color{lightgray}\ttfamily \begin{verbatim}
PredictOutput = 

        value: [4x1 double]
    covMatrix: [4x4 double]
           SE: [4x1 double]


ans =

   49.9458   51.0000
   34.2592   35.0000
   14.5771   14.0000
    2.4513    2.0000


ans =

    1.0978
    0.7146
    0.9265
    0.4357
\end{verbatim} \rmfamily
\color{black}
\begin{verbatim}       PredictOutput = predict_ienv(ModelOutput, Xnew, 'prediction')
       PredictOutput.SE\end{verbatim}
    
        \color{lightgray}\ttfamily \begin{verbatim}
PredictOutput = 

        value: [4x1 double]
    covMatrix: [4x4 double]
           SE: [4x1 double]


ans =

    5.2197
    3.3897
    4.3996
    2.0642

\end{verbatim} \rmfamily
\color{black}

\newpage

\rmfamily
\color{black}\section{testcoefficient\_ienv}


\begin{verbatim}This function tests the null hypothesis L * beta * R = A versus the
alternative hypothesis L * beta * R ~= A, where beta is estimated under
the inner envelope model.\end{verbatim}
    
\subsection*{Contents}

\begin{itemize}
\setlength{\itemsep}{-1ex}
   \item Syntax
   \item Input
   \item Output
   \item Description
   \item Example
\end{itemize}


\subsection*{Syntax}


\begin{verbatim}       TestOutput = testcoefficient_ienv(ModelOutput)
       TestOutput = testcoefficient_ienv(ModelOutput, TestInput)\end{verbatim}
    

\subsection*{Input}

\begin{par}
\textbf{ModelOutput}: A list containing the maximum likelihood estimators and other statistics inherited from ienv.
\end{par} \vspace{1em}
\begin{par}
\textbf{TestInput}: A list that specifies the null hypothesis, including L, R, and A.  If not provided by the user, default values will be used.
\end{par} \vspace{1em}
\begin{itemize}
\setlength{\itemsep}{-1ex}
   \item TestInput.L: The matrix multiplied to $\beta$ on the left.  It is a d1 by r matrix, while d1 is less than or equal to r.  Default value: identity matrix $I_r$.
   \item TestInput.R: The matrix multiplied to $\beta$ on the right.  It is a p by d2 matrix, while d2 is less than or equal to p.  Default value: identity matrix $I_p$.
   \item TestInput.A: The matrix on the right hand side of the equation.  It is a d1 by d2 matrix.  Default value: d1 by d2 zero matrix.
\end{itemize}


\subsection*{Output}

\begin{par}
\textbf{TestOutput}: A list containing test statistics, degrees of freedom for the reference chi-squared distribution, the p-value, and the covariance matrix of vec($L\beta R$).  At the same time, a table is printed out.
\end{par} \vspace{1em}
\begin{itemize}
\setlength{\itemsep}{-1ex}
   \item TestOutput.chisqStatistic: The test statistics. A real number.
   \item TestOutput.df: The degrees of freedom of the reference chi-squared distribution.  A positive integer.
   \item TestOutput.pValue: p-value of the test.  A real number in [0, 1].
   \item TestOutput.covMatrix: The covariance matrix of vec($L\beta R$). A d1 * d2 by d1 * d2 matrix.
\end{itemize}

\subsection*{Description}

\begin{par}
This function tests for hypothesis $H_0: L\beta R = A$, versus $H_\alpha: L\beta R\neq A$.  The $\beta$ is estimated by the inner envelope model.  If the user does not specify the values for L, R and A, then the test is equivalent to the standard F test on if $\beta = 0$.  The test statistics used is vec $(L\beta R - A)$ $\hat{\Sigma}^{-1}$ vec $(L\beta R - A)^{T}$, and the reference distribution is chi-squared distribution with degrees of freedom d1 * d2.
\end{par} \vspace{1em}


\subsection*{Example}


\begin{verbatim}       load irisf.mat
       d = bic_ienv(X,Y);
       ModelOutput = ienv(X,Y,d);
       TestOutout = testcoefficient_ienv(ModelOutput);\end{verbatim}
       
        \color{lightgray}\ttfamily \begin{verbatim}
Test Hypothesis     Chisq Statistic    DF     P-value
------------------------------------------------------------------------
L * beta * R = A        4642.913        8       0.0000
------------------------------------------------------------------------
\end{verbatim} \rmfamily
\color{black}
\begin{verbatim}       TestInput.L = rand(2, 4);
       TestInput.R = rand(2, 1);
       TestInput.A = zeros(2, 1);
       TestOutout = testcoefficient_ienv(ModelOutput, TestInput);\end{verbatim}
    
        \color{lightgray}\ttfamily \begin{verbatim}
Test Hypothesis     Chisq Statistic    DF     P-value
------------------------------------------------------------------------
L * beta * R = A        1834.229        2       0.0000
------------------------------------------------------------------------
\end{verbatim} \rmfamily
\color{black}
    
\newpage 
    
\chapter{penv}

\rmfamily
\color{black}\section{aic\_penv}

\begin{par}
Select the dimension of the partial envelope subspace using Akaike information criterion.
\end{par} \vspace{1em}

\subsection*{Contents}

\begin{itemize}
\setlength{\itemsep}{-1ex}
   \item Syntax
   \item Input
   \item Output
   \item Description
   \item Example
\end{itemize}


\subsection*{Syntax}


\begin{verbatim}       u = aic_penv(X, Y)
       u = aic_penv(X, Y, Opts)\end{verbatim}
    

\subsection*{Input}

\begin{par}
\textbf{X}: A list containing the value of X1 and X2.
\end{par} \vspace{1em}
\begin{itemize}
\setlength{\itemsep}{-1ex}
   \item X.X1: Predictors of main interest. An n by p1 matrix, n is the number of observations, and p1 is the number of main predictors. The predictors can be univariate or multivariate, discrete or continuous.
   \item X.X2: Covariates, or predictors not of main interest.  An n by p2 matrix, p2 is the number of covariates.
\end{itemize}
\begin{par}
\textbf{Y}: Multivariate responses. An n by r matrix, r is the number of responses and n is number of observations. The responses must be continuous variables.
\end{par} \vspace{1em}
\begin{par}
\textbf{Opts}: A list containing the optional input parameter, to control the iterations in sg\_min. If one or several (even all) fields are not defined, the default settings are used.
\end{par} \vspace{1em}
\begin{itemize}
\setlength{\itemsep}{-1ex}
   \item Opts.maxIter: Maximum number of iterations.  Default value: 300.
   \item Opts.ftol: Tolerance parameter for F.  Default value: 1e-10.
   \item Opts.gradtol: Tolerance parameter for dF.  Default value: 1e-7.
   \item Opts.verbose: Flag for print out dimension selection process, logical 0 or 1. Default value: 0.
\end{itemize}


\subsection*{Output}

\begin{par}
\textbf{u}: Dimension of the envelope. An integer between 0 and r.
\end{par} \vspace{1em}


\subsection*{Description}

\begin{par}
This function implements the Akaike information criteria (AIC) to select the dimension of the partial envelope subspace.
\end{par} \vspace{1em}


\subsection*{Example}


\begin{verbatim}       load fiberpaper.dat
       Y = fiberpaper(:, 1 : 4);
       Xtemp = fiberpaper(:, 5 : 7);
       X.X1 = Xtemp(:, 3);
       X.X2 = Xtemp(:, 1 : 2);
       u = aic_penv(X, Y)\end{verbatim}
    
        \color{lightgray}\ttfamily \begin{verbatim}
u =

     3

\end{verbatim} \rmfamily
\color{black}
    

\newpage

\rmfamily
\color{black}\section{bic\_penv}

\begin{par}
Select the dimension of the partial envelope subspace using Bayesian information criterion.
\end{par} \vspace{1em}

\subsection*{Contents}

\begin{itemize}
\setlength{\itemsep}{-1ex}
   \item Syntax
   \item Input
   \item Output
   \item Description
   \item Example
\end{itemize}


\subsection*{Syntax}


\begin{verbatim}       u = bic_penv(X, Y)
       u = bic_penv(X, Y, Opts)\end{verbatim}
    

\subsection*{Input}

\begin{par}
\textbf{X}: A list containing the value of X1 and X2.
\end{par} \vspace{1em}
\begin{itemize}
\setlength{\itemsep}{-1ex}
   \item X.X1: Predictors of main interest. An n by p1 matrix, n is the number of observations, and p1 is the number of main predictors. The predictors can be univariate or multivariate, discrete or continuous.
   \item X.X2: Covariates, or predictors not of main interest.  An n by p2 matrix, p2 is the number of covariates.
\end{itemize}
\begin{par}
\textbf{Y}: Multivariate responses. An n by r matrix, r is the number of responses and n is number of observations. The responses must be continuous variables.
\end{par} \vspace{1em}
\begin{par}
\textbf{Opts}: A list containing the optional input parameter, to control the iterations in sg\_min. If one or several (even all) fields are not defined, the default settings are used.
\end{par} \vspace{1em}
\begin{itemize}
\setlength{\itemsep}{-1ex}
   \item Opts.maxIter: Maximum number of iterations.  Default value: 300.
   \item Opts.ftol: Tolerance parameter for F.  Default value: 1e-10.
   \item Opts.gradtol: Tolerance parameter for dF.  Default value: 1e-7.
   \item Opts.verbose: Flag for print out dimension selection process, logical 0 or 1. Default value: 0.
\end{itemize}


\subsection*{Output}

\begin{par}
\textbf{u}: Dimension of the envelope. An integer between 0 and r.
\end{par} \vspace{1em}


\subsection*{Description}

\begin{par}
This function implements the Bayesian information criteria (BIC) to select the dimension of the partial envelope subspace.
\end{par} \vspace{1em}


\subsection*{Example}


\begin{verbatim}       load fiberpaper.dat
       Y = fiberpaper(:, 1 : 4);
       Xtemp = fiberpaper(:, 5 : 7);
       X.X1 = Xtemp(:, 3);
       X.X2 = Xtemp(:, 1 : 2);
       u = bic_penv(X, Y)\end{verbatim}
    
        \color{lightgray}\ttfamily \begin{verbatim}
u =

     1

\end{verbatim} \rmfamily
\color{black}
    
\newpage


\rmfamily
\color{black}\section{bstrp\_penv}

\begin{par}
Compute bootstrap standard error for the partial envelope model.
\end{par} \vspace{1em}

\subsection*{Contents}

\begin{itemize}
\setlength{\itemsep}{-1ex}
   \item Syntax
   \item Input
   \item Output
   \item Description
   \item Example
\end{itemize}


\subsection*{Syntax}


\begin{verbatim}       bootse = bstrp_penv(X, Y, u, B)
       bootse = bstrp_penv(X, Y, u, B, Opts)\end{verbatim}
    

\subsection*{Input}

\begin{par}
\textbf{X}: A list containing the value of X1 and X2.
\end{par} \vspace{1em}
\begin{itemize}
\setlength{\itemsep}{-1ex}
   \item X.X1: Predictors of main interest. An n by p1 matrix, n is the number of observations, and p1 is the number of main predictors. The predictors can be univariate or multivariate, discrete or continuous.
   \item X.X2: Covariates, or predictors not of main interest.  An n by p2 matrix, p2 is the number of covariates.
\end{itemize}
\begin{par}
\textbf{Y}: Multivariate responses, an n by r matrix, r is the number of responses and n is number of observations.  The responses must be continuous variables.
\end{par} \vspace{1em}
\begin{par}
\textbf{u}: Dimension of the partial envelope subspace.  A positive integer between 0 and r.
\end{par} \vspace{1em}
\begin{par}
\textbf{B}: Number of bootstrap samples.  A positive integer.
\end{par} \vspace{1em}
\begin{par}
\textbf{Opts}: A list containing the optional input parameter, to control the iterations in sg\_min. If one or several (even all) fields are not defined, the default settings are used.
\end{par} \vspace{1em}
\begin{itemize}
\setlength{\itemsep}{-1ex}
   \item Opts.maxIter: Maximum number of iterations.  Default value: 300.
   \item Opts.ftol: Tolerance parameter for F.  Default value: 1e-10.
   \item Opts.gradtol: Tolerance parameter for dF.  Default value: 1e-7.
   \item Opts.verbose: Flag for print out the number of bootstrap samples, logical 0 or 1. Default value: 0.
\end{itemize}


\subsection*{Output}

\begin{par}
\textbf{bootse}: The standard error for elements in $\beta_1$ computed by bootstrap.  An r by p1 matrix.
\end{par} \vspace{1em}


\subsection*{Description}

\begin{par}
This function computes the bootstrap standard errors for the regression coefficients in the partial envelope model by bootstrapping the residuals.
\end{par} \vspace{1em}


\subsection*{Example}


\begin{verbatim}       load fiberpaper.dat
       Y = fiberpaper(:, 1 : 4);
       Xtemp = fiberpaper(:, 5 : 7);
       X.X1 = Xtemp(:, 3);
       X.X2 = Xtemp(:, 1 : 2);
       alpha = 0.01;
       u = lrt_penv(X, Y, alpha)\end{verbatim}
               \color{lightgray}\ttfamily \begin{verbatim}
u =

     1
\end{verbatim} \rmfamily
\color{black}
\begin{verbatim}       B = 100;
       bootse = bstrp_penv(X, Y, u, B)\end{verbatim}
    
        \color{lightgray}\ttfamily \begin{verbatim}
bootse =

    0.0074
    0.0021
    0.0043
    0.0019

\end{verbatim} \rmfamily
\color{black}
    

\newpage


\rmfamily
\color{black}\section{lrt\_penv}

\begin{par}
Select the dimension of the partial envelope subspace using likelihood ratio testing.
\end{par} \vspace{1em}

\subsection*{Contents}

\begin{itemize}
\setlength{\itemsep}{-1ex}
   \item Syntax
   \item Input
   \item Output
   \item Description
   \item Example
\end{itemize}


\subsection*{Syntax}


\begin{verbatim}       u = lrt_penv(X, Y, alpha)
       u = lrt_penv(X, Y, alpha, Opts)\end{verbatim}
    

\subsection*{Input}

\begin{par}
\textbf{X}: A list containing the value of X1 and X2.
\end{par} \vspace{1em}
\begin{itemize}
\setlength{\itemsep}{-1ex}
   \item X.X1: Predictors of main interest. An n by p1 matrix, n is the number of observations, and p1 is the number of main predictors. The predictors can be univariate or multivariate, discrete or continuous.
   \item X.X2: Covariates, or predictors not of main interest.  An n by p2 matrix, p2 is the number of covariates.
\end{itemize}
\begin{par}
\textbf{Y}: Multivariate responses. An n by r matrix, r is the number of responses and n is number of observations. The responses must be continuous variables.
\end{par} \vspace{1em}
\begin{par}
\textbf{alpha}: Significance level for testing.  A real number between 0 and 1, often taken at 0.05 or 0.01.
\end{par} \vspace{1em}
\begin{par}
\textbf{Opts}: A list containing the optional input parameter, to control the iterations in sg\_min. If one or several (even all) fields are not defined, the default settings are used.
\end{par} \vspace{1em}
\begin{itemize}
\setlength{\itemsep}{-1ex}
   \item Opts.maxIter: Maximum number of iterations.  Default value: 300.
   \item Opts.ftol: Tolerance parameter for F.  Default value: 1e-10.
   \item Opts.gradtol: Tolerance parameter for dF.  Default value: 1e-7.
   \item Opts.verbose: Flag for print out dimension selection process, logical 0 or 1. Default value: 0.
\end{itemize}


\subsection*{Output}

\begin{par}
\textbf{u}: Dimension of the partial envelope subspace. An integer between 0 and r.
\end{par} \vspace{1em}


\subsection*{Description}

\begin{par}
This function implements the likelihood ratio testing procedure to select the dimension of the partial envelope subspace, with pre-specified significance level $\alpha$.
\end{par} \vspace{1em}


\subsection*{Example}


\begin{verbatim}       load fiberpaper.dat
       Y = fiberpaper(:, 1 : 4);
       Xtemp = fiberpaper(:, 5 : 7);
       X.X1 = Xtemp(:, 3);
       X.X2 = Xtemp(:, 1 : 2);
       alpha = 0.01;
       u = lrt_penv(X, Y, alpha)\end{verbatim}
    
        \color{lightgray}\ttfamily \begin{verbatim}
u =

     1

\end{verbatim} \rmfamily
\color{black}


\newpage

\rmfamily
\color{black}\section{penv}

\begin{par}
Fit the partial envelope model.
\end{par} \vspace{1em}

\subsection*{Contents}

\begin{itemize}
\setlength{\itemsep}{-1ex}
   \item Syntax
   \item Input
   \item Output
   \item Description
   \item References
   \item Example
\end{itemize}


\subsection*{Syntax}


\begin{verbatim}       ModelOutput = penv(X, Y, u)
       ModelOutput = penv(X, Y, u, Opts)\end{verbatim}
    

\subsection*{Input}

\begin{par}
\textbf{X}: A list containing the value of X1 and X2.
\end{par} \vspace{1em}
\begin{itemize}
\setlength{\itemsep}{-1ex}
   \item X.X1: Predictors of main interest. An n by p1 matrix, n is the number of observations, and p1 is the number of main predictors. The predictors can be univariate or multivariate, discrete or continuous.
   \item X.X2: Covariates, or predictors not of main interest.  An n by p2 matrix, p2 is the number of covariates.
\end{itemize}
\begin{par}
\textbf{Y}: Multivariate responses. An n by r matrix, r is the number of responses and n is number of observations. The responses must be continuous variables, and r should be strictly greater than p1.
\end{par} \vspace{1em}
\begin{par}
\textbf{u}: Dimension of the partial envelope. An integer between 0 and r.
\end{par} \vspace{1em}
\begin{par}
\textbf{Opts}: A list containing the optional input parameter, to control the iterations in sg\_min. If one or several (even all) fields are not defined, the default settings are used.
\end{par} \vspace{1em}
\begin{itemize}
\setlength{\itemsep}{-1ex}
   \item Opts.maxIter: Maximum number of iterations.  Default value: 300.
   \item Opts.ftol: Tolerance parameter for F.  Default value: 1e-10.
   \item Opts.gradtol: Tolerance parameter for dF.  Default value: 1e-7.
   \item Opts.verbose: Flag for print out Grassmann manifold optimization process, logical 0 or 1. Default value: 0.
   \item Opts.init: The initial value for the partial envelope subspace. An r by u matrix. Default value is the one generated by function get\_init.
\end{itemize}


\subsection*{Output}

\begin{par}
\textbf{ModelOutput}: A list that contains the maximum likelihood estimators and some statistics.
\end{par} \vspace{1em}
\begin{itemize}
\setlength{\itemsep}{-1ex}
   \item ModelOutput.beta1: The partial envelope estimator of $\beta_1$, which is the regression coefficients for X1. An r by p1 matrix.
   \item ModelOutput.beta2: The partial envelope estimator of $\beta_2$, which is the regression coefficients for X2. An r by p2 matrix.
   \item ModelOutput.Sigma: The partial envelope estimator of the error covariance matrix.  An r by r matrix.
   \item ModelOutput.Gamma: The orthogonal basis of the partial envelope subspace. An r by u semi-orthogonal matrix.
   \item ModelOutput.Gamma0: The orthogonal basis of the complement of the partial envelope subspace.  An r by r - u semi-orthogonal matrix.
   \item ModelOutput.eta: The coordinates of $\beta_1$ with respect to Gamma. An u by p1 matrix.
   \item ModelOutput.Omega: The coordinates of Sigma with respect to Gamma. An u by u matrix.
   \item ModelOutput.Omega0: The coordinates of Sigma with respect to Gamma0. An r - u by r - u matrix.
   \item ModelOutput.alpha: The estimated intercept in the partial envelope model.  An r by 1 vector.
   \item ModelOutput.l: The maximized log likelihood function.  A real number.
   \item ModelOutput.covMatrix: The asymptotic covariance of (vec($\beta_2$)', vec($\beta_1$)')'.  An rp by rp matrix.  The covariance matrix returned are asymptotic.  For the actual standard errors, multiply by 1/n.
   \item ModelOutput.asyPenv: Asymptotic standard error for elements in $\beta_1$ under the partial envelope model.  An r by p1 matrix.  The standard errors returned are asymptotic, for actual standard errors, multiply by 1/sqrt(n).
   \item ModelOutput.ratio: The asymptotic standard error ratio of the standard multivariate linear regression estimator over the partial envelope estimator, for each element in $\beta_1$.  An r by p1 matrix.
   \item ModelOutput.np: The number of parameters in the envelope model.  A positive integer.
   \item ModelOutput.n: The number of observations in the data.  A positive integer.
\end{itemize}


\subsection*{Description}

\begin{par}
This function fits the partial envelope model to the responses Y and predictors X1 and X2, using the maximum likelihood estimation.  When the dimension of the envelope is between 1 and r - 1, we implemented the algorithm in Su and Cook (2011).  When the dimension is r, then the partial envelope model degenerates to the standard multivariate linear regression with Y as the responses and both X1 and X2 as predictors.  When the dimension is 0, X1 and Y are uncorrelated, and the fitting is the standard multivariate linear regression with Y as the responses and X2 as the predictors.
\end{par} \vspace{1em}


\subsection*{References}

\begin{enumerate}
\setlength{\itemsep}{-1ex}
   \item The codes is implemented based on the algorithm in Section 3.2 of Su and Cook (2012).
   \item The Grassmann manifold optimization step calls the package sg\_min 2.4.1 by Ross Lippert (http://web.mit.edu/$\sim$ripper/www.sgmin.html).
\end{enumerate}


\subsection*{Example}

\begin{par}
The following codes reconstruct the results of the paper and fiber example in Su and Cook (2012).
\end{par} \vspace{1em}

\begin{verbatim}       load fiberpaper.dat
       Y = fiberpaper(:, 1 : 4);
       Xtemp = fiberpaper(:, 5 : 7);
       X.X1 = Xtemp(:, 3);
       X.X2 = Xtemp(:, 1 : 2);
       alpha = 0.01;
       u = lrt_penv(X, Y, alpha)\end{verbatim}
        \color{lightgray}\ttfamily \begin{verbatim}
u =

     1
\end{verbatim} \rmfamily
\color{black}
       \begin{verbatim}
       ModelOutput = penv(X, Y, u)
       ModelOutput.Omega
       eig(ModelOutput.Omega0)
       ModelOutput.ratio\end{verbatim}
    
        \color{lightgray}\ttfamily \begin{verbatim}
ModelOutput = 

        beta1: [4x1 double]
        beta2: [4x2 double]
        alpha: [4x1 double]
        Gamma: [4x1 double]
          eta: 0.0047
        Omega: 0.0149
       Omega0: [3x3 double]
        Sigma: [4x4 double]
            l: -35.6323
           np: 23
    covMatrix: [12x12 double]
      asyPenv: [4x1 double]
        ratio: [4x1 double]
            n: 62


ans =

    0.0149


ans =

    4.9819
    0.0999
    0.0050


ans =

   66.0742
    6.9326
   10.5048
    9.6279

\end{verbatim} \rmfamily
\color{black}

\newpage

\rmfamily
\color{black}\section{predict\_penv}

\begin{par}
Perform estimation or prediction under the partial envelope model.
\end{par} \vspace{1em}

\subsection*{Contents}

\begin{itemize}
\setlength{\itemsep}{-1ex}
   \item Syntax
   \item Input
   \item Output
   \item Description
   \item Example
\end{itemize}


\subsection*{Syntax}

\begin{par}
PredictOutput = predict\_penv(ModelOutput, Xnew, infType)
\end{par} \vspace{1em}


\subsection*{Input}

\begin{par}
\textbf{ModelOutput}: A list containing the maximum likelihood estimators and other statistics inherited from penv.
\end{par} \vspace{1em}
\begin{par}
\textbf{Xnew}: A list containing the value of X1 and X2 with which to estimate or predict Y.
\end{par} \vspace{1em}

\begin{verbatim}* Xnew.X1: A p1 by 1 vector containing the value of X1.
* Xnew.X2: A p2 by 1 vector containing the value of X2.\end{verbatim}
    \begin{par}
\textbf{infType}: A string of characters indicting the inference type, the choices can be 'estimation' or 'prediction'.
\end{par} \vspace{1em}


\subsection*{Output}

\begin{par}
\textbf{PredictOutput}: A list containing the results of the inference.
\end{par} \vspace{1em}
\begin{itemize}
\setlength{\itemsep}{-1ex}
   \item PredictOutput.value: The fitted value or the prediction value evaluated at Xnew. An r by 1 vector.
   \item PredictOutput.covMatrix: The covariance matrix of PredictOutput.value. An r by r matrix.
   \item PredictOutput.SE: The standard error of elements in PredictOutput.value. An r by 1 vector.
\end{itemize}


\subsection*{Description}

\begin{par}
This function evaluates the envelope model at new value Xnew.  It can perform estimation: find the fitted value when X = Xnew, or prediction: predict Y when X = Xnew.  The covariance matrix and the standard errors are also provided.
\end{par} \vspace{1em}


\subsection*{Example}


\begin{verbatim}       load fiberpaper.dat
       Y = fiberpaper(:, 1 : 4);
       Xtemp = fiberpaper(:, 5 : 7);
       X.X1 = Xtemp(:, 3);
       X.X2 = Xtemp(:, 1 : 2);
       alpha = 0.01;
       u = lrt_penv(X, Y, alpha);
       ModelOutput = penv(X, Y, u);
       Xnew.X1 = X.X1(1, :)';
       Xnew.X2 = X.X2(1, :)';
       PredictOutput = predict_penv(ModelOutput, Xnew, 'estimation')
       [PredictOutput.value, Y(1, :)'] % Compare the fitted value with the data
       PredictOutput.SE\end{verbatim}
    
        \color{lightgray}\ttfamily \begin{verbatim}
PredictOutput = 

        value: [4x1 double]
    covMatrix: [4x4 double]
           SE: [4x1 double]


ans =

   21.1169   21.3120
    7.1173    7.0390
    5.3637    5.3260
    0.8737    0.9320


ans =

    1.4680
    0.4234
    0.7145
    0.3161
\end{verbatim} \rmfamily
\color{black}

       \begin{verbatim}
       PredictOutput = predict_penv(ModelOutput, Xnew, 'prediction')
       PredictOutput.SE\end{verbatim}
        \color{lightgray}\ttfamily \begin{verbatim}
PredictOutput = 

        value: [4x1 double]
    covMatrix: [4x4 double]
           SE: [4x1 double]


ans =

    2.4277
    0.6982
    1.1802
    0.5220

\end{verbatim} \rmfamily
\color{black}

\newpage

\rmfamily
\color{black}\section{testcoefficient\_penv}


\begin{verbatim}This function tests the null hypothesis L * beta1 * R = A versus the
alternative hypothesis L * beta1 * R ~= A, where beta1 is estimated under
the envelope model.\end{verbatim}
    
\subsection*{Contents}

\begin{itemize}
\setlength{\itemsep}{-1ex}
   \item Syntax
   \item Input
   \item Output
   \item Description
   \item Example
\end{itemize}


\subsection*{Syntax}


\begin{verbatim}       TestOutput = testcoefficient_penv(ModelOutput)
       TestOutput = testcoefficient_penv(ModelOutput, TestInput)\end{verbatim}
    

\subsection*{Input}

\begin{par}
\textbf{ModelOutput}: A list containing the maximum likelihood estimators and other statistics inherited from penv.
\end{par} \vspace{1em}
\begin{par}
\textbf{TestInput}: A list that specifies the null hypothesis, including L, R, and A.  If not provided by the user, default values will be used.
\end{par} \vspace{1em}
\begin{itemize}
\setlength{\itemsep}{-1ex}
   \item TestInput.L: The matrix multiplied to $\beta_1$ on the left.  It is a d1 by r matrix, while d1 is less than or equal to r.  Default value: identity matrix $I_r$.
   \item TestInput.R: The matrix multiplied to $\beta_1$ on the right.  It is a p1 by d2 matrix, while d2 is less than or equal to p1.  Default value: identity matrix $I_{p1}$.
   \item TestInput.A: The matrix on the right hand side of the equation.  It is a d1 by d2 matrix.  Default value: d1 by d2 zero matrix.
\end{itemize}


\subsection*{Output}

\begin{par}
\textbf{TestOutput}: A list containing test statistics, degrees of freedom for the reference chi-squared distribution, the p-value, and the covariance matrix of vec($L\beta R$).  At the same time, a table is printed out.
\end{par} \vspace{1em}
\begin{itemize}
\setlength{\itemsep}{-1ex}
   \item TestOutput.chisqStatistic: The test statistics. A real number.
   \item TestOutput.df: The degrees of freedom of the reference chi-squared distribution.  A positive integer.
   \item TestOutput.pValue: p-value of the test.  A real number in [0, 1].
   \item TestOutput.covMatrix: The covariance matrix of vec($L\beta_1 R$). A d1 * d2 by d1 * d2 matrix.
\end{itemize}

\subsection*{Description}

\begin{par}
This function tests for hypothesis $H_0: L\beta_1 R = A$, versus $H_\alpha: L\beta_1 R\neq A$.  The $\beta_1$ is estimated by the partial envelope model.  If the user does not specify the values for L, R and A, then the test is equivalent to the standard F test on if $\beta_1 = 0$.  The test statistics used is vec $(L\beta_1 R - A)$ $\hat{\Sigma}^{-1}$ vec $(L\beta_1 R - A)^{T}$, and the reference distribution is chi-squared distribution with degrees of freedom d1 * d2.
\end{par} \vspace{1em}


\subsection*{Example}


\begin{verbatim}       load fiberpaper.dat
       Y = fiberpaper(:, 1 : 4);
       Xtemp = fiberpaper(:, 5 : 7);
       X.X1 = Xtemp(:, 3);
       X.X2 = Xtemp(:, 1 : 2);
       alpha = 0.01;
       u = lrt_penv(X, Y, alpha);
       ModelOutput = penv(X, Y, u);
       TestOutout = testcoefficient_penv(ModelOutput);\end{verbatim}
        \color{lightgray}\ttfamily \begin{verbatim}
 Test Hypothesis     Chisq Statistic    DF     P-value
------------------------------------------------------------------------
L * beta * R = A          12.604         4        0.0134
------------------------------------------------------------------------
\end{verbatim} \rmfamily
\color{black}
\begin{verbatim}
       r = size(Y, 2);
       p1 = size(X.X1, 2);
       TestInput.L = rand(2, r);
       TestInput.R = rand(p1, 1);
       TestInput.A = zeros(2, 1);
       TestOutout = testcoefficient_penv(ModelOutput, TestInput);\end{verbatim}
   

         \color{lightgray}\ttfamily \begin{verbatim}
Test Hypothesis     Chisq Statistic    DF     P-value
------------------------------------------------------------------------
L * beta * R = A          11.452         2        0.0033
------------------------------------------------------------------------
\end{verbatim} \rmfamily
\color{black}

\newpage
\chapter{senv}
    
    
\rmfamily
\color{black}\section{aic\_senv}

\begin{par}
Select the dimension of the scaled envelope subspace using Akaike information criterion.
\end{par} \vspace{1em}

\subsection*{Contents}

\begin{itemize}
\setlength{\itemsep}{-1ex}
   \item Syntax
   \item Input
   \item Output
   \item Description
   \item Example
\end{itemize}


\subsection*{Syntax}


\begin{verbatim}       u = aic_senv(X, Y)
       u = aic_senv(X, Y, Opts)\end{verbatim}
    

\subsection*{Input}

\begin{par}
\textbf{X}: Predictors. An n by p matrix, p is the number of predictors and n is the number of observations. The predictors can be univariate or multivariate, discrete or continuous.
\end{par} \vspace{1em}
\begin{par}
\textbf{Y}: Multivariate responses. An n by r matrix, r is the number of responses. The responses must be continuous variables.
\end{par} \vspace{1em}
\begin{par}
\textbf{Opts}: A list containing the optional input parameter, to control the iterations in sg\_min. If one or several (even all) fields are not defined, the default settings are used.
\end{par} \vspace{1em}
\begin{itemize}
\setlength{\itemsep}{-1ex}
   \item Opts.maxIter: Maximum number of iterations.  Default value: 300.
   \item Opts.ftol: Tolerance parameter for F.  Default value: 1e-10.
   \item Opts.gradtol: Tolerance parameter for dF.  Default value: 1e-7.
   \item Opts.verbose: Flag for print out dimension selection process, logical 0 or 1. Default value: 0.
\end{itemize}


\subsection*{Output}

\begin{par}
\textbf{u}: Dimension of the inner envelope. An integer between 0 and r.
\end{par} \vspace{1em}


\subsection*{Description}

\begin{par}
This function implements the Akaike information criteria (AIC) to select the dimension of the scaled envelope subspace.
\end{par} \vspace{1em}


\subsection*{Example}


\begin{verbatim}       load('sales.txt')
       Y = sales(:, 4 : 7);
       X = sales(:, 1 : 3);
       u = aic_senv(X, Y)\end{verbatim}
    
        \color{lightgray}\ttfamily \begin{verbatim}
u =

     4

\end{verbatim} \rmfamily
\color{black}

\newpage


\rmfamily
\color{black}\section{bic\_senv}

\begin{par}
Select the dimension of the scaled envelope subspace using Bayesian information criterion.
\end{par} \vspace{1em}

\subsection*{Contents}

\begin{itemize}
\setlength{\itemsep}{-1ex}
   \item Syntax
   \item Input
   \item Output
   \item Description
   \item Example
\end{itemize}


\subsection*{Syntax}


\begin{verbatim}       u = bic_senv(X, Y)
       u = bic_senv(X, Y, Opts)\end{verbatim}
    

\subsection*{Input}

\begin{par}
\textbf{X}: Predictors. An n by p matrix, p is the number of predictors and n is the number of observations. The predictors can be univariate or multivariate, discrete or continuous.
\end{par} \vspace{1em}
\begin{par}
\textbf{Y}: Multivariate responses. An n by r matrix, r is the number of responses. The responses must be continuous variables.
\end{par} \vspace{1em}
\begin{par}
\textbf{Opts}: A list containing the optional input parameter, to control the iterations in sg\_min. If one or several (even all) fields are not defined, the default settings are used.
\end{par} \vspace{1em}
\begin{itemize}
\setlength{\itemsep}{-1ex}
   \item Opts.maxIter: Maximum number of iterations.  Default value: 300.
   \item Opts.ftol: Tolerance parameter for F.  Default value: 1e-10.
   \item Opts.gradtol: Tolerance parameter for dF.  Default value: 1e-7.
   \item Opts.verbose: Flag for print out dimension selection process, logical 0 or 1. Default value: 0.
\end{itemize}


\subsection*{Output}

\begin{par}
\textbf{u}: Dimension of the inner envelope. An integer between 0 and r.
\end{par} \vspace{1em}


\subsection*{Description}

\begin{par}
This function implements the Bayesian information criteria (BIC) to select the dimension of the scaled envelope subspace.
\end{par} \vspace{1em}


\subsection*{Example}


\begin{verbatim}       load('sales.txt')
       Y = sales(:, 4 : 7);
       X = sales(:, 1 : 3);
       u = bic_senv(X, Y)\end{verbatim}
    
        \color{lightgray}\ttfamily \begin{verbatim}

u =

     2

\end{verbatim} \rmfamily
\color{black}
\newpage

\rmfamily
\color{black}\section{bstrp\_senv}

\begin{par}
Compute bootstrap standard error for the scaled envelope model.
\end{par} \vspace{1em}

\subsection*{Contents}

\begin{itemize}
\setlength{\itemsep}{-1ex}
   \item Syntax
   \item Input
   \item Output
   \item Description
   \item Example
\end{itemize}


\subsection*{Syntax}


\begin{verbatim}       bootse = bstrp_senv(X, Y, u, B)
       bootse = bstrp_senv(X, Y, u, B, Opts)\end{verbatim}
    

\subsection*{Input}

\begin{par}
\textbf{X}: Predictors, an n by p matrix, p is the number of predictors.  The predictors can be univariate or multivariate, discrete or continuous.
\end{par} \vspace{1em}
\begin{par}
\textbf{Y}: Multivariate responses, an n by r matrix, r is the number of responses and n is number of observations.  The responses must be continuous variables.
\end{par} \vspace{1em}
\begin{par}
\textbf{u}: Dimension of the envelope subspace.  A positive integer between 0 and r.
\end{par} \vspace{1em}
\begin{par}
\textbf{B}: Number of bootstrap samples.  A positive integer.
\end{par} \vspace{1em}
\begin{par}
\textbf{Opts}: A list containing the optional input parameter, to control the iterations in sg\_min. If one or several (even all) fields are not defined, the default settings are used.
\end{par} \vspace{1em}
\begin{itemize}
\setlength{\itemsep}{-1ex}
   \item Opts.maxIter: Maximum number of iterations.  Default value: 300.
   \item Opts.ftol: Tolerance parameter for F.  Default value: 1e-10.
   \item Opts.gradtol: Tolerance parameter for dF.  Default value: 1e-7.
   \item Opts.verbose: Flag for print out the number of bootstrap samples, logical 0 or 1. Default value: 0.
\end{itemize}


\subsection*{Output}

\begin{par}
\textbf{bootse}: The standard error for elements in $\beta$ computed by bootstrap.  An r by p matrix.
\end{par} \vspace{1em}


\subsection*{Description}

\begin{par}
This function computes the bootstrap standard errors for the regression coefficients in the scaled envelope model by bootstrapping the residuals.
\end{par} \vspace{1em}


\subsection*{Example}


\begin{verbatim}       load('sales.txt')
       Y = sales(:, 4 : 7);
       X = sales(:, 1 : 3);\end{verbatim}
    
\begin{verbatim}       u = bic_senv(X, Y)\end{verbatim}
        \color{lightgray}\ttfamily \begin{verbatim}
u =

     2
\end{verbatim} \rmfamily
\color{black}
\begin{verbatim}
       B = 20;
       bootse = bstrp_senv(X, Y, u, B)\end{verbatim}
    
        \color{lightgray}\ttfamily \begin{verbatim}

bootse =

    0.0539    0.0472    0.0554
    0.0675    0.0912    0.1178
    0.0567    0.0781    0.0791
    0.0986    0.0944    0.1283

\end{verbatim} \rmfamily
\color{black}
    

\newpage

\rmfamily
\color{black}\section{dF4senv}

\begin{par}
First derivative of the objective function for computing the envelope subspace in the scaled envelope model.
\end{par} \vspace{1em}

\subsection*{Contents}

\begin{itemize}
\setlength{\itemsep}{-1ex}
   \item Syntax
   \item Input
   \item Output
   \item Description
\end{itemize}


\subsection*{Syntax}


\begin{verbatim}       df = dF4senv(R, DataParameter)\end{verbatim}
    

\subsection*{Input}

\begin{par}
\textbf{R}: An r by u semi-orthogonal matrix, 0 \ensuremath{<} u \ensuremath{<}= r.
\end{par} \vspace{1em}
\begin{par}
\textbf{DataParameter}: A structure that contains the statistics calculated from the data.
\end{par} \vspace{1em}


\subsection*{Output}

\begin{par}
\textbf{df}: The first derivative of the objective function for computing the  envelope subspace.  An r by u matrix.
\end{par} \vspace{1em}


\subsection*{Description}

\begin{par}
This first derivative of F4senv obtained by matrix calculus calculations.
\end{par} \vspace{1em}

\newpage

\rmfamily
\color{black}\section{F4senv}

\begin{par}
Objective function for computing the envelope subspace in scaled envelope model.
\end{par} \vspace{1em}

\subsection*{Contents}

\begin{itemize}
\setlength{\itemsep}{-1ex}
   \item Syntax
   \item Input
   \item Output
   \item Description
\end{itemize}


\subsection*{Syntax}


\begin{verbatim}       f = F4senv(R, DataParameter)\end{verbatim}
    

\subsection*{Input}

\begin{par}
\textbf{R}: An r by u semi orthogonal matrix, 0 \ensuremath{<} u \ensuremath{<}= r.
\end{par} \vspace{1em}
\begin{par}
\textbf{DataParameter}: A structure that contains the statistics calculated from the data.
\end{par} \vspace{1em}


\subsection*{Output}

\begin{par}
\textbf{f}: A scalar containing the value of the objective function evaluated at R.
\end{par} \vspace{1em}


\subsection*{Description}

\begin{par}
The objective function is derived in Section 4.1 in Cook and Su (2012) using maximum likelihood estimation. The columns of the semi-orthogonal matrix that minimizes this function span the estimated envelope subspace.
\end{par} \vspace{1em}


\newpage

\rmfamily
\color{black}\section{objfun}

\begin{par}
Objective function for computing the scales in the scaled envelope model.
\end{par} \vspace{1em}

\subsection*{Contents}

\begin{itemize}
\setlength{\itemsep}{-1ex}
   \item Syntax
   \item Input
   \item Output
   \item Description
\end{itemize}


\subsection*{Syntax}


\begin{verbatim}       f = objfun(d, Gamma, DataParameter)\end{verbatim}
    

\subsection*{Input}

\begin{par}
\textbf{d}: An r - 1 dimensional column vector containing the scales for the 2nd to the rth responses.  All the entries in d are positive.
\end{par} \vspace{1em}
\begin{par}
\textbf{Gamma}: A r by u semi-orthogonal matrix that spans the envelope subspace or the estimated envelope subspace.
\end{par} \vspace{1em}
\begin{par}
\textbf{DataParameter}: A structure that contains the statistics calculated form the data.
\end{par} \vspace{1em}


\subsection*{Output}

\begin{par}
\textbf{f}: A scalar containing the value of the objective function evaluated at d.
\end{par} \vspace{1em}


\subsection*{Description}

\begin{par}
The objective function is derived in Section 4.1 of Su and Cook (2012)  using maximum likelihood estimation.
\end{par} \vspace{1em}


\newpage

    
    
\rmfamily
\color{black}\section{predict\_senv}

\begin{par}
Perform estimation or prediction under the scaled envelope model.
\end{par} \vspace{1em}

\subsection*{Contents}

\begin{itemize}
\setlength{\itemsep}{-1ex}
   \item Syntax
   \item Input
   \item Output
   \item Description
   \item Example
\end{itemize}


\subsection*{Syntax}


\begin{verbatim}       PredictOutput = predict_senv(ModelOutput, Xnew, infType)\end{verbatim}
    

\subsection*{Input}

\begin{par}
\textbf{ModelOutput}: A list containing the maximum likelihood estimators and other statistics inherited from senv.
\end{par} \vspace{1em}
\begin{par}
\textbf{Xnew}: The value of X with which to estimate or predict Y.  A p by 1 vector.
\end{par} \vspace{1em}
\begin{par}
\textbf{infType}: A string of characters indicting the inference type, the choices can be 'estimation' or 'prediction'.
\end{par} \vspace{1em}


\subsection*{Output}

\begin{par}
\textbf{PredictOutput}: A list containing the results of the inference.
\end{par} \vspace{1em}
\begin{itemize}
\setlength{\itemsep}{-1ex}
   \item PredictOutput.value: The fitted value or the prediction value evaluated at Xnew. An r by 1 vector.
   \item PredictOutput.covMatrix: The covariance matrix of PredictOutput.value. An r by r matrix.
   \item PredictOutput.SE: The standard error of elements in PredictOutput.value. An r by 1 vector.
\end{itemize}


\subsection*{Description}

\begin{par}
This function evaluates the scaled envelope model at new value Xnew.  It can perform estimation: find the fitted value when X = Xnew, or prediction: predict Y when X = Xnew.  The covariance matrix and the standard errors are also provided.
\end{par} \vspace{1em}


\subsection*{Example}


\begin{verbatim}       load('sales.txt')
       Y = sales(:, 4 : 7);
       X = sales(:, 1 : 3);
       u = bic_senv(X, Y);
       ModelOutput = senv(X, Y, u);
       Xnew = X(1, :)';
       PredictOutput = predict_senv(ModelOutput, Xnew, 'estimation')
       [PredictOutput.value, Y(1, :)']  % Compare the fitted value with the data
       PredictOutput.SE\end{verbatim}    
        \color{lightgray}\ttfamily \begin{verbatim}
PredictOutput = 

        value: [4x1 double]
    covMatrix: [4x4 double]
           SE: [4x1 double]


ans =

    8.9109    9.0000
   11.5096   12.0000
    9.6063    9.0000
   19.5119   20.0000


ans =

    7.7627
    5.8952
    4.2205
    8.0134
\end{verbatim} \rmfamily
\color{black}       
       \begin{verbatim}
       PredictOutput = predict_senv(ModelOutput, Xnew, 'prediction')
       PredictOutput.SE\end{verbatim}
        \color{lightgray}\ttfamily \begin{verbatim}

PredictOutput = 

        value: [4x1 double]
    covMatrix: [4x4 double]
           SE: [4x1 double]


ans =

    8.3024
    6.3052
    4.4254
    8.5956

\end{verbatim} \rmfamily
\color{black}
    


\newpage

\rmfamily
\color{black}\section{senv}

\begin{par}
Fit the scaled envelope model.
\end{par} \vspace{1em}

\subsection*{Contents}

\begin{itemize}
\setlength{\itemsep}{-1ex}
   \item Syntax
   \item Input
   \item Output
   \item Description
   \item References
   \item Example
\end{itemize}


\subsection*{Syntax}


\begin{verbatim}       ModelOutput = senv(X, Y, u)
       ModelOutput = senv(X, Y, u, Opts)\end{verbatim}
    

\subsection*{Input}

\begin{par}
\textbf{X}: Predictors. An n by p matrix, p is the number of predictors. The predictors can be univariate or multivariate, discrete or continuous.
\end{par} \vspace{1em}
\begin{par}
\textbf{Y}: Multivariate responses. An n by r matrix, r is the number of responses and n is number of observations. The responses must be continuous variables, and r should be strictly greater than p.
\end{par} \vspace{1em}
\begin{par}
\textbf{u}: Dimension of the envelope. An integer between 0 and r.
\end{par} \vspace{1em}
\begin{par}
\textbf{Opts}: A list containing the optional input parameter, to control the iterations in sg\_min. If one or several (even all) fields are not defined, the default settings are used.
\end{par} \vspace{1em}
\begin{itemize}
\setlength{\itemsep}{-1ex}
   \item Opts.maxIter: Maximum number of iterations.  Default value: 300.
   \item Opts.ftol: Tolerance parameter for F.  Default value: 1e-10.
   \item Opts.gradtol: Tolerance parameter for dF.  Default value: 1e-7.
   \item Opts.verbose: Flag for print out number of iterations, logical 0 or 1. Default value: 0.
      \item Opts.init: The initial value for the envelope subspace. An r by u matrix. Default value is the one generated by function get\_init.
\end{itemize}


\subsection*{Output}

\begin{par}
\textbf{ModelOutput}: A list that contains the maximum likelihood estimators and some statistics.
\end{par} \vspace{1em}
\begin{itemize}
\setlength{\itemsep}{-1ex}
   \item ModelOutput.beta: The scaled envelope estimator of the regression coefficients $\beta$. An r by p matrix.
   \item ModelOutput.Sigma: The scaled envelope estimator of the error covariance matrix.  An r by r matrix.
   \item ModelOutput.Lambda: The matrix of estimated scales. An r by r diagonal matrix with the first diagonal element equal to 1 and other diagonal elements being positive.
   \item ModelOutput.Gamma: The orthogonal basis of the envelope subspace. An r by u semi-orthogonal matrix.
   \item ModelOutput.Gamma0: The orthogonal basis of the complement of the envelope subspace.  An r by r - u semi-orthogonal matrix.
   \item ModelOutput.eta: The coordinates of $\beta$ with respect to Gamma. An u by p matrix.
   \item ModelOutput.Omega: The coordinates of Sigma with respect to Gamma. An u by u matrix.
   \item ModelOutput.Omega0: The coordinates of Sigma with respect to Gamma0. An r - u by r - u matrix.
   \item ModelOutput.alpha: The estimated intercept in the scaled envelope model.  An r by 1 vector.
   \item ModelOutput.l: The maximized log likelihood function.  A real number.
   \item ModelOutput.covMatrix: The asymptotic covariance of vec($\beta$).  An rp by rp matrix.  The covariance matrix returned are asymptotic.  For the actual standard errors, multiply by 1 / n.
   \item ModelOutput.asySenv: Asymptotic standard error for elements in $\beta$ under the scaled envelope model.  An r by p matrix.  The standard errors returned are asymptotic, for actual standard errors, multiply by 1 / sqrt(n).
   \item ModelOutput.ratio: The asymptotic standard error ratio of the standard multivariate linear regression estimator over the scaled envelope estimator, for each element in $\beta$.  An r by p matrix.
   \item ModelOutput.np: The number of parameters in the scaled envelope model.  A positive integer.
   \item ModelOutput.n: The number of observations in the data.  A positive integer.
\end{itemize}


\subsection*{Description}

\begin{par}
This function fits the scaled envelope model to the responses and predictors, using the maximum likelihood estimation.  When the dimension of the envelope is between 1 and r - 1, we implemented the algorithm in Cook and Su (2012).  When the dimension is r, then the scaled envelope model degenerates to the standard multivariate linear regression.  When the dimension is 0, it means that X and Y are uncorrelated, and the fitting is different.
\end{par} \vspace{1em}


\subsection*{References}

\begin{enumerate}
\setlength{\itemsep}{-1ex}
   \item The codes is implemented based on the algorithm in Section 4.1 of Cook and Su (2012).
   \item The Grassmann manifold optimization step calls the package sg\_min 2.4.1 by Ross Lippert (http://web.mit.edu/$\sim$ripper/www.sgmin.html).
\end{enumerate}


\subsection*{Example}

\begin{par}
The following codes produce the results of the test and performance example in Cook and Su (2012).
\end{par} \vspace{1em}

\begin{verbatim}       load('sales.txt')
       Y = sales(:, 4 : 7);
       X = sales(:, 1 : 3);
       u = bic_senv(X, Y)\end{verbatim}    
        \color{lightgray}\ttfamily \begin{verbatim}
u =

     2
\end{verbatim} \rmfamily
\color{black}       
       \begin{verbatim}
       ModelOutput = senv(X, Y, u)
       ModelOutput.Lambda
       1 - 1 ./ ModelOutput.ratio\end{verbatim}

        \color{lightgray}\ttfamily \begin{verbatim}
ModelOutput = 

         beta: [4x3 double]
        Sigma: [4x4 double]
       Lambda: [4x4 double]
        Gamma: [4x2 double]
       Gamma0: [4x2 double]
          eta: [2x3 double]
        Omega: [2x2 double]
       Omega0: [2x2 double]
        alpha: [4x1 double]
           np: 23
            l: -386.1900
    covMatrix: [12x12 double]
      asySenv: [4x3 double]
        ratio: [4x3 double]
            n: 50


ans =

    1.0000         0         0         0
         0    0.9729         0         0
         0         0    0.8067         0
         0         0         0    1.7016


ans =

    0.7020    0.5218    0.6517
    0.6137    0.3219    0.5614
    0.4439    0.1858    0.3924
    0.5375    0.4639    0.5228

\end{verbatim} \rmfamily
\color{black}
\newpage

\rmfamily
\color{black}\section{testcoefficient\_senv}


\begin{verbatim}This function tests the null hypothesis L * beta * R = A versus the
alternative hypothesis L * beta * R ~= A, where beta is estimated under
the scaled envelope model.\end{verbatim}
    
\subsection*{Contents}

\begin{itemize}
\setlength{\itemsep}{-1ex}
   \item Syntax
   \item Input
   \item Output
   \item Description
   \item Example
\end{itemize}


\subsection*{Syntax}


\begin{verbatim}       TestOutput = testcoefficient_senv(ModelOutput)
       TestOutput = testcoefficient_senv(ModelOutput, TestInput)\end{verbatim}
    

\subsection*{Input}

\begin{par}
\textbf{ModelOutput}: A list containing the maximum likelihood estimators and other statistics inherited from senv.
\end{par} \vspace{1em}
\begin{par}
\textbf{TestInput}: A list that specifies the null hypothesis, including L, R, and A.  If not provided by the user, default values will be used.
\end{par} \vspace{1em}
\begin{itemize}
\setlength{\itemsep}{-1ex}
   \item TestInput.L: The matrix multiplied to $\beta$ on the left.  It is a d1 by r matrix, while d1 is less than or equal to r.  Default value: identity matrix $I_r$.
   \item TestInput.R: The matrix multiplied to $\beta$ on the right.  It is a p by d2 matrix, while d2 is less than or equal to p.  Default value: identity matrix $I_p$.
   \item TestInput.A: The matrix on the right hand side of the equation.  It is a d1 by d2 matrix.  Default value: d1 by d2 zero matrix.
\end{itemize}


\subsection*{Output}

\begin{par}
\textbf{TestOutput}: A list containing test statistics, degrees of freedom for the reference chi-squared distribution, the p-value, and the covariance matrix of vec($L\beta R$).  At the same time, a table is printed out.
\end{par} \vspace{1em}
\begin{itemize}
\setlength{\itemsep}{-1ex}
   \item TestOutput.chisqStatistic: The test statistics. A real number.
   \item TestOutput.df: The degrees of freedom of the reference chi-squared distribution.  A positive integer.
   \item TestOutput.pValue: p-value of the test.  A real number in [0, 1].
   \item TestOutput.covMatrix: The covariance matrix of vec($L\beta R$). A d1 * d2 by d1 * d2 matrix.
\end{itemize}


\subsection*{Description}

\begin{par}
This function tests for hypothesis $H_0: L\beta R = A$, versus $H_\alpha: L\beta R\neq A$.  The $\beta$ is estimated by the scaled envelope model.  If the user does not specify the values for L, R and A, then the test is equivalent to the standard F test on if $\beta = 0$.  The test statistics used is vec $(L\beta R - A)$ $\hat{\Sigma}^{-1}$ vec $(L\beta R - A)^{T}$, and the reference distribution is chi-squared distribution with degrees of freedom d1 * d2.
\end{par} \vspace{1em}


\subsection*{Example}


\begin{verbatim}       load('sales.txt')
       Y = sales(:,4:7);
       X = sales(:,1:3);
       u = bic_senv(X,Y)
       ModelOutput = senv(X,Y,u);
       TestOutout = testcoefficient_senv(ModelOutput);\end{verbatim}
        \color{lightgray}\ttfamily \begin{verbatim}

 Test Hypothesis     Chisq Statistic    DF     P-value
------------------------------------------------------------------------
L * beta * R = A        4827.816        12      0.0000
------------------------------------------------------------------------
\end{verbatim} \rmfamily
\color{black}
\begin{verbatim}
       r = size(Y, 2);
       p = size(X, 2);
       TestInput.L = rand(2, r);
       TestInput.R = rand(p, 1);
       TestInput.A = zeros(2, 1);
       TestOutout = testcoefficient_senv(ModelOutput, TestInput);\end{verbatim}
         \color{lightgray}\ttfamily \begin{verbatim}
Test Hypothesis     Chisq Statistic    DF     P-value
------------------------------------------------------------------------
L * beta * R = A        1025.948        2       0.0000
------------------------------------------------------------------------
\end{verbatim} \rmfamily
\color{black}
    
\newpage

\chapter{xenv}

\rmfamily
\color{black}\section{aic\_xenv}

\begin{par}
Use Akaike information criterion to select the dimension of the envelope subspace for the reduction on X.
\end{par} \vspace{1em}

\subsection*{Contents}

\begin{itemize}
\setlength{\itemsep}{-1ex}
   \item Syntax
   \item Input
   \item Output
   \item Description
   \item Example
\end{itemize}


\subsection*{Syntax}


\begin{verbatim}       u = aic_xenv(X, Y)
       u = aic_xenv(X, Y, Opts)\end{verbatim}
    

\subsection*{Input}

\begin{par}
\textbf{X}: Predictors. An n by p matrix, p is the number of predictors. The predictors can be univariate or multivariate, discrete or continuous.
\end{par} \vspace{1em}
\begin{par}
\textbf{Y}: Multivariate responses. An n by r matrix, r is the number of responses and n is number of observations. The responses must be continuous variables.
\end{par} \vspace{1em}
\begin{par}
\textbf{Opts}: A list containing the optional input parameter, to control the iterations in sg\_min. If one or several (even all) fields are not defined, the default settings are used.
\end{par} \vspace{1em}
\begin{itemize}
\setlength{\itemsep}{-1ex}
   \item Opts.maxIter: Maximum number of iterations.  Default value: 300.
   \item Opts.ftol: Tolerance parameter for F.  Default value: 1e-10.
   \item Opts.gradtol: Tolerance parameter for dF.  Default value: 1e-7.
   \item Opts.verbose: Flag for print out dimension selection process, logical 0 or 1. Default value: 0.
\end{itemize}


\subsection*{Output}

\begin{par}
\textbf{u}: Dimension of the envelope. An integer between 0 and p.
\end{par} \vspace{1em}


\subsection*{Description}

\begin{par}
This function implements the Akaike information criteria (AIC) to select the dimension of the envelope subspace for the reduction on X.
\end{par} \vspace{1em}


\subsection*{Example}


\begin{verbatim}       load wheatprotein.txt
       X = wheatprotein(:, 1 : 6);
       Y = wheatprotein(:, 7);
       u = aic_xenv(X, Y)\end{verbatim}
    
        \color{lightgray}\ttfamily \begin{verbatim}
u =

     4

\end{verbatim} \rmfamily
\color{black}
    
\newpage

\rmfamily
\color{black}\section{bic\_xenv}

\begin{par}
Use Bayesian information criterion to select the dimension of the envelope subspace for the reduction on X.
\end{par} \vspace{1em}

\subsection*{Contents}

\begin{itemize}
\setlength{\itemsep}{-1ex}
   \item Syntax
   \item Input
   \item Output
   \item Description
   \item Example
\end{itemize}


\subsection*{Syntax}


\begin{verbatim}       u = bic_xenv(X, Y)
       u = bic_xenv(X, Y, Opts)\end{verbatim}
    

\subsection*{Input}

\begin{par}
\textbf{X}: Predictors. An n by p matrix, p is the number of predictors. The predictors can be univariate or multivariate, discrete or continuous.
\end{par} \vspace{1em}
\begin{par}
\textbf{Y}: Multivariate responses. An n by r matrix, r is the number of responses and n is number of observations. The responses must be continuous variables.
\end{par} \vspace{1em}
\begin{par}
\textbf{Opts}: A list containing the optional input parameter, to control the iterations in sg\_min. If one or several (even all) fields are not defined, the default settings are used.
\end{par} \vspace{1em}
\begin{itemize}
\setlength{\itemsep}{-1ex}
   \item Opts.maxIter: Maximum number of iterations.  Default value: 300.
   \item Opts.ftol: Tolerance parameter for F.  Default value: 1e-10.
   \item Opts.gradtol: Tolerance parameter for dF.  Default value: 1e-7.
   \item Opts.verbose: Flag for print out dimension selection process, logical 0 or 1. Default value: 0.
\end{itemize}


\subsection*{Output}

\begin{par}
\textbf{u}: Dimension of the envelope. An integer between 0 and p.
\end{par} \vspace{1em}


\subsection*{Description}

\begin{par}
This function implements the Bayesian information criteria (BIC) to select the dimension of the envelope subspace for the reduction on X.
\end{par} \vspace{1em}


\subsection*{Example}


\begin{verbatim}       load wheatprotein.txt
       X = wheatprotein(:, 1 : 6);
       Y = wheatprotein(:, 7);
       u = bic_xenv(X, Y)\end{verbatim}
    
        \color{lightgray}\ttfamily \begin{verbatim}
u =

     4

\end{verbatim} \rmfamily
\color{black}

\newpage

\rmfamily
\color{black}\section{bstrp\_xenv}

\begin{par}
Compute bootstrap standard error of the envelope model for the reduction on X.
\end{par} \vspace{1em}

\subsection*{Contents}

\begin{itemize}
\setlength{\itemsep}{-1ex}
   \item Syntax
   \item Input
   \item Output
   \item Description
   \item Example
\end{itemize}


\subsection*{Syntax}


\begin{verbatim}       bootse = bstrp_xenv(X, Y, u, B)
       bootse = bstrp_xenv(X, Y, u, B, Opts)\end{verbatim}
    

\subsection*{Input}

\begin{par}
\textbf{X}: Predictors, an n by p matrix, p is the number of predictors.  The predictors can be univariate or multivariate, discrete or continuous.
\end{par} \vspace{1em}
\begin{par}
\textbf{Y}: Multivariate responses, an n by r matrix, r is the number of responses and n is number of observations.  The responses must be continuous variables.
\end{par} \vspace{1em}
\begin{par}
\textbf{u}: Dimension of the envelope subspace.  A positive integer between 0 and p.
\end{par} \vspace{1em}
\begin{par}
\textbf{B}: Number of bootstrap samples.  A positive integer.
\end{par} \vspace{1em}
\begin{par}
\textbf{Opts}: A list containing the optional input parameter, to control the iterations in sg\_min. If one or several (even all) fields are not defined, the default settings are used.
\end{par} \vspace{1em}
\begin{itemize}
\setlength{\itemsep}{-1ex}
   \item Opts.maxIter: Maximum number of iterations.  Default value: 300.
   \item Opts.ftol: Tolerance parameter for F.  Default value: 1e-10.
   \item Opts.gradtol: Tolerance parameter for dF.  Default value: 1e-7.
   \item Opts.verbose: Flag for print out the number of bootstrap samples, logical 0 or 1. Default value: 0.
\end{itemize}


\subsection*{Output}

\begin{par}
\textbf{bootse}: The standard error for elements in $\beta$ computed by bootstrap.  An p by r matrix.
\end{par} \vspace{1em}


\subsection*{Description}

\begin{par}
This function computes the bootstrap standard errors for the regression coefficients in the envelope model by bootstrapping the residuals. The envelope model here is for the reduction on X.
\end{par} \vspace{1em}


\subsection*{Example}


\begin{verbatim}       load wheatprotein.txt
       X = wheatprotein(:, 1 : 6);
       Y = wheatprotein(:, 7);
       alpha = 0.01;
       u = lrt_xenv(X, Y, alpha)\end{verbatim}
        \color{lightgray}\ttfamily \begin{verbatim}
u =

     4
\end{verbatim} \rmfamily
\color{black}    
     \begin{verbatim}
       B = 100;
       bootse = bstrp_xenv(X, Y, u, B)\end{verbatim}
    
        \color{lightgray}\ttfamily \begin{verbatim}
bootse =

    0.0222
    0.0387
    0.0413
    0.0167
    0.0022
    0.0087

\end{verbatim} \rmfamily
\color{black}    
\newpage
    
\rmfamily
\color{black}\section{dF4xenv}

\begin{par}
The first derivative of the objective function for computing the envelope subspace for the reduction on X.
\end{par} \vspace{1em}

\subsection*{Contents}

\begin{itemize}
\setlength{\itemsep}{-1ex}
   \item Syntax
   \item Input
   \item Output
   \item Description
\end{itemize}


\subsection*{Syntax}


\begin{verbatim}       df = dF4xenv(R, DataParameter)\end{verbatim}
    

\subsection*{Input}

\begin{par}
\textbf{R}: A p by u semi orthogonal matrix, 0 \ensuremath{<} u \ensuremath{<}= p.
\end{par} \vspace{1em}
\begin{par}
\textbf{DataParameter}: A structure that contains the statistics calculated from the data.
\end{par} \vspace{1em}


\subsection*{Output}

\begin{par}
\textbf{df}: An p by u matrix containing the value of the derivative function evaluated at R.
\end{par} \vspace{1em}


\subsection*{Description}

\begin{par}
The objective function is derived in Section 4.5.1 of Cook et al. (2012) by  using maximum likelihood estimation. This function is the derivative of  the objective function.
\end{par} \vspace{1em}

\newpage

\rmfamily
\color{black}\section{F4xenv}

\begin{par}
Objective function for computing the envelope subspace for the reduction on X.
\end{par} \vspace{1em}

\subsection*{Contents}

\begin{itemize}
\setlength{\itemsep}{-1ex}
   \item Syntax
   \item Input
   \item Output
   \item Description
\end{itemize}


\subsection*{Syntax}


\begin{verbatim}       f = F4xenv(R, DataParameter)\end{verbatim}
    

\subsection*{Input}

\begin{par}
\textbf{R}: A p by u semi orthogonal matrix, 0 \ensuremath{<} u \ensuremath{<}= p.
\end{par} \vspace{1em}
\begin{par}
\textbf{DataParameter}: A structure that contains the statistics calculated from the data.
\end{par} \vspace{1em}


\subsection*{Output}

\begin{par}
\textbf{f}: A scalar containing the value of the objective function evaluated at R.
\end{par} \vspace{1em}


\subsection*{Description}

\begin{par}
The objective function is derived in Section 4.5.1 of Cook et al. (2012)  using maximum likelihood estimation. The columns of the semi-orthogonal matrix that minimizes this function span the estimated envelope subspace.
\end{par} \vspace{1em}

    
\newpage    

\rmfamily
\color{black}\section{lrt\_xenv}

\begin{par}
Use likelihood ratio testing to select the dimension of the envelope subspace for the reduction on X.
\end{par} \vspace{1em}

\subsection*{Contents}

\begin{itemize}
\setlength{\itemsep}{-1ex}
   \item Syntax
   \item Input
   \item Output
   \item Description
   \item Example
\end{itemize}


\subsection*{Syntax}


\begin{verbatim}       u = lrt_xenv(X, Y, alpha)
       u = lrt_xenv(X, Y, alpha, Opts)\end{verbatim}
    

\subsection*{Input}

\begin{par}
\textbf{X}: Predictors. An n by p matrix, p is the number of predictors. The predictors can be univariate or multivariate, discrete or continuous.
\end{par} \vspace{1em}
\begin{par}
\textbf{Y}: Multivariate responses. An n by r matrix, r is the number of responses and n is number of observations. The responses must be continuous variables.
\end{par} \vspace{1em}
\begin{par}
\textbf{alpha}: Significance level for testing.  A real number between 0 and 1, often taken at 0.05 or 0.01.
\end{par} \vspace{1em}
\begin{par}
\textbf{Opts}: A list containing the optional input parameter, to control the iterations in sg\_min. If one or several (even all) fields are not defined, the default settings are used.
\end{par} \vspace{1em}
\begin{itemize}
\setlength{\itemsep}{-1ex}
   \item Opts.maxIter: Maximum number of iterations.  Default value: 300.
   \item Opts.ftol: Tolerance parameter for F.  Default value: 1e-10.
   \item Opts.gradtol: Tolerance parameter for dF.  Default value: 1e-7.
   \item Opts.verbose: Flag for print out dimension selection process, logical 0 or 1. Default value: 0.
\end{itemize}


\subsection*{Output}

\begin{par}
\textbf{u}: Dimension of the envelope. An integer between 0 and p.
\end{par} \vspace{1em}


\subsection*{Description}

\begin{par}
This function implements the likelihood ratio testing procedure to select the dimension of the envelope subspace for the reduction on X, with pre-specified significance level $\alpha$.
\end{par} \vspace{1em}


\subsection*{Example}


\begin{verbatim}       load wheatprotein.txt
       X = wheatprotein(:, 1 : 6);
       Y = wheatprotein(:, 7);
       alpha = 0.01;
       u = lrt_xenv(X, Y, alpha)\end{verbatim}
    
        \color{lightgray}\ttfamily \begin{verbatim}
u =

     4

\end{verbatim} \rmfamily
\color{black}
    

\newpage

\rmfamily
\color{black}\section{predict\_xenv}

\begin{par}
Perform estimation or prediction under the envelope model for the reduction on X.
\end{par} \vspace{1em}

\subsection*{Contents}

\begin{itemize}
\setlength{\itemsep}{-1ex}
   \item Syntax
   \item Input
   \item Output
   \item Description
   \item Example
\end{itemize}


\subsection*{Syntax}


\begin{verbatim}       PredictOutput = predict_xenv(ModelOutput, Xnew, infType)\end{verbatim}
    

\subsection*{Input}

\begin{par}
\textbf{ModelOutput}: A list containing the maximum likelihood estimators and other statistics inherited from xenv.
\end{par} \vspace{1em}
\begin{par}
\textbf{Xnew}: The value of X with which to estimate or predict Y.  A p by 1 vector.
\end{par} \vspace{1em}
\begin{par}
\textbf{infType}: A string of characters indicting the inference type, the choices can be 'estimation' or 'prediction'.
\end{par} \vspace{1em}


\subsection*{Output}

\begin{par}
\textbf{PredictOutput}: A list containing the results of the inference.
\end{par} \vspace{1em}
\begin{itemize}
\setlength{\itemsep}{-1ex}
   \item PredictOutput.value: The fitted value or the prediction value evaluated at Xnew. An r by 1 vector.
   \item PredictOutput.covMatrix: The covariance matrix of PredictOutput.value. An r by r matrix.
   \item PredictOutput.SE: The standard error of elements in PredictOutput.value. An r by 1 vector.
\end{itemize}


\subsection*{Description}

\begin{par}
This function evaluates the envelope model for the reduction on X at new value Xnew.  It can perform estimation: find the fitted value when X = Xnew, or prediction: predict Y when X = Xnew.  The covariance matrix and the standard errors are also provided.
\end{par} \vspace{1em}


\subsection*{Example}


\begin{verbatim}       load wheatprotein.txt
       X = wheatprotein(:, 1 : 6);
       Y = wheatprotein(:, 7);
       u = bic_xenv(X, Y)
       ModelOutput = xenv(X, Y, u);
       Xnew = X(1, :)';
       PredictOutput = predict_xenv(ModelOutput, Xnew, 'estimation')
       [PredictOutput.value, Y(1, :)']  % Compare the fitted value with the data
       PredictOutput.SE\end{verbatim}
        \color{lightgray}\ttfamily \begin{verbatim}
PredictOutput = 

        value: 9.1751
    covMatrix: 16.8439
           SE: 4.1041


ans =

    9.1751    9.2300


ans =

    4.1041
\end{verbatim} \rmfamily
\color{black}
       
       \begin{verbatim}
       PredictOutput = predict_xenv(ModelOutput, Xnew, 'prediction')
       PredictOutput.SE\end{verbatim}

\color{lightgray}\ttfamily \begin{verbatim}
PredictOutput = 

        value: 9.1751
    covMatrix: 16.8760
           SE: 4.1080


ans =

    4.1080

\end{verbatim} \rmfamily
\color{black}

\newpage

\rmfamily
\color{black}\section{testcoefficient\_xenv}


\begin{verbatim}This function tests the null hypothesis L * beta * R = A versus the
alternative hypothesis L * beta * R ~= A, where beta is estimated under
the envelope model for the reduction on X.\end{verbatim}
    
\subsection*{Contents}

\begin{itemize}
\setlength{\itemsep}{-1ex}
   \item Syntax
   \item Input
   \item Output
   \item Description
   \item Example
\end{itemize}


\subsection*{Syntax}


\begin{verbatim}       TestOutput = testcoefficient_xenv(ModelOutput)
       TestOutput = testcoefficient_xenv(ModelOutput, TestInput)\end{verbatim}
    

\subsection*{Input}

\begin{par}
\textbf{ModelOutput}: A list containing the maximum likelihood estimators and other statistics inherited from xenv.
\end{par} \vspace{1em}
\begin{par}
\textbf{TestInput}: A list that specifies the null hypothesis, including L, R, and A.  If not provided by the user, default values will be used.
\end{par} \vspace{1em}
\begin{itemize}
\setlength{\itemsep}{-1ex}
   \item TestInput.L: The matrix multiplied to $\beta$ on the left.  It is a d1 by p matrix, while d1 is less than or equal to p.  Default value: identity matrix $I_p$.
   \item TestInput.R: The matrix multiplied to $\beta$ on the right.  It is a r by d2 matrix, while d2 is less than or equal to r.  Default value: identity matrix $I_r$.
   \item TestInput.A: The matrix on the right hand side of the equation.  It is a d1 by d2 matrix.  Default value: d1 by d2 zero matrix.
\end{itemize}


\subsection*{Output}

\begin{par}
\textbf{TestOutput}: A list containing test statistics, degrees of freedom for the reference chi-squared distribution, the p-value, and the covariance matrix of vec($L\beta R$).  At the same time, a table is printed out.
\end{par} \vspace{1em}
\begin{itemize}
\setlength{\itemsep}{-1ex}
   \item TestOutput.chisqStatistic: The test statistics. A real number.
   \item TestOutput.df: The degrees of freedom of the reference chi-squared distribution.  A positive integer.
   \item TestOutput.pValue: p-value of the test.  A real number in [0, 1].
   \item TestOutput.covMatrix: The covariance matrix of vec($L\beta R$). A d1 * d2 by d1 * d2 matrix.
\end{itemize}


\subsection*{Description}

\begin{par}
This function tests for hypothesis $H_0: L\beta R = A$, versus $H_\alpha: L\beta R\neq A$.  The $\beta$ is estimated by the envelope model for the reduction on X.  If the user does not specify the values for L, R and A, then the test is equivalent to the standard F test on if $\beta = 0$.  The test statistics used is vec $(L\beta R - A)$ $\hat{\Sigma}^{-1}$ vec $(L\beta R - A)^{T}$, and the reference distribution is chi-squared distribution with degrees of freedom d1 * d2.
\end{par} \vspace{1em}


\subsection*{Example}


\begin{verbatim}      load wheatprotein.txt
       X=wheatprotein(:, 1 : 6);
       Y=wheatprotein(:, 7);
       u = bic_xenv(X, Y);
       ModelOutput=xenv(X, Y, u);
       TestOutout = testcoefficient_xenv(ModelOutput);\end{verbatim}
    
        \color{lightgray}\ttfamily \begin{verbatim}
 Test Hypothesis     Chisq Statistic    DF     P-value
------------------------------------------------------------------------
L * beta * R = A        3233.053         6        0.0000
------------------------------------------------------------------------
\end{verbatim} \rmfamily
\color{black}
\begin{verbatim}
       r = size(Y, 2);
       p = size(X, 2);
       TestInput.L = rand(2, p);
       TestInput.R = rand(r, 1);
       TestInput.A = zeros(2, 1);
       TestOutout = testcoefficient_xenv(ModelOutput, TestInput);\end{verbatim}
        \color{lightgray}\ttfamily \begin{verbatim}
 Test Hypothesis     Chisq Statistic    DF     P-value
------------------------------------------------------------------------
L * beta * R = A          33.578         2        0.0000
------------------------------------------------------------------------
\end{verbatim} \rmfamily
\color{black}
    
    
\newpage
    
\rmfamily
\color{black}\section{xenv}

\begin{par}
Fit the envelope model for the reduction on X.
\end{par} \vspace{1em}

\subsection*{Contents}

\begin{itemize}
\setlength{\itemsep}{-1ex}
   \item Syntax
   \item Input
   \item Output
   \item Description
   \item References
   \item Example
\end{itemize}


\subsection*{Syntax}


\begin{verbatim}       ModelOutput = xenv(X, Y, u)
       ModelOutput = xenv(X, Y, u, Opts)\end{verbatim}
    

\subsection*{Input}

\begin{par}
\textbf{X}: Predictors. An n by p matrix, p is the number of predictors. The predictors can be univariate or multivariate, discrete or continuous.
\end{par} \vspace{1em}
\begin{par}
\textbf{Y}: Multivariate responses. An n by r matrix, r is the number of responses and n is number of observations. The responses must be continuous variables, and r should be strictly greater than p.
\end{par} \vspace{1em}
\begin{par}
\textbf{u}: Dimension of the envelope. An integer between 0 and p.
\end{par} \vspace{1em}
\begin{par}
\textbf{Opts}: A list containing the optional input parameter, to control the iterations in sg\_min. If one or several (even all) fields are not defined, the default settings are used.
\end{par} \vspace{1em}
\begin{itemize}
\setlength{\itemsep}{-1ex}
   \item Opts.maxIter: Maximum number of iterations.  Default value: 300.
   \item Opts.ftol: Tolerance parameter for F.  Default value: 1e-10.
   \item Opts.gradtol: Tolerance parameter for dF.  Default value: 1e-7.
   \item Opts.verbose: Flag for print out Grassmann manifold optimization process, logical 0 or 1. Default value: 0.
      \item Opts.init: The initial value for the envelope subspace. An p by u matrix. Default value is the one generated by function get\_init.
\end{itemize}


\subsection*{Output}

\begin{par}
\textbf{ModelOutput}: A list that contains the maximum likelihood estimators and some statistics.
\end{par} \vspace{1em}
\begin{itemize}
\setlength{\itemsep}{-1ex}
   \item ModelOutput.beta: The envelope estimator of the regression coefficients $\beta$. An p by r matrix.
   \item ModelOutput.SigX: The envelope estimator of the covariance matrix of X, $\Sigma_X$.  A p by p matrix.
   \item ModelOutput.Gamma: The orthogonal basis of the envelope subspace. An p by u semi-orthogonal matrix.
   \item ModelOutput.Gamma0: The orthogonal basis of the complement of the envelope subspace.  An p by p-u semi-orthogonal matrix.
   \item ModelOutput.eta: The coordinates of $\beta$ with respect to Gamma. An u by r matrix.
   \item ModelOutput.Omega: The coordinates of $\Sigma_X$ with respect to Gamma. An u by u matrix.
   \item ModelOutput.Omega0: The coordinates of $\Sigma_X$ with respect to Gamma0. An p - u by p - u matrix.
   \item ModelOutput.mu: The estimated intercept.  An r by 1 vector.
   \item ModelOutput.sigYcX: The estimated conditional covariance matrix of Y given X. An r by r matrix.
   \item ModelOutput.l: The maximized log likelihood function.  A real number.
   \item ModelOutput.covMatrix: The asymptotic covariance of vec($\beta$).  An pr by pr matrix.  The covariance matrix returned are asymptotic.  For the actual standard errors, multiply by 1 / n.
   \item ModelOutput.asyXenv: Asymptotic standard error for elements in $\beta$ under the envelope model.  An r by p matrix.  The standard errors returned are asymptotic, for actual standard errors, multiply by 1 / sqrt(n).
   \item ModelOutput.ratio: The asymptotic standard error ratio of the standard multivariate linear regression estimator over the envelope estimator, for each element in $\beta$.  An p by r matrix.
   \item ModelOutput.np: The number of parameters in the envelope model.  A positive integer.
   \item ModelOutput.n: The number of observations in the data.  A positive integer.
\end{itemize}


\subsection*{Description}

\begin{par}
This function fits the envelope model to the responses and predictors, using the maximum likelihood estimation.  When the dimension of the envelope is between 1 and r - 1, we implemented the algorithm in Cook et al. (2012).  When the dimension is r, then the envelope model degenerates to the standard multivariate linear regression.  When the dimension is 0, it means that X and Y are uncorrelated, and the fitting is different.
\end{par} \vspace{1em}


\subsection*{References}

\begin{enumerate}
\setlength{\itemsep}{-1ex}
   \item The codes is implemented based on the algorithm in Section 4.5.1 of Cook et al (2012).
   \item The Grassmann manifold optimization step calls the package sg\_min 2.4.1 by Ross Lippert (http://web.mit.edu/$\sim$ripper/www.sgmin.html).
\end{enumerate}


\subsection*{Example}


\begin{verbatim}       load wheatprotein.txt
       X = wheatprotein(:, 1 : 6);
       Y = wheatprotein(:, 7);\end{verbatim}
    
\begin{verbatim}       p = size(X, 2);
       ModelOutput = xenv(X, Y, p);\end{verbatim}
    
\begin{verbatim}       % When u = p, the envelope model reduces to the ordinary least squares
       % regression \end{verbatim}
    
\begin{verbatim}       temp = fit_OLS(X, Y);
       temp.SigmaOLS
       ModelOutput.sigYcX
       temp.betaOLS'
       ModelOutput.beta\end{verbatim}
        \color{lightgray}\ttfamily \begin{verbatim}
ans =

    0.0321


ans =

    0.0321


ans =

   -0.0416
   -0.0490
    0.3368
   -0.1981
    0.0020
   -0.0480


ans =

   -0.0416
   -0.0490
    0.3368
   -0.1981
    0.0020
   -0.0480
\end{verbatim} \rmfamily
\color{black}
    
\begin{verbatim}       u = bic_xenv(X, Y);
       ModelOutput = xenv(X, Y, u)\end{verbatim}

\color{lightgray}\ttfamily \begin{verbatim}

ModelOutput = 

         beta: [6x1 double]
         SigX: [6x6 double]
        Gamma: [6x4 double]
       Gamma0: [6x2 double]
          eta: [4x1 double]
        Omega: [4x4 double]
       Omega0: [2x2 double]
           mu: 24.8863
       sigYcX: 0.0321
            l: -6.7737e+05
    covMatrix: [6x6 double]
      asyXenv: [6x1 double]
        ratio: [6x1 double]
           np: 27
            n: 50
\end{verbatim} \rmfamily
\color{black}
    
\begin{verbatim}       %  To compare with the results obtained by Partial Least Squares, use the 
       %  plsregress command
       [XL, YL, XS, YS, BETA, PCTVAR, MSE, stats] = plsregress(X, Y, u);
       ModelOutput.beta
       BETA(2 : end, :)\end{verbatim}
    

        \color{lightgray}\ttfamily \begin{verbatim}

ans =

   -0.0443
   -0.0481
    0.3377
   -0.1963
    0.0019
   -0.0487



ans =

   -0.0199
    0.1373
    0.1309
   -0.1827
    0.0056
   -0.0708

\end{verbatim} \rmfamily
\color{black}

\newpage

\chapter{auxiliary}

    
\rmfamily
\color{black}\section{center}

\begin{par}
Subtract the mean of each column.
\end{par} \vspace{1em}

\subsection*{Contents}

\begin{itemize}
\setlength{\itemsep}{-1ex}
   \item Syntax
   \item Input
   \item Output
   \item Description
\end{itemize}


\subsection*{Syntax}


\begin{verbatim}       XC = center(X)\end{verbatim}
    

\subsection*{Input}

\begin{par}
\textbf{X}: A matrix or a column vector.
\end{par} \vspace{1em}


\subsection*{Output}

\begin{par}
\textbf{XC}: A matrix or a column vector with the mean for each column equal to 0.
\end{par} \vspace{1em}


\subsection*{Description}

\begin{par}
This function centerizes a matrix or a vector, by subtracting each column by its column mean.
\end{par} \vspace{1em}

\newpage

\rmfamily
\color{black}\section{Contr}

\begin{par}
Compute the contraction matrix of dimension r.
\end{par} \vspace{1em}

\subsection*{Contents}

\begin{itemize}
\setlength{\itemsep}{-1ex}
   \item Syntax
   \item Input
   \item Output
   \item Description
\end{itemize}


\subsection*{Syntax}


\begin{verbatim}       C = Contr(r)\end{verbatim}
    

\subsection*{Input}

\begin{par}
\textbf{r}: Dimension of the contraction matrix.  A positive integer.
\end{par} \vspace{1em}


\subsection*{Output}

\begin{par}
\textbf{C}: Contraction matrix of dimension r.  C is an r(r + 1) / 2 by r \^{} 2 matrix.
\end{par} \vspace{1em}


\subsection*{Description}

\begin{par}
The contraction and expansion matrices are links between the "vec" operator and "vech"operator: for an r by r symmetric matrix A, vech(A) = Contr(r) * vec(A), and vec(A) = Expan(r) * vech(A). The "vec" operator stacks the matrix A into an r \^{} 2 by 1 vector columnwise.  The "vech" operator stacks the lower triangle or the upper triangle of a symmetric matrix into an r(r+1)/2 vector. For more details of "vec", "vech", contraction and expansion matrix, refer to Henderson and Searle (1979).
\end{par} \vspace{1em}

\newpage

\rmfamily
\color{black}\section{Expan}

\begin{par}
Compute the expansion matrix of dimension r.
\end{par} \vspace{1em}

\subsection*{Contents}

\begin{itemize}
\setlength{\itemsep}{-1ex}
   \item Syntax
   \item Input
   \item Output
   \item Description
\end{itemize}


\subsection*{Syntax}


\begin{verbatim}       E = Expan(r)\end{verbatim}
    

\subsection*{Input}

\begin{par}
\textbf{r}: Dimension of the expansion matrix.  A positive integer.
\end{par} \vspace{1em}


\subsection*{Output}

\begin{par}
\textbf{E}: Expansion matrix of dimension r.  E is an r \^{} 2 by r(r + 1) / 2 matrix.
\end{par} \vspace{1em}


\subsection*{Description}

\begin{par}
The contraction and expansion matrices are links between the "vec" operator and "vech" operator: for an r by r symmetric matrix A, vech(A) = Contr(r) * vec(A), and vec(A) = Expan(r) * vech(A). The "vec" operator stacks the matrix A into an r \^{} 2 by 1 vector columnwise.  The "vech" operator stacks the lower triangle or the upper triangle of a symmetric matrix into an r(r + 1) / 2 vector. For more details of "vec", "vech", contraction and expansion matrix, refer to Henderson and Searle (1979).
\end{par} \vspace{1em}

\newpage

\rmfamily
\color{black}\section{fit\_OLS}

\begin{par}
Multivariate linear regression.
\end{par} \vspace{1em}

\subsection*{Contents}

\begin{itemize}
\setlength{\itemsep}{-1ex}
   \item Syntax
   \item Input
   \item Output
   \item Description
   \item Example
\end{itemize}


\subsection*{Syntax}


\begin{verbatim}       ModelOutput = fit_OLS(X, Y)\end{verbatim}
    

\subsection*{Input}

\begin{par}
\textbf{X}: Predictors, an n by p matrix, p is the number of predictors.  The predictors can be univariate or multivariate, discrete or continuous.
\end{par} \vspace{1em}
\begin{par}
\textbf{Y}: Multivariate responses, an n by r matrix, r is the number of responses and n is number of observations.  The responses must be continuous variables.
\end{par} \vspace{1em}


\subsection*{Output}

\begin{par}
\textbf{ModelOutput}: A list that contains the maximum likelihood estimators of regression coefficients and error covariance matrix.
\end{par} \vspace{1em}
\begin{itemize}
\setlength{\itemsep}{-1ex}
   \item ModelOutput.betaOLS: An r by p matrix containing estimate of the regression coefficients $\beta$.
   \item ModelOutput.SigmaOLS: An r by r matrix containing estimate of the error covariance matrix.
   \item ModelOutput.alpha: An r by 1 vector containing estimate of the intercept.
   \item ModelOutput.n: The number of observations in the data.  A positive integer.
\end{itemize}


\subsection*{Description}

\begin{par}
In a multivariate linear model, Y and X follows the following relationship: $Y=\alpha+\beta X+\varepsilon$, where $\varepsilon$ contains the errors.  This function performs the ordinary least squares fit to the inputs, and returns the estimates of $\beta$ and the covariance matrix of $\varepsilon$.
\end{par} \vspace{1em}


\subsection*{Example}


\begin{verbatim}       load wheatprotein.txt
       X = wheatprotein(:, 8);
       Y = wheatprotein(:, 1 : 6);
       ModelOutput = fit_OLS(X, Y)
       ModelOutput.betaOLS
       ModelOutput.SigmaOLS\end{verbatim}
    
        \color{lightgray}\ttfamily \begin{verbatim}
ModelOutput = 

     betaOLS: [6x1 double]
    SigmaOLS: [6x6 double]
       alpha: [6x1 double]
           n: 50


ans =

    3.2724
    8.0288
    7.5224
   -2.0609
    3.2244
    0.6538


ans =

   1.0e+03 *

    1.1905    0.9759    1.0506    1.1524    1.5384    0.6335
    0.9759    0.8061    0.8657    0.9432    1.2636    0.5266
    1.0506    0.8657    0.9310    1.0164    1.3664    0.5640
    1.1524    0.9432    1.0164    1.1228    1.5234    0.6183
    1.5384    1.2636    1.3664    1.5234    2.3229    0.8360
    0.6335    0.5266    0.5640    0.6183    0.8360    0.3618

\end{verbatim} \rmfamily
\color{black}
    
\newpage

\rmfamily
\color{black}\section{get\_Init}

\begin{par}
Starting value for the envelope subspace.
\end{par} \vspace{1em}

\subsection*{Contents}

\begin{itemize}
\setlength{\itemsep}{-1ex}
   \item Syntax
   \item Input
   \item Output
   \item Description
   \item Reference
\end{itemize}


\subsection*{Syntax}


\begin{verbatim}       WInit = get_Init(X, Y, u, DataParameter)\end{verbatim}
    

\subsection*{Input}

\begin{par}
\textbf{X}: Predictors. An n by p matrix, p is the number of predictors.
\end{par} \vspace{1em}
\begin{par}
\textbf{Y}: Multivariate responses. An n by r matrix, r is the number of responses and n is number of observations.
\end{par} \vspace{1em}
\begin{par}
\textbf{u}: Dimension of the envelope. An integer between 1 and r - 1.
\end{par} \vspace{1em}
\begin{par}
\textbf{DataParameter}: A list containing commonly used statistics computed from the data.
\end{par} \vspace{1em}


\subsection*{Output}

\begin{par}
\textbf{WInit}: The initial estimate of the orthogonal basis of the envelope subspace. An r by u orthogonal matrix.
\end{par} \vspace{1em}


\subsection*{Description}

\begin{par}
We compute the eigenvectors for the covariance matrices of Y and the estimated errors, and get 2r vectors.  Then we get all the combinations of u vectors out of the 2r vectors. If the number of 2r choose u is small(\ensuremath{<}= 50), we search over all the combinations and find out the one that minimizes the objective function F. If that number is large, then we do it iteratively: we pick up any u eigenvectors, fix all of them except the first one. Then we search over all the vectors orthogonal to the fixed ones, and record the one that minimizes F. Next, we fix the first u eigenvectors again but this time search for second one, then we record the vector. This goes on and on until the last one. We do it for 5 rounds and use the final set as our starting value.
\end{par} \vspace{1em}


\subsection*{Reference}

\begin{par}
The codes is implemented based on the algorithm in Section 3.5 of Su and Cook (2011).
\end{par} \vspace{1em}

\newpage

\rmfamily
\color{black}\section{get\_Init4henv}

\begin{par}
Starting value for the heteroscedastic envelope subspace.
\end{par} \vspace{1em}

\subsection*{Contents}

\begin{itemize}
\setlength{\itemsep}{-1ex}
   \item Syntax
   \item Input
   \item Output
   \item Description
   \item Reference
\end{itemize}


\subsection*{Syntax}


\begin{verbatim}       WInit = get_Init4henv(F, X, Y, u, DataParameter)\end{verbatim}
    

\subsection*{Input}

\begin{par}
\textbf{F}: Objective function to get the heteroscedastic envelope subspace.
\end{par} \vspace{1em}
\begin{par}
\textbf{X}: Group indicators. A matrix with n rows.  X can only have p unique  rows, where p is the number of groups. For example, if there are two groups, X can only have 2 different kinds of rows, such as (0, 1) and (1, 0), or (1, 0, 10) and (0, 5, 6).  The number of columns is not restricted, as long as X only has p unique rows.
\end{par} \vspace{1em}
\begin{par}
\textbf{Y}: Multivariate responses. An n by r matrix, r is the number of responses and n is number of observations.
\end{par} \vspace{1em}
\begin{par}
\textbf{u}: Dimension of the envelope. An integer between 1 and r - 1.
\end{par} \vspace{1em}
\begin{par}
\textbf{DataParameter}: A list containing commonly used statistics computed from the data.
\end{par} \vspace{1em}


\subsection*{Output}

\begin{par}
\textbf{WInit}: The initial estimate of the orthogonal basis of the heteroscedastic envelope subspace. An r by u orthogonal matrix.
\end{par} \vspace{1em}


\subsection*{Description}

\begin{par}
We compute the eigenvectors for the estimated errors, and get r vectors.  Then we get all the combinations of u vectors out of the r vectors. If the number of r choose u is small(\ensuremath{<}=50), we search over all the combinations and find out the one that minimizes the objective function F. If that number is large, then we do it iteratively: we pick up any u eigenvectors, fix all of them except the first one. Then we search over all the vectors orthogonal to the fixed ones, and record the one that minimizes F. Next, we fix the first u eigenvectors again but this time search for second one, then we record the vector. This goes on and on until the last one. We do it for 3 rounds and use the final set as our starting value.
\end{par} \vspace{1em}


\subsection*{Reference}

\begin{par}
The codes is implemented based on the algorithm in Section 3.5 of Su and Cook (2011).
\end{par} \vspace{1em}


\newpage

\rmfamily
\color{black}\section{Kpd}

\begin{par}
Compute the communication matrix Kpd.
\end{par} \vspace{1em}

\subsection*{Contents}

\begin{itemize}
\setlength{\itemsep}{-1ex}
   \item Syntax
   \item Input
   \item Output
   \item Description
   \item Reference
\end{itemize}


\subsection*{Syntax}


\begin{verbatim}       k = Kpd(p, d)\end{verbatim}
    

\subsection*{Input}

\begin{par}
\textbf{p}, \textbf{d}: two positive integers represent the dimension parameters for the communication matrix.
\end{par} \vspace{1em}


\subsection*{Output}

\begin{par}
\textbf{k}: The communication matrix Kpd. An p * d by p * d matrix.
\end{par} \vspace{1em}


\subsection*{Description}

\begin{par}
For a p by d matrix A, vec(A') = Kpd * vec(A), and Kpd is called a communication matrix.
\end{par} \vspace{1em}


\subsection*{Reference}

\begin{par}
The codes is implemented based on Definition 3.1 in Magnus and Neudecker (1979).
\end{par} \vspace{1em}

\newpage

\rmfamily
\color{black}\section{Lmatrix}

\begin{par}
Extract the 2nd to the last diagonal element of a matrix into a vector.
\end{par} \vspace{1em}

\subsection*{Contents}

\begin{itemize}
\setlength{\itemsep}{-1ex}
   \item Syntax
   \item Input
   \item Output
   \item Description
\end{itemize}


\subsection*{Syntax}


\begin{verbatim}       L = Lmatrix(r)\end{verbatim}
    

\subsection*{Input}

\begin{par}
\textbf{r}: The dimension of the matrix being extracted.  The matrix should be an r by r matrix.
\end{par} \vspace{1em}


\subsection*{Output}

\begin{par}
\textbf{L}: An r - 1 dimensional vector that contains all the diagonal elements but the first one of the matrix.
\end{par} \vspace{1em}


\subsection*{Description}

\begin{par}
Let A be an r by r matrix, and vec be the vector operator, then Lmatrix(r) * vec(A) will give the 2nd to the rth diagonal elements of A, arranged in a column vector.
\end{par} \vspace{1em}

\newpage

\rmfamily
\color{black}\section{make\_dF}

\begin{par}
Generic function to generate the derivative function of the objective function F.
\end{par} \vspace{1em}

\subsection*{Contents}

\begin{itemize}
\setlength{\itemsep}{-1ex}
   \item Syntax
   \item Input
   \item Output
   \item Description
\end{itemize}


\subsection*{Syntax}

\begin{par}
dF = make\_dF(dfun\_method\_handle, FParameters)
\end{par} \vspace{1em}


\subsection*{Input}

\begin{itemize}
\setlength{\itemsep}{-1ex}
   \item dfun\_method\_handle: A specific model derivative function of the objective function.
   \item FParameters: A structure that contains data parameters as input for the function dfun\_method\_handle.
\end{itemize}


\subsection*{Output}

\begin{itemize}
\setlength{\itemsep}{-1ex}
   \item dF: The generic derivative function of the objective function for computing the envelope subspace.
\end{itemize}


\subsection*{Description}

\begin{par}
Generic function to generate the derivative function of the objective function F. The function first sets a handle to the specific model function and fixes the data parameters from the sample needed for its computation. The handle fixed with those parameters is then evaluated at a given value for argument W. A generic derivative function dF is returned.
\end{par} \vspace{1em}

\newpage

\rmfamily
\color{black}\section{make\_F}

\begin{par}
Generic function to generate the objective function F.
\end{par} \vspace{1em}

\subsection*{Contents}

\begin{itemize}
\setlength{\itemsep}{-1ex}
   \item Syntax
   \item Input
   \item Output
   \item Description
\end{itemize}


\subsection*{Syntax}

\begin{par}
F = make\_F(fun\_method\_handle, FParameters)
\end{par} \vspace{1em}


\subsection*{Input}

\begin{itemize}
\setlength{\itemsep}{-1ex}
   \item fun\_method\_handle: A specific model objective function.
   \item FParameters: A structure that contains data parameters as input for the function fun\_method\_handle.
\end{itemize}


\subsection*{Output}

\begin{itemize}
\setlength{\itemsep}{-1ex}
   \item F: The generic objective function for computing the envelope subspace.
\end{itemize}


\subsection*{Description}

\begin{par}
Generic function to generate the objective function F. The function first sets a handle to the specific model function and fixes the data parameters from the sample needed for its computation. The handle fixed with those parameters is then evaluated at a given value for argument W. A generic objective function F is returned.
\end{par} \vspace{1em}

\newpage 

\rmfamily
\color{black}\section{make\_opts}

\begin{par}
Make optional input parameters for running the sg\_min package.
\end{par} \vspace{1em}

\subsection*{Contents}

\begin{itemize}
\setlength{\itemsep}{-1ex}
   \item Syntax
   \item Input
   \item Output:
   \item Description
\end{itemize}


\subsection*{Syntax}


\begin{verbatim}       Opts = make_opts(Opts)\end{verbatim}
    

\subsection*{Input}

\begin{par}
\textbf{Opts}: A list containing optional input parameter for sg\_min.m specified by users.  One or several (even all) fields could be empty.
\end{par} \vspace{1em}
\begin{itemize}
\setlength{\itemsep}{-1ex}
   \item Opts.maxIter: Maximum number of iterations.
   \item Opts.ftol: Tolerance parameter for F.
   \item Opts.gradtol: Tolerance parameter for dF.
   \item Opts.verbose: Flag for print out output, logical 0 or 1.
\end{itemize}


\subsection*{Output:}

\begin{par}
\textbf{Opts}: A list containing optional input parameter for sg\_min.m, specified by users or the default values are used.
\end{par} \vspace{1em}
\begin{itemize}
\setlength{\itemsep}{-1ex}
   \item Opts.maxIter: Maximum number of iterations.  Default value: 300.
   \item Opts.ftol: Tolerance parameter for F.  Default value: 1e-10.
   \item Opts.gradtol: Tolerance parameter for dF.  Default value: 1e-7.
   \item Opts.verbose: Flag for print out output, logical 0 or 1. Default value: 0.
\end{itemize}


\subsection*{Description}

\begin{par}
The sg\_min function has some optional input parameters that control the iteration process.  These parameters include maximum number of iteration, tolerance parameters for convergence of the objective function F and the derivative of the objective function dF, and the print out of the iteration process.  The user can set one or all of parameters, if not, default values will be used.
\end{par} \vspace{1em}

\newpage

\rmfamily
\color{black}\section{make\_parameter}

\begin{par}
Compute summary statistics from the data.
\end{par} \vspace{1em}

\subsection*{Contents}

\begin{itemize}
\setlength{\itemsep}{-1ex}
   \item Syntax
   \item Input
   \item Output
   \item Description
\end{itemize}


\subsection*{Syntax}


\begin{verbatim}       DataParameter = make_parameter(X, Y, method)\end{verbatim}
    

\subsection*{Input}

\begin{par}
\textbf{X}: Predictors. An n by p matrix, p is the number of predictors. The predictors can be univariate or multivariate, discrete or continuous.
\end{par} \vspace{1em}
\begin{par}
\textbf{Y}: Multivariate responses. An n by r matrix, r is the number of responses and n is number of observations. The responses must be continuous variables, and r should be strictly greater than p.
\end{par} \vspace{1em}
\begin{par}
\textbf{method}: A string of characters indicating which member of the envelope family to be used, the choices can be 'env', 'ienv', 'henv', 'senv' or 'xenv'.
\end{par} \vspace{1em}


\subsection*{Output}

\begin{par}
\textbf{DataParameter}: A list that contains summary statistics computed from the data.  The output list can vary from method to method.
\end{par} \vspace{1em}
\begin{itemize}
\setlength{\itemsep}{-1ex}
   \item DataParameter.n: The number of observations in the data.  A positive integer.
   \item DataParameter.ng: A p by 1 vector containing the number of observations in each group.  p is the number of groups.  Only for 'henv'.
   \item DataParameter.ncum: A p by 1 vector containing the total number of observations till this group.  Only for 'henv'.
   \item DataParameter.ind: An n by 1 vector indicating the sequence of the observations after sorted by groups.
   \item DataParameter.p: The number of predictors or number of groups for 'henv'.  A positive integer.
   \item DataParameter.r: The number of responses.  A positive integer.
   \item DataParameter.XC: Centered predictors.  An n by p matrix with the ith row being the ith observation of X subtracted by the mean of X.  Only for 'env' and 'ienv'.
   \item DataParameter.YC: Centered responses.  An n by r matrix with the ith row being the ith observation of Y subtracted by the mean of Y.  Only for 'env' and 'ienv'.
   \item DataParameter.mX: The mean of predictors.  A p by 1 vector.  For all method except 'henv'.
   \item DataParameter.mY: The mean of responses.  An r by 1 vector.
   \item DataParameter.mYg: An r by p matrix with the ith column being the sample mean of the ith group.
   \item DataParameter.sigX: The sample covariance matrix of X.  A p by p matrix.
   \item DataParameter.sigY: The sample covariance matrix of Y.  An r by r matrix.
   \item DataParameter.sigRes: For 'env', 'senv', 'ienv': The sample covariance matrix of the residuals from the ordinary least squares regression of Y on X.  An r by r matrix. For 'henv', an r by r by p three dimensional matrix with the ith depth is the ith sample covariance matrix for the ith group.
   \item DataParameter.sigFit: The sample covariance matrix of the fitted value from the ordinary least squares regression of Y on X.  An r by r matrix. Only for method 'ienv'.
   \item DataParameter.betaOLS: The regression coefficients from the ordinary least squares regression of Y on X.  An r by p matrix.  For all method except 'henv'.
\end{itemize}


\subsection*{Description}

\begin{par}
This function computes statistics that will be used frequently in the estimation for each method.
\end{par} \vspace{1em}

\newpage

\section{mtest}

\begin{par}
Perform Box's M test to check the homogeneity of the covariance matrices.
\end{par} \vspace{1em}

\subsection*{Contents}

\begin{itemize}
\setlength{\itemsep}{-1ex}
   \item Syntax
   \item Input
   \item Output
   \item Description
   \item References
   \item Example
\end{itemize}


\subsection*{Syntax}


\begin{verbatim}       TestOutput = mtest(X, Y, alpha)\end{verbatim}
    

\subsection*{Input}

\begin{par}
\textbf{X}: Group indicators. A matrix with n rows.  X can only have p unique  rows, where p is the number of groups. For example, if there are two groups, X can only have 2 different kinds of rows, such as (0, 1) and (1, 0), or (1, 0, 10) and (0, 5, 6).  The number of columns is not restricted, as long as X only has p unique rows.
\end{par} \vspace{1em}
\begin{par}
\textbf{Y}: Multivariate responses. An n by r matrix, r is the number of responses and n is number of observations. The responses must be continuous variables, and r should be greater than p.
\end{par} \vspace{1em}
\begin{par}
\textbf{alpha}: Significance level for testing.  A real number between 0 and 1, often taken at 0.05 or 0.01.
\end{par} \vspace{1em}


\subsection*{Output}

\begin{par}
\textbf{TestOutput}: A list containing the Box's M statistic, the approximation test statistic, degrees of freedom for the approximation statistic test, and the p-value.  At the same time, a table is printed out.
\end{par} \vspace{1em}
\begin{itemize}
\setlength{\itemsep}{-1ex}
   \item TestOutput.mStatistic: The Box's M statistic. A real number.
   \item TestOutput.approxStatistic: The approximation test statistic.
   \item TestOutput.df: The degrees of freedom of the approximation statistic test.  A positive integer.
   \item TestOutput.pValue: p-value of the test.  A real number in [0, 1].
\end{itemize}


\subsection*{Description}

\begin{par}
This function performs the Box's M test for homegeneity of the covariance matrices for different groups, indicated by X. If the groups sample-size is at least 20 (sufficiently large), Box's M test takes a Chi-square approximation; otherwise it takes an F approximation.
\end{par} \vspace{1em}


\subsection*{References}

\begin{par}
The codes are implemented based on\\ 
Trujillo-Ortiz, A., R. Hernandez-Walls, K. Castro-Morales, A. Espinoza-Tenorio, A. Guia-Ramirez and R. Carmona-Pina. (2002). MBoxtest: Multivariate Statistical Testing for the Homogeneity of Covariance Matrices by the Box's M. A MATLAB file. [WWW document]. URL: \\
http://www.mathworks.com/matlabcentral/fileexchange\\
/loadFile.do
?objectId=2733\&objectType=FILE
\end{par} \vspace{1em}


\subsection*{Example}


\begin{verbatim}       load waterstrider.mat
       alpha = 0.01;
       TestOutput = mtest(X, Y, alpha);\end{verbatim}
    
        \color{lightgray} \begin{verbatim} 
----------------------------------------------------------
     MBox     Chi-sqr.         df          P
----------------------------------------------------------
  157.5977   137.3361          72       0.0000
----------------------------------------------------------
Covariance matrices are significantly different.
\end{verbatim} \color{black}
    

\end{document}
    
