
% This LaTeX was auto-generated from an M-file by MATLAB.
% To make changes, update the M-file and republish this document.

\documentclass{article}
\usepackage{graphicx}
\usepackage{color}

\sloppy
\definecolor{lightgray}{gray}{0.5}
\setlength{\parindent}{0pt}

\begin{document}





\title{Matlab Package: Envelope}



\maketitle



\setcounter{tocdepth}{2}
\tableofcontents


\newpage


\part{Tools}
    
\section{bstrp\_OLS}

\begin{par}
Compute bootstrap standard error for ordinary least squares.
\end{par} \vspace{1em}

\subsection*{Contents}

\begin{itemize}
\setlength{\itemsep}{-1ex}
   \item Usage
   \item Description
\end{itemize}


\subsection*{Usage}

\begin{par}
bootse=bstrp\_OLS(X,Y,B)
\end{par} \vspace{1em}
\begin{par}
Input
\end{par} \vspace{1em}
\begin{itemize}
\setlength{\itemsep}{-1ex}
   \item X: Predictors, an n by p matrix, p is the number of predictors.  The predictors can be univariate or multivariate, discrete or continuous.
   \item Y: Multivariate responses, an n by r matrix, r is the number of responses and n is number of observations.  The responses must be continuous variables.
   \item B: Number of boostrap samples.  A positive integer.
\end{itemize}
\begin{par}
Output
\end{par} \vspace{1em}
\begin{itemize}
\setlength{\itemsep}{-1ex}
   \item bootse: The standard error for elements in $\beta$ computed by bootstrap.  An r by p matrix.
\end{itemize}


\subsection*{Description}

\begin{par}
This function computes the bootstrap standard errors for the regression coefficients in ordinary least squares by bootstrapping the residuals.
\end{par} \vspace{1em}

\newpage


\section{center}

\begin{par}
Substract the mean of each column
\end{par} \vspace{1em}

\subsection*{Contents}

\begin{itemize}
\setlength{\itemsep}{-1ex}
   \item Usage
   \item Description
   \item Example
\end{itemize}


\subsection*{Usage}

\begin{par}
XC = center(X)
\end{par} \vspace{1em}
\begin{par}
Input
\end{par} \vspace{1em}
\begin{itemize}
\setlength{\itemsep}{-1ex}
   \item X: A matrix or a column vector.
\end{itemize}
\begin{par}
Output
\end{par} \vspace{1em}
\begin{itemize}
\setlength{\itemsep}{-1ex}
   \item XC: A matrix or a column vector with the mean for each column equal to 0.
\end{itemize}


\subsection*{Description}

\begin{par}
This function centerizes a matrix or a vector, by substracting each column by its column mean.
\end{par} \vspace{1em}


\subsection*{Example}

\begin{par}
a = [1 2 3; 4 5 6];
\end{par} \vspace{1em}
\begin{par}
center(a)
\end{par} \vspace{1em}


\newpage


    
\section{Contr}

\begin{par}
Compute the contraction matrix of dimension r.
\end{par} \vspace{1em}

\subsection*{Contents}

\begin{itemize}
\setlength{\itemsep}{-1ex}
   \item Usage
   \item Description
\end{itemize}


\subsection*{Usage}

\begin{par}
C = Contr(r)
\end{par} \vspace{1em}
\begin{par}
Input
\end{par} \vspace{1em}
\begin{itemize}
\setlength{\itemsep}{-1ex}
   \item r: Dimension of the contraction matrix.  A positive integer.
\end{itemize}
\begin{par}
Output
\end{par} \vspace{1em}
\begin{itemize}
\setlength{\itemsep}{-1ex}
   \item C: Contraction matrix of dimension r.  C is an r(r+1)/2 by r\^{}2 matrix.
\end{itemize}


\subsection*{Description}

\begin{par}
The contraction and expansion matrices are links between the "vec" operator and "vech"operator: for an r by r symmetric matrix A, vech(A)=Contr(r)vec(A), and vec(A)=Expan(r)vech(A). The "vec" operator stacks the matrix A into an r\^{}2 by 1 vector columnwise.  The "vech" operator stacks the lower triangle or the upper triangle of a symmetric matrix into an r(r+1)/2 vector. For more details of "vec", "vech", contraction and expansion matrix, refer to Henderson and Searle (1979).
\end{par} \vspace{1em}




\newpage



 
    
\section{Expan}

\begin{par}
Compute the expansion matrix of dimension r.
\end{par} \vspace{1em}

\subsection*{Contents}

\begin{itemize}
\setlength{\itemsep}{-1ex}
   \item Usage
   \item Description
\end{itemize}


\subsection*{Usage}

\begin{par}
E = Expan(r)
\end{par} \vspace{1em}
\begin{par}
Input
\end{par} \vspace{1em}
\begin{itemize}
\setlength{\itemsep}{-1ex}
   \item r: Dimension of the expansion matrix.  A positive integer.
\end{itemize}
\begin{par}
Output
\end{par} \vspace{1em}
\begin{itemize}
\setlength{\itemsep}{-1ex}
   \item E: Expansion matrix of dimension r.  E is an r\^{}2 by r(r+1)/2 matrix.
\end{itemize}


\subsection*{Description}

\begin{par}
The contraction and expansion matrices are links between the "vec" operator and "vech"operator: for an r by r symmetric matrix A, vech(A)=Contr(r)vec(A), and vec(A)=Expan(r)vech(A). The "vec" operator stacks the matrix A into an r\^{}2 by 1 vector columnwise.  The "vech" operator stacks the lower triangle or the upper triangle of a symmetric matrix into an r(r+1)/2 vector. For more details of "vec", "vech", contraction and expansion matrix, refer to Henderson and Searle (1979).
\end{par} \vspace{1em}


\newpage



    
\section{fit\_OLS}

\begin{par}
Multivariate linear regression.
\end{par} \vspace{1em}

\subsection*{Contents}

\begin{itemize}
\setlength{\itemsep}{-1ex}
   \item Usage
   \item Description
\end{itemize}


\subsection*{Usage}

\begin{par}
[betaOLS SigmaOLS]=fit\_OLS(X,Y)
\end{par} \vspace{1em}
\begin{par}
Input
\end{par} \vspace{1em}
\begin{itemize}
\setlength{\itemsep}{-1ex}
   \item X: Predictors, an n by p matrix, p is the number of predictors.  The predictors can be univariate or multivariate, discrete or continuous.
   \item Y: Multivariate responses, an n by r matrix, r is the number of responses and n is number of observations.  The responses must be continuous variables.
\end{itemize}
\begin{par}
Output
\end{par} \vspace{1em}
\begin{itemize}
\setlength{\itemsep}{-1ex}
   \item betaOLS: An r by p matrix containing estimate of the regression coefficients $\beta$.
   \item SigmaOLS: An r by r matrix containing estimate of the error covariance matrix.
\end{itemize}


\subsection*{Description}

\begin{par}
In a multivariate linear model, Y and X follows the following relationship: $Y=\alpha+\beta X+\varepsilon$, where $\varepsilon$ contains the errors.  This function performs the ordinary least squares fit to the inputs, and returns the estimates of $\beta$ and the covariance matrix of $\varepsilon$.
\end{par} \vspace{1em}

\newpage



    
\section{get\_Init}

\begin{par}
Starting value for the envelope subspace.
\end{par} \vspace{1em}

\subsection*{Contents}

\begin{itemize}
\setlength{\itemsep}{-1ex}
   \item Usage
   \item Description
   \item Reference
\end{itemize}


\subsection*{Usage}

\begin{par}
WInit=get\_Init(X,Y,u,dataParameter)
\end{par} \vspace{1em}
\begin{par}
Input
\end{par} \vspace{1em}
\begin{itemize}
\setlength{\itemsep}{-1ex}
   \item X: Predictors. An n by p matrix, p is the number of predictors.
   \item Y: Multivariate responses. An n by r matrix, r is the number of responses and n is number of observations.
   \item u: Dimension of the envelope. An integer between 1 and r-1.
   \item dataParameter: A list containing commonly used statistics computed from the data.
\end{itemize}
\begin{par}
Output
\end{par} \vspace{1em}
\begin{itemize}
\setlength{\itemsep}{-1ex}
   \item WInit: The initial estimate of the orthogonal basis of the envelope subspace. An r by u orthogonal matrix.
\end{itemize}


\subsection*{Description}

\begin{par}
We compute the eigenvectors for the covariance matrices of Y and the estimated errors, and get 2r vectors.  Then we get all the combinations of u vectors out of the 2r vectors. If the number of 2r choose u is small(\ensuremath{<}=50), we search over all the combinations and find out the one that minimizes the objective function F. If that number is large, then we do it iteratively: we pick up any u eigenvectors, fix all of them except the first one. Then we search over all the vectors orthogonal to the fixed ones, and record the one that minimizes F. Next, we fix the first u eigenvectors again but this time search for second one, then we record the vector. This goes on and on until the last one. We do it for 5 rounds and use the final set as our starting value.
\end{par} \vspace{1em}


\subsection*{Reference}

\begin{par}
The codes is implemented based on the algorithm in Section 3.5 of Su and Cook (2011).
\end{par} \vspace{1em}




\newpage



 
\section{Kpd}

\begin{par}
Compute the communication matrix Kpd.
\end{par} \vspace{1em}

\subsection*{Contents}

\begin{itemize}
\setlength{\itemsep}{-1ex}
   \item Usage
   \item Description
   \item Reference
\end{itemize}


\subsection*{Usage}

\begin{par}
k=Kpd(p,d)
\end{par} \vspace{1em}
\begin{par}
Input
\end{par} \vspace{1em}
\begin{itemize}
\setlength{\itemsep}{-1ex}
   \item p and d are two positive integers represent the dimension parameters for the communication matrix.
\end{itemize}
\begin{par}
Output
\end{par} \vspace{1em}
\begin{itemize}
\setlength{\itemsep}{-1ex}
   \item k: The communication matrix Kpd. An p*d by p*d matrix.
\end{itemize}


\subsection*{Description}

\begin{par}
For a p by d matrix A, vec(A')=Kpd*vec(A), and Kpd is called a communication matrix.
\end{par} \vspace{1em}


\subsection*{Reference}

\begin{par}
The codes is implemented based on Definition 3.1 in Magnus and Neudecker (1979).
\end{par} \vspace{1em}



\newpage



\section{Lmatrix}

\begin{par}
Extract the 2nd to the last diagonal element of a matrix into a vector.
\end{par} \vspace{1em}

\subsection*{Contents}

\begin{itemize}
\setlength{\itemsep}{-1ex}
   \item Usage
   \item Description
\end{itemize}


\subsection*{Usage}

\begin{par}
L=Lmatrix(r)
\end{par} \vspace{1em}
\begin{par}
Input
\end{par} \vspace{1em}
\begin{itemize}
\setlength{\itemsep}{-1ex}
   \item r: The dimension of the matrix being extracted.  The matrix should be an r by r matrix.
\end{itemize}
\begin{par}
Output
\end{par} \vspace{1em}
\begin{itemize}
\setlength{\itemsep}{-1ex}
   \item L: An r-1 dimensional vector that contains all the diagonal elements but the first one of the matrix.
\end{itemize}


\subsection*{Description}

\begin{par}
Let A be an r by r matrix, and vec be the vector operator, then Lmatrix(r)*vec(A) will give the 2nd to the rth diagonal elements of A, arranged in a column vector.
\end{par} \vspace{1em}

\newpage
\section{make\_parameter}

\begin{par}
Compute summary statistics from the data.
\end{par} \vspace{1em}

\subsection*{Contents}

\begin{itemize}
\setlength{\itemsep}{-1ex}
   \item Usage
   \item Description
\end{itemize}


\subsection*{Usage}

\begin{par}
dataParameter=make\_parameter(X,Y,method)
\end{par} \vspace{1em}
\begin{par}
Input
\end{par} \vspace{1em}
\begin{itemize}
\setlength{\itemsep}{-1ex}
   \item X: Predictors. An n by p matrix, p is the number of predictors. The predictors can be univariate or multivariate, discrete or continuous.
   \item Y: Multivariate responses. An n by r matrix, r is the number of responses and n is number of observations. The responses must be continuous variables, and r should be strictly greater than p.
   \item method: A string of characters indicating which member of the envelope family to be used, the choices can be 'env', 'ienv', 'henv' or 'senv'.
\end{itemize}
\begin{par}
Output
\end{par} \vspace{1em}
\begin{par}
dataParameter: A list that contains summary statistics computed from the data.  The output list can vary from method to method.
\end{par} \vspace{1em}
\begin{itemize}
\setlength{\itemsep}{-1ex}
   \item dataParameter.n: The number of observations in the data.  A positive integer.
   \item dataParameter.ng: A p by 1 vector containing the number of observations in each group.  p is the number of groups.  Only for 'henv'.
   \item dataParameter.ncum: A p by 1 vector containing the total number of observations till this group.  Only for 'henv'.
   \item dataParameter.ind: An n by 1 vector indicating the sequence of the observations after sorted by groups.
   \item dataParameter.p: The number of predictors or number of groups for 'henv'.  A positive integer.
   \item dataParameter.r: The number of responses.  A positive integer.
   \item dataParameter.XC: Centered predictors.  An n by p matrix with the ith row being the ith observation of X subtracted by the mean of X.  Only for 'env' and 'ienv'.
   \item dataParameter.YC: Centered responses.  An n by r matrix with the ith row being the ith observation of Y subtracted by the mean of Y.  Only for 'env' and 'ienv'.
   \item dataParameter.mX: The mean of predictors.  A p by 1 vector.  For all method except 'henv'.
   \item dataParameter.mY: The mean of responses.  An r by 1 vector.
   \item dataParameter.mYg: An r by p matrix with the ith column being the sample mean of the ith group.
   \item dataParameter.sigX: The sample covariance matrix of X.  A p by p matrix.
   \item dataParameter.sigY: The sample covariance matrix of Y.  An r by r matrix.
   \item dataParameter.sigRes: For 'env', 'senv', 'ienv': The sample covariance matrix of the residuals from the ordinary least squares regression of Y on X.  An r by r matrix. For 'henv', an r by r by p three dimensional matrix with the ith depth is the ith sample covariance matrix for the ith group.
   \item dataParameter.sigFit: The sample covariance matrix of the fitted value from the ordinary least squares regression of Y on X.  An r by r matrix. Only for method 'ienv'.
   \item dataParameter.betaOLS: The regression coefficients from the ordinary least squares regression of Y on X.  An r by p matrix.  For all method except 'henv'.
\end{itemize}


\subsection*{Description}

\begin{par}
This function computes statistics that will be used frequently in the estimation for each method.
\end{par} \vspace{1em}

\newpage

\part{env}
    

 
\section{aic\_env}

\begin{par}
Select the dimension of the envelope subspace using Akaike information criterion.
\end{par} \vspace{1em}

\subsection*{Contents}

\begin{itemize}
\setlength{\itemsep}{-1ex}
   \item Usage
   \item Description
   \item Example
\end{itemize}


\subsection*{Usage}

\begin{par}
u=aic\_env(X,Y)
\end{par} \vspace{1em}
\begin{par}
Input
\end{par} \vspace{1em}
\begin{itemize}
\setlength{\itemsep}{-1ex}
   \item X: Predictors. An n by p matrix, p is the number of predictors. The predictors can be univariate or multivariate, discrete or continuous.
   \item Y: Multivariate responses. An n by r matrix, r is the number of responses and n is number of observations. The responses must be continuous variables.
\end{itemize}
\begin{par}
Output
\end{par} \vspace{1em}
\begin{itemize}
\setlength{\itemsep}{-1ex}
   \item u: Dimension of the envelope. An integer between 0 and r.
\end{itemize}


\subsection*{Description}

\begin{par}
This function implements the Akaike information criteria (AIC) to select the dimension of the envelope subspace.
\end{par} \vspace{1em}


\subsection*{Example}

\begin{par}
load wheatprotein.txt \\
X=wheatprotein(:,8); \\
Y=wheatprotein(:,1:6); \\
u=aic\_env(X,Y)
\end{par} \vspace{1em}


\newpage



\section{bic\_env}

\begin{par}
Select the dimension of the envelope subspace using Bayesian information criterion.
\end{par} \vspace{1em}

\subsection*{Contents}

\begin{itemize}
\setlength{\itemsep}{-1ex}
   \item Usage
   \item Description
   \item Example
\end{itemize}


\subsection*{Usage}

\begin{par}
u=bic\_env(X,Y)
\end{par} \vspace{1em}
\begin{par}
Input
\end{par} \vspace{1em}
\begin{itemize}
\setlength{\itemsep}{-1ex}
   \item X: Predictors. An n by p matrix, p is the number of predictors and n is the number of observations. The predictors can be univariate or multivariate, discrete or continuous.
   \item Y: Multivariate responses. An n by r matrix, r is the number of responses. The responses must be continuous variables.
\end{itemize}
\begin{par}
Output
\end{par} \vspace{1em}
\begin{itemize}
\setlength{\itemsep}{-1ex}
   \item u: Dimension of the envelope. An integer between 0 and r.
\end{itemize}


\subsection*{Description}

\begin{par}
This function implements the Bayesian information criteria (BIC) to select the dimension of the envelope subspace.
\end{par} \vspace{1em}

\subsection*{Example}

\begin{par}
load wheatprotein.txt \\
X=wheatprotein(:,8); \\
Y=wheatprotein(:,1:6); \\
u=bic\_env(X,Y)
\end{par} \vspace{1em}

\newpage




   
    
\section{bstrp\_env}

\begin{par}
Compute bootstrap standard error for the envelope model.
\end{par} \vspace{1em}

\subsection*{Contents}

\begin{itemize}
\setlength{\itemsep}{-1ex}
   \item Usage
   \item Description
   \item Example
\end{itemize}


\subsection*{Usage}

\begin{par}
bootse=bstrp\_env(X,Y,B,u)
\end{par} \vspace{1em}
\begin{par}
Input
\end{par} \vspace{1em}
\begin{itemize}
\setlength{\itemsep}{-1ex}
   \item X: Predictors, an n by p matrix, p is the number of predictors.  The predictors can be univariate or multivariate, discrete or continuous.
   \item Y: Multivariate responses, an n by r matrix, r is the number of responses and n is number of observations.  The responses must be continuous variables.
   \item B: Number of boostrap samples.  A positive integer.
   \item u: Dimension of the envelope subspace.  A positive integer between 0 and r.
\end{itemize}
\begin{par}
Output
\end{par} \vspace{1em}
\begin{itemize}
\setlength{\itemsep}{-1ex}
   \item bootse: The standard error for elements in $\beta$ computed by bootstrap.  An r by p matrix.
\end{itemize}


\subsection*{Description}

\begin{par}
This function computes the bootstrap standard errors for the regression coefficients in the envelope model by bootstrapping the residuals.
\end{par} \vspace{1em}


\subsection*{Example}

\begin{par}
load wheatprotein.txt\\
X=wheatprotein(:,8);\\
Y=wheatprotein(:,1:6);\\
alpha=0.01;\\
u=lrt\_env(Y,X,alpha)\\
B=100;\\
bootse=bstrp\_env(X,Y,B,u)\\
\end{par} \vspace{1em}


\newpage




    
\section{dF4env}

\begin{par}
The first derivative of the objective funtion for computing the envelope subspace.
\end{par} \vspace{1em}

\subsection*{Contents}

\begin{itemize}
\setlength{\itemsep}{-1ex}
   \item Usage
   \item Description
\end{itemize}


\subsection*{Usage}

\begin{par}
df = dF4env(R,dataParameter)
\end{par} \vspace{1em}
\begin{par}
Input
\end{par} \vspace{1em}
\begin{itemize}
\setlength{\itemsep}{-1ex}
   \item R: An r by u semi orthogonal matrix, 0\ensuremath{<}u\ensuremath{<}=r.
   \item dataParameter: A structure that contains the statistics calculated form the data.
\end{itemize}
\begin{par}
Output
\end{par} \vspace{1em}
\begin{itemize}
\setlength{\itemsep}{-1ex}
   \item df: An r by u containing the value of the derivative function evaluated at R.
\end{itemize}


\subsection*{Description}

\begin{par}
The objective function is derived in Section 4.3 in Cook et al. (2010) by  using maximum likelihood estimation. This function is the derivative of  the objective function.
\end{par} \vspace{1em}



\newpage



\section{env}

\begin{par}
Fit the envelope model.
\end{par} \vspace{1em}

\subsection*{Contents}

\begin{itemize}
\setlength{\itemsep}{-1ex}
   \item Usage
   \item Description
   \item References
   \item Example
\end{itemize}


\subsection*{Usage}

\begin{par}
stat=env(X,Y,u)
\end{par} \vspace{1em}
\begin{par}
Input
\end{par} \vspace{1em}
\begin{itemize}
\setlength{\itemsep}{-1ex}
   \item X: Predictors. An n by p matrix, p is the number of predictors. The predictors can be univariate or multivariate, discrete or continuous.
   \item Y: Multivariate responses. An n by r matrix, r is the number of responses and n is number of observations. The responses must be continuous variables, and r should be strictly greater than p.
   \item u: Dimension of the envelope. An integer between 0 and r.
\end{itemize}
\begin{par}
Output
\end{par} \vspace{1em}
\begin{par}
stat: A list that contains the maximum likelihood estimators and some statistics.
\end{par} \vspace{1em}
\begin{itemize}
\setlength{\itemsep}{-1ex}
   \item stat.beta: The envelope estimator of the regression coefficients $\beta$. An r by p matrix.
   \item stat.Sigma: The envelope estimator of the error covariance matrix.  An r by r matrix.
   \item stat.Gamma: The orthogonal basis of the envelope subspace. An r by u semi-orthogonal matrix.
   \item stat.Gamma0: The orthogonal basis of the complement of the envelope subspace.  An r by r-u semi-orthogonal matrix.
   \item stat.eta: The coordinates of $\beta$ with respect to Gamma. An u by p matrix.
   \item stat.Omega: The coordinates of Sigma with respect to Gamma. An u by u matrix.
   \item stat.Omega0: The coordinates of Sigma with respect to Gamma0. An r-u by r-u matrix.
   \item stat.alpha: The estimated intercept in the envelope model.  An r by 1 vector.
   \item stat.l: The maximized log likelihood function.  A real number.
   \item stat.asyEnv: Asymptotic standard error for elements in $\beta$ under the envelope model.  An r by p matrix.  The standard errors returned are asymptotic, for actual standard errors, multiply by 1/sqrt(n).
   \item stat.ratio: The asymptotic standard error ratio of the standard multivariate linear regression estimator over the envelope estimator, for each element in $\beta$.  An r by p matrix.
   \item stat.np: The number of parameters in the envelope model.  A positive integer.
\end{itemize}


\subsection*{Description}

\begin{par}
This function fits the envelope model to the responses and predictors, using the maximum likehood estimation.  When the dimension of the envelope is between 1 and r-1, we implemented the algorithm in Cook et al. (2010).  When the dimension is r, then the envelope model degenerates to the standard multivariate linear regression.  When the dimension is 0, it means that X and Y are uncorrelated, and the fitting is different.
\end{par} \vspace{1em}


\subsection*{References}

\begin{itemize}
\setlength{\itemsep}{-1ex}
   \item The codes is implemented based on the algorithm in Section 4.3 of Cook et al (2010).
   \item The Grassmann manifold optimization step calls the package sg\_min 2.4.1 by Ross Lippert (http://web.mit.edu/$\sim$ripper/www.sgmin.html).
\end{itemize}


\subsection*{Example}

\begin{par}
The following codes will reconstruct the results in the wheat protein data example in Cook et al. (2010).
\end{par} \vspace{1em}
\begin{par}
load wheatprotein.txt \\
X=wheatprotein(:,8); \\
Y=wheatprotein(:,1:6);\\
 alpha=0.01; \\
 u=lrt\_env(Y,X,alpha) \\
 stat=env(X,Y,u) \\
 stat.Omega\\
  eig(stat.Omega0)\\
   stat.ratio
\end{par} \vspace{1em}



\newpage



\section{F4env}

\begin{par}
Objective funtion for computing the envelope subspace.
\end{par} \vspace{1em}

\subsection*{Contents}

\begin{itemize}
\setlength{\itemsep}{-1ex}
   \item Usage
   \item Description
\end{itemize}


\subsection*{Usage}

\begin{par}
f = F4env(R,dataParameter)
\end{par} \vspace{1em}
\begin{par}
Input
\end{par} \vspace{1em}
\begin{itemize}
\setlength{\itemsep}{-1ex}
   \item R: An r by u semi orthogonal matrix, 0\ensuremath{<}u\ensuremath{<}=r.
   \item dataParameter: A structure that contains the statistics calculated form the data.
\end{itemize}
\begin{par}
Output
\end{par} \vspace{1em}
\begin{itemize}
\setlength{\itemsep}{-1ex}
   \item f: A scalar containing the value of the objective function evaluated at R.
\end{itemize}


\subsection*{Description}

\begin{par}
The objective function is derived in Section 4.3 of Cook et al. (2010)  using maximum likelihood estimation. The columns of the semi-orthogonal matrix that minimizes this function span the estimated envelope subspace.
\end{par} \vspace{1em}



\newpage




    
\section{lrt\_env}

\begin{par}
Select the dimension of the envelope subspace using likelihood ratio testing.
\end{par} \vspace{1em}

\subsection*{Contents}

\begin{itemize}
\setlength{\itemsep}{-1ex}
   \item Usage
   \item Description
   \item Example
\end{itemize}


\subsection*{Usage}

\begin{par}
u=lrt\_env(X,Y,alpha)
\end{par} \vspace{1em}
\begin{par}
Input
\end{par} \vspace{1em}
\begin{itemize}
\setlength{\itemsep}{-1ex}
   \item X: Predictors. An n by p matrix, p is the number of predictors. The predictors can be univariate or multivariate, discrete or continuous.
   \item Y: Multivariate responses. An n by r matrix, r is the number of responses and n is number of observations. The responses must be continuous variables.
   \item alpha: Significance level for testing.  A real number between 0 and 1, often taken at 0.05 or 0.01.
\end{itemize}
\begin{par}
Output
\end{par} \vspace{1em}
\begin{itemize}
\setlength{\itemsep}{-1ex}
   \item u: Dimension of the envelope. An integer between 0 and r.
\end{itemize}


\subsection*{Description}

\begin{par}
This function implements the likelihood ratio testing procedure to select the dimension of the envelope subspace, with prespecified significance level $\alpha$.
\end{par} \vspace{1em}


\subsection*{Example}

\begin{par}
load wheatprotein.txt \\
X=wheatprotein(:,8); \\
Y=wheatprotein(:,1:6); \\
alpha=0.01; \\
u=lrt\_env(Y,X,alpha)
\end{par} \vspace{1em}


\newpage


\part{henv}



\section{aic\_henv}

\begin{par}
Select the dimension of the envelope subspace using Akaike information criterion for the heteroscedastic envelope model.
\end{par} \vspace{1em}

\subsection*{Contents}

\begin{itemize}
\setlength{\itemsep}{-1ex}
   \item Usage
   \item Description
   \item Example
\end{itemize}


\subsection*{Usage}

\begin{par}
u=aic\_henv(X,Y)
\end{par} \vspace{1em}
\begin{par}
Input
\end{par} \vspace{1em}
\begin{itemize}
\setlength{\itemsep}{-1ex}
   \item X: Group indicators. An n by p matrix, p is the number of groups. X can only take p different values, one for each group.
   \item Y: Multivariate responses. An n by r matrix, r is the number of responses and n is number of observations. The responses must be continuous variables.
\end{itemize}
\begin{par}
Output
\end{par} \vspace{1em}
\begin{itemize}
\setlength{\itemsep}{-1ex}
   \item u: Dimension of the envelope. An integer between 0 and r.
\end{itemize}


\subsection*{Description}

\begin{par}
This function implements the Akaike information criteria (AIC) to select the dimension of the envelope subspace for the heteroscedastic envelope model.
\end{par} \vspace{1em}

\subsection*{Example}

\begin{par}
load waterstrider.mat\\
u=aic\_henv(X,Y)
\end{par} \vspace{1em}

\newpage


    
\section{bic\_henv}

\begin{par}
Select the dimension of the envelope subspace using Bayesian information criterion for the heteroscedastic envelope model.
\end{par} \vspace{1em}

\subsection*{Contents}

\begin{itemize}
\setlength{\itemsep}{-1ex}
   \item Usage
   \item Description
   \item Example
\end{itemize}


\subsection*{Usage}

\begin{par}
u=bic\_henv(X,Y)
\end{par} \vspace{1em}
\begin{par}
Input
\end{par} \vspace{1em}
\begin{itemize}
\setlength{\itemsep}{-1ex}
   \item X: Group indicators. An n by p matrix, p is the number of groups. X can only take p different values, one for each group.
   \item Y: Multivariate responses. An n by r matrix, r is the number of responses and n is number of observations. The responses must be continuous variables.
\end{itemize}
\begin{par}
Output
\end{par} \vspace{1em}
\begin{itemize}
\setlength{\itemsep}{-1ex}
   \item u: Dimension of the envelope. An integer between 0 and r.
\end{itemize}


\subsection*{Description}

\begin{par}
This function implements the Bayesian information criteria (BIC) to select the dimension of the envelope subspace for the heteroscedastic envelope model.
\end{par} \vspace{1em}


\subsection*{Example}

\begin{par}
load waterstrider.mat\\
u=bic\_henv(X,Y)
\end{par} \vspace{1em}


\newpage




 
    
\section{bstrp\_henv}

\begin{par}
Compute bootstrap standard error for the heteroscedastic envelope model.
\end{par} \vspace{1em}

\subsection*{Contents}

\begin{itemize}
\setlength{\itemsep}{-1ex}
   \item Usage
   \item Description
   \item Example
\end{itemize}


\subsection*{Usage}

\begin{par}
bootse=bstrp\_henv(X,Y,B,u)
\end{par} \vspace{1em}
\begin{par}
Input
\end{par} \vspace{1em}
\begin{itemize}
\setlength{\itemsep}{-1ex}
   \item X: Group indicators. An n by p matrix, p is the number of groups. X can only take p different values, one for each group.
   \item Y: Multivariate responses, an n by r matrix, r is the number of responses and n is number of observations.  The responses must be continuous variables.
   \item B: Number of boostrap samples.  A positive integer.
   \item u: Dimension of the envelope subspace.  A positive integer between 0 and r.
\end{itemize}
\begin{par}
Output
\end{par} \vspace{1em}
\begin{itemize}
\setlength{\itemsep}{-1ex}
   \item bootse: The standard error for elements in $\beta$ computed by bootstrap.  An r by p matrix.
\end{itemize}


\subsection*{Description}

\begin{par}
This function computes the bootstrap standard errors for the regression coefficients in the heteroscedastic envelope model by bootstrapping the residuals.
\end{par} \vspace{1em}

\subsection*{Example}

\begin{par}
load waterstrider.mat\\
u=lrt\_henv(X,Y,0.01)\\
B=100;\\
bootse=bstrp\_henv(X,Y,B,u)

\end{par} \vspace{1em}


\newpage


    
\section{dF4henv}

\begin{par}
The first derivative of the objective funtion for computing the envelope subspace in the heteroscedastic envelope model.
\end{par} \vspace{1em}

\subsection*{Contents}

\begin{itemize}
\setlength{\itemsep}{-1ex}
   \item Usage
   \item Description
\end{itemize}


\subsection*{Usage}

\begin{par}
df = dF4henv(R,dataParameter)
\end{par} \vspace{1em}
\begin{par}
Input
\end{par} \vspace{1em}
\begin{itemize}
\setlength{\itemsep}{-1ex}
   \item R: An r by u semi orthogonal matrix, 0\ensuremath{<}u\ensuremath{<}=r.
   \item dataParameter: A structure that contains the statistics calculated form the data.
\end{itemize}
\begin{par}
Output
\end{par} \vspace{1em}
\begin{itemize}
\setlength{\itemsep}{-1ex}
   \item df: An r by u containing the value of the derivative function evaluated at R.
\end{itemize}


\subsection*{Description}

\begin{par}
The objective function is derived in Section 2.2 in Su and Cook (2012) by  using maximum likelihood estimation. This function is the derivative of  the objective function.
\end{par} \vspace{1em}



\newpage




    
\section{F4henv}

\begin{par}
Objective funtion for computing the envelope subspace in heteroscedastic envelope model.
\end{par} \vspace{1em}

\subsection*{Contents}

\begin{itemize}
\setlength{\itemsep}{-1ex}
   \item Usage
   \item Description
\end{itemize}


\subsection*{Usage}

\begin{par}
f = F4henv(R,dataParameter)
\end{par} \vspace{1em}
\begin{par}
Input
\end{par} \vspace{1em}
\begin{itemize}
\setlength{\itemsep}{-1ex}
   \item R: An r by u semi orthogonal matrix, 0\ensuremath{<}u\ensuremath{<}=r.
   \item dataParameter: A structure that contains the statistics calculated form the data.
\end{itemize}
\begin{par}
Output
\end{par} \vspace{1em}
\begin{itemize}
\setlength{\itemsep}{-1ex}
   \item f: A scalar containing the value of the objective function evaluated at R.
\end{itemize}


\subsection*{Description}

\begin{par}
The objective function is derived in Section 2.2 of Su and Cook (2012)  using maximum likelihood estimation. The columns of the semi-orthogonal matrix that minimizes this function span the estimated envelope subspace in the heteroscedastic envelope model.
\end{par} \vspace{1em}

\newpage



    
\section{henv}

\begin{par}
Fit the heteroscedastic envelope model.
\end{par} \vspace{1em}

\subsection*{Contents}

\begin{itemize}
\setlength{\itemsep}{-1ex}
   \item Usage
   \item Description
   \item References
   \item Example
\end{itemize}


\subsection*{Usage}

\begin{par}
stat=henv(X,Y,u)
\end{par} \vspace{1em}
\begin{par}
Input
\end{par} \vspace{1em}
\begin{itemize}
\setlength{\itemsep}{-1ex}
   \item X: Group indicators. An n by p matrix, p is the number of groups. X can only take p different values, one for each group.
   \item Y: Multivariate responses. An n by r matrix, r is the number of responses and n is number of observations. The responses must be continuous variables, and r should be strictly greater than p.
   \item u: Dimension of the envelope. An integer between 0 and r.
\end{itemize}
\begin{par}
Output
\end{par} \vspace{1em}
\begin{par}
stat: A list that contains the maximum likelihood estimators and some statistics.
\end{par} \vspace{1em}
\begin{itemize}
\setlength{\itemsep}{-1ex}
   \item stat.mu: The heteroscedastic envelope estimator of the grand mean. A r by 1 vector.
   \item stat.mug: The heteroscedastic envelope estimator of the group mean. A r by p matrix, the ith column of the matrix contains the mean for the ith group.
   \item stat.Yfit: A n by r matrix, the ith row gives the group mean of the group that the ith observation belongs to.  As X is just a group indicator, and is not ordinal, stat.mug alone does not tell which group corresponds to which group mean.
   \item stat.Gamma: The orthogonal basis of the envelope subspace. An r by u semi-orthogonal matrix.
   \item stat.Gamma0: The orthogonal basis of the complement of the envelope subspace.  An r by r-u semi-orthogonal matrix.
   \item stat.beta: The heteroscedastic envelope estimator of the group mean effect. An r by p matrix, the ith column of the matrix contains the main effect for the ith group.
   \item stat.Sigma: The heteroscedastic envelope estimator of the error covariance matrix.  A three dimensional matrix with dimension r, r and p, stat.Sigma(:,:,i) contains the estimated covariance matrix for the ith group.
   \item stat.eta: The coordinates of $\beta$ with respect to Gamma. An u by p matrix, the ith column contains the coordinates of the main effect of the ith group with respect to Gamma.
   \item stat.Omega: The coordinates of Sigma with respect to Gamma. An u by u by p matrix, stat.Omega(:,:,i) contains the coordinates of the covariance matrix of the ith group with respect to Gamma.
   \item stat.Omega0: The coordinates of Sigma with respect to Gamma0. An r-u by r-u matrix.
   \item stat.l: The maximized log likelihood function.  A real number.
   \item stat.np: The number of parameters in the heteroscedastic envelope model.  A positive integer.
   \item stat.asyHenv: The asymptotic standard errors for elements in $beta$ under the heteroscedastic envelope model. An r by p matrix.  The standard errors returned are asymptotic, for actual standard errors, multiply by 1/sqrt(n).
   \item stat.ratio: The asymptotic standard error ratio of the standard multivariate linear regression estimator over the heteroscedastic envelope estimator. An r by p matrix, the (i, j)th element in stat.ratio is the elementwise standard error ratio for the ith element in the jth group mean effect.
\end{itemize}


\subsection*{Description}

\begin{par}
This function fits the heteroscedatic envelope model to the responses and predictors, using the maximum likehood estimation.  When the dimension of the envelope is between 1 and r-1, we implemented the algorithm in Su and Cook (2012). When the dimension is r, then the envelope model degenerates to the standard multivariate linear model for comparing group means.  When the dimension is 0, it means there is not any group effect, and the fitting is different.
\end{par} \vspace{1em}


\subsection*{References}

\begin{itemize}
\setlength{\itemsep}{-1ex}
   \item The codes is implemented based on the algorithm in Section 2.2 of Su and Cook (2012).
   \item The Grassmann manifold optimization step calls the package sg\_min 2.4.1 by Ross Lippert (http://web.mit.edu/$\sim$ripper/www.sgmin.html).
\end{itemize}


\subsection*{Example}

\begin{par}
The following codes produce the results of the waterstrider example in Su
and Cook (2011).

load waterstrider.mat\\
u=lrt\_henv(X,Y,0.01)\\
stat=henv(X,Y,u)\\
stat.ratio

\end{par} \vspace{1em}





\newpage




 
    
\section{lrt\_henv}

\begin{par}
Select the dimension of the envelope subspace using likelihood ratio testing for the heteroscedastic envelope model.
\end{par} \vspace{1em}

\subsection*{Contents}

\begin{itemize}
\setlength{\itemsep}{-1ex}
   \item Usage
   \item Description
   \item Example
\end{itemize}


\subsection*{Usage}

\begin{par}
u=lrt\_henv(X,Y,alpha)
\end{par} \vspace{1em}
\begin{par}
Input
\end{par} \vspace{1em}
\begin{itemize}
\setlength{\itemsep}{-1ex}
   \item X: Group indicators. An n by p matrix, p is the number of groups. X can only take p different values, one for each group.
   \item Y: Multivariate responses. An n by r matrix, r is the number of responses and n is number of observations. The responses must be continuous variables.
   \item alpha: Significance level for testing.  A real number between 0 and 1, often taken at 0.05 or 0.01.
\end{itemize}
\begin{par}
Output
\end{par} \vspace{1em}
\begin{itemize}
\setlength{\itemsep}{-1ex}
   \item u: Dimension of the envelope. An integer between 0 and r.
\end{itemize}


\subsection*{Description}

\begin{par}
This function implements the likelihood ratio testing procedure to select the dimension of the envelope subspace in heteroscedastic envelope model, with prespecified significance level $\alpha$.
\end{par} \vspace{1em}


\subsection*{Example}

\begin{par}
load waterstrider.mat\\
u=lrt\_henv(X,Y,0.01)


\end{par} \vspace{1em}



\newpage




\part{ienv}




    
\section{aic\_ienv}

\begin{par}
Select the dimension of the inner envelope subspace using Akaike information criterion.
\end{par} \vspace{1em}

\subsection*{Contents}

\begin{itemize}
\setlength{\itemsep}{-1ex}
   \item Usage
   \item Description
   \item Example
\end{itemize}


\subsection*{Usage}

\begin{par}
u=aic\_ienv(X,Y)
\end{par} \vspace{1em}
\begin{par}
Input
\end{par} \vspace{1em}
\begin{itemize}
\setlength{\itemsep}{-1ex}
   \item X: Predictors. An n by p matrix, p is the number of predictors and n is the number of observations. The predictors can be univariate or multivariate, discrete or continuous.
   \item Y: Multivariate responses. An n by r matrix, r is the number of responses. The responses must be continuous variables.
\end{itemize}
\begin{par}
Output
\end{par} \vspace{1em}
\begin{itemize}
\setlength{\itemsep}{-1ex}
   \item u: Dimension of the inner envelope. An integer between 0 and p or equal to r.
\end{itemize}


\subsection*{Description}

\begin{par}
This function implements the Akaike information criteria (AIC) to select the dimension of the inner envelope subspace.
\end{par} \vspace{1em}


\subsection*{Example}

\begin{par}
load irisf.mat \\
u=aic\_ienv(X,Y)
\end{par} \vspace{1em}




\newpage





 
    
\section{bic\_ienv}

\begin{par}
Select the dimension of the inner envelope subspace using Bayesian information criterion.
\end{par} \vspace{1em}

\subsection*{Contents}

\begin{itemize}
\setlength{\itemsep}{-1ex}
   \item Usage
   \item Description
   \item Example
\end{itemize}


\subsection*{Usage}

\begin{par}
u=bic\_ienv(X,Y)
\end{par} \vspace{1em}
\begin{par}
Input
\end{par} \vspace{1em}
\begin{itemize}
\setlength{\itemsep}{-1ex}
   \item X: Predictors. An n by p matrix, p is the number of predictors and n is the number of observations. The predictors can be univariate or multivariate, discrete or continuous.
   \item Y: Multivariate responses. An n by r matrix, r is the number of responses. The responses must be continuous variables.
\end{itemize}
\begin{par}
Output
\end{par} \vspace{1em}
\begin{itemize}
\setlength{\itemsep}{-1ex}
   \item u: Dimension of the inner envelope. An integer between 0 and p or equal to r.
\end{itemize}


\subsection*{Description}

\begin{par}
This function implements the Bayesian information criteria (BIC) to select the dimension of the inner envelope subspace.
\end{par} \vspace{1em}


\subsection*{Example}

\begin{par}
load irisf.mat \\
u=bic\_ienv(X,Y)
\end{par} \vspace{1em}



\newpage




   
    
\section{bstrp\_ienv}

\begin{par}
Compute bootstrap standard error for the inner envelope model.
\end{par} \vspace{1em}

\subsection*{Contents}

\begin{itemize}
\setlength{\itemsep}{-1ex}
   \item Usage
   \item Description
   \item Example
\end{itemize}


\subsection*{Usage}

\begin{par}
bootse=bstrp\_ienv(X,Y,B,u)
\end{par} \vspace{1em}
\begin{par}
Input
\end{par} \vspace{1em}
\begin{itemize}
\setlength{\itemsep}{-1ex}
   \item X: Predictors, an n by p matrix, p is the number of predictors.  The predictors can be univariate or multivariate, discrete or continuous.
   \item Y: Multivariate responses, an n by r matrix, r is the number of responses and n is number of observations.  The responses must be continuous variables.
   \item B: Number of boostrap samples.  A positive integer.
   \item u: Dimension of the inner envelope. An integer between 0 and p or equal to r.
\end{itemize}
\begin{par}
Output
\end{par} \vspace{1em}
\begin{itemize}
\setlength{\itemsep}{-1ex}
   \item bootse: The standard error for elements in $\beta$ computed by bootstrap.  An r by p matrix.
\end{itemize}


\subsection*{Description}

\begin{par}
This function computes the bootstrap standard errors for the regression coefficients in the inner envelope model by bootstrapping the residuals.
\end{par} \vspace{1em}


\subsection*{Example}

\begin{par}
load irisf.mat
\end{par} \vspace{1em}
\begin{par}
u=bic\_ienv(X,Y) \\
B=100; \\
bootse=bstrp\_ienv(X,Y,B,u)
\end{par} \vspace{1em}




\newpage





  
\section{dF4ienv}

\begin{par}
First derivative of the objective funtion for computing the inner envelope subspace.
\end{par} \vspace{1em}

\subsection*{Contents}

\begin{itemize}
\setlength{\itemsep}{-1ex}
   \item Usage
   \item Description
\end{itemize}


\subsection*{Usage}

\begin{par}
f = dF4ienv(R,dataParameter)
\end{par} \vspace{1em}
\begin{par}
Input
\end{par} \vspace{1em}
\begin{itemize}
\setlength{\itemsep}{-1ex}
   \item R: An r by u semi-orthogonal matrix, 0\ensuremath{<}u\ensuremath{<}=p.
   \item dataParameter: A structure that contains the statistics calculated form the data.
\end{itemize}
\begin{par}
Output
\end{par} \vspace{1em}
\begin{itemize}
\setlength{\itemsep}{-1ex}
   \item dF: The first derivative of the objective function for computing the inner envelope subspace.  An r by u matrix.
\end{itemize}


\subsection*{Description}

\begin{par}
This first derivative of F4ienv obtained by matrix calculus calculations.
\end{par} \vspace{1em}




\newpage






\section{F4ienv}

\begin{par}
Objective funtion for computing the inner envelope subspace
\end{par} \vspace{1em}

\subsection*{Contents}

\begin{itemize}
\setlength{\itemsep}{-1ex}
   \item Usage
   \item Description
\end{itemize}


\subsection*{Usage}

\begin{par}
f = F4ienv(R,dataParameter)
\end{par} \vspace{1em}
\begin{par}
Input
\end{par} \vspace{1em}
\begin{itemize}
\setlength{\itemsep}{-1ex}
   \item R: An r by u semi orthogonal matrix, 0\ensuremath{<}u\ensuremath{<}=p.
   \item dataParameter: A structure that contains the statistics calculated form the data.
\end{itemize}
\begin{par}
Output
\end{par} \vspace{1em}
\begin{itemize}
\setlength{\itemsep}{-1ex}
   \item f: A scalar containing the value of the objective function evaluated at R.
\end{itemize}


\subsection*{Description}

\begin{par}
The objective function is derived in Section 3.3 in Su and Cook (2012) by  using maximum likelihood estimation. The columns of the semi-orthogonal matrix that minimizes this function span the estimated inner envelope subspace.
\end{par} \vspace{1em}



\newpage



  
    
\section{ienv}

\begin{par}
Fit the inner envelope model.
\end{par} \vspace{1em}

\subsection*{Contents}

\begin{itemize}
\setlength{\itemsep}{-1ex}
   \item Usage
   \item Description
   \item References
   \item Example
\end{itemize}


\subsection*{Usage}

\begin{par}
stat=ienv(X,Y,u)
\end{par} \vspace{1em}
\begin{par}
Input
\end{par} \vspace{1em}
\begin{itemize}
\setlength{\itemsep}{-1ex}
   \item X: Predictors. An n by p matrix, p is the number of predictors. The predictors can be univariate or multivariate, discrete or continuous.
   \item Y: Multivariate responses. An n by r matrix, r is the number of responses and n is number of observations. The responses must be continuous variables, and r should be strictly greater than p.
   \item u: Dimension of the inner envelope. An integer between 0 and p or equal to r.
\end{itemize}
\begin{par}
Output
\end{par} \vspace{1em}
\begin{par}
stat: A list that contains the maximum likelihood estimators and some statistics.
\end{par} \vspace{1em}
\begin{itemize}
\setlength{\itemsep}{-1ex}
   \item stat.beta: The envelope estimator of the regression coefficients $\beta$. An r by p matrix.
   \item stat.Sigma: The envelope estimator of the error covariance matrix.  An r by r matrix.
   \item stat.Gamma1: The orthogonal basis of the inner envelope subspace. An r by u semi-orthogonal matrix.
   \item stat.Gamma0: The orthogonal basis of the complement of the inner envelope subspace.  An r by r-u semi-orthogonal matrix.
   \item stat.eta1: The transpose of the coordinates of $\beta$ with respect to Gamma1. An p by u matrix.
   \item stat.B: An (r-u) by (p-u) semi-orthogonal matrix, so that (Gamma, Gamma0*B) spans $\beta$.
   \item stat.eta2: The transpose of the coordinates of $\beta$ with respect to Gamma0. An p by (p-u) matrix.
   \item stat.Omega1: The coordinates of Sigma with respect to Gamma1. An u by u matrix.
   \item stat.Omega0: The coordinates of Sigma with respect to Gamma0. An r-u by r-u matrix.
   \item stat.alpha: The estimated intercept in the inner envelope model.  An r by 1 vector.
   \item stat.l: The maximized log likelihood function.  A real number.
   \item stat.asyIenv: Asymptotic standard error for elements in $\beta$ under the inner envelope model.  An r by p matrix.  The standard errors returned are asymptotic, for actual standard errors, multiply by 1/sqrt(n).
   \item stat.ratio: The asymptotic standard error ratio of the stanard multivariate linear regression estimator over the inner envelope estimator, for each element in $\beta$.  An r by p matrix.
   \item stat.np: The number of parameters in the inner envelope model.  A positive integer.
\end{itemize}


\subsection*{Description}

\begin{par}
This function fits the inner envelope model to the responses and predictors, using the maximum likehood estimation.  When the dimension of the envelope is between 1 and p-1, we implemented the algorithm in Su and Cook (2012).  When the dimension is p, then the inner envelope model degenerates to the standard multivariate linear regression.  When the dimension is 0, it means that X and Y are uncorrelated, and the fitting is different.
\end{par} \vspace{1em}


\subsection*{References}

\begin{itemize}
\setlength{\itemsep}{-1ex}
   \item The codes is implemented based on the algorithm in Su and Cook (2012).
   \item The Grassmann manifold optimization step calls the package sg\_min 2.4.1 by Ross Lippert (http://web.mit.edu/$\sim$ripper/www.sgmin.html).
\end{itemize}


\subsection*{Example}

\begin{par}
The following codes gives the results of the Fisher's iris data example in Su and Cook (2012).
\end{par} \vspace{1em}
\begin{par}
load irisf.mat
\end{par} \vspace{1em}
\begin{par}
u=bic\_env(X,Y) \\
d=bic\_ienv(X,Y) \\
stat=ienv(X,Y,d) \\
1-1./stat.ratio
\end{par} \vspace{1em}



\newpage





    
\section{lrt\_ienv}

\begin{par}
Select the dimension of the inner envelope subspace using likelihood ratio testing.
\end{par} \vspace{1em}

\subsection*{Contents}

\begin{itemize}
\setlength{\itemsep}{-1ex}
   \item Usage
   \item Description
   \item Example
\end{itemize}


\subsection*{Usage}

\begin{par}
u=lrt\_ienv(X,Y,alpha)
\end{par} \vspace{1em}
\begin{par}
Input
\end{par} \vspace{1em}
\begin{itemize}
\setlength{\itemsep}{-1ex}
   \item X: Predictors. An n by p matrix, p is the number of predictors. The predictors can be univariate or multivariate, discrete or continuous.
   \item Y: Multivariate responses. An n by r matrix, r is the number of responses and n is number of observations. The responses must be continuous variables.
   \item alpha: Significance level for testing.  A real number between 0 and 1, often taken at 0.05 or 0.01.
\end{itemize}
\begin{par}
Output
\end{par} \vspace{1em}
\begin{itemize}
\setlength{\itemsep}{-1ex}
   \item u: Dimension of the inner envelope. An integer between 0 and p or equal to r.
\end{itemize}


\subsection*{Description}

\begin{par}
This function implements the likelihood ratio testing procedure to select the dimension of the inner envelope subspace, with prespecified significance level $\alpha$.
\end{par} \vspace{1em}


\subsection*{Example}

\begin{par}
load irisf.mat
\end{par} \vspace{1em}
\begin{par}
alpha=0.01;\\
 u=lrt\_ienv(X,Y,alpha)
\end{par} \vspace{1em}



\part{penv}





    
    
\section{aic\_penv}

\begin{par}
Select the dimension of the partial envelope subspace using Akaike information criterion.
\end{par} \vspace{1em}

\subsection*{Contents}

\begin{itemize}
\setlength{\itemsep}{-1ex}
   \item Usage
   \item Description
   \item Example
\end{itemize}


\subsection*{Usage}

\begin{par}
u=aic\_penv(X1,X2,Y)
\end{par} \vspace{1em}
\begin{par}
Input
\end{par} \vspace{1em}
\begin{itemize}
\setlength{\itemsep}{-1ex}
   \item X1: Predictors of main interst. An n by p1 matrix, n is the number of observations, and p1 is the number of main predictors. The predictors can be univariate or multivariate, discrete or continuous.
   \item X2: Covariates, or predictors not of main interest.  An n by p2 matrix, p2 is the number of covariates.
   \item Y: Multivariate responses. An n by r matrix, r is the number of responses and n is number of observations. The responses must be continuous variables.
\end{itemize}
\begin{par}
Output
\end{par} \vspace{1em}
\begin{itemize}
\setlength{\itemsep}{-1ex}
   \item u: Dimension of the envelope. An integer between 0 and r.
\end{itemize}


\subsection*{Description}

\begin{par}
This function implements the Akaike information criteria (AIC) to select the dimension of the partial envelope subspace.
\end{par} \vspace{1em}


\subsection*{Example}

\begin{par}
load T7-7.dat \\
Y=T7\_7(:,1:4); \\
X=T7\_7(:,5:7); \\
X1=X(:,3); \\
X2=X(:,1:2); \\
u=aic\_penv(X1,X2,Y)
\end{par} \vspace{1em}



\newpage




  
    
\section{bic\_penv}

\begin{par}
Select the dimension of the partial envelope subspace using Bayesian information criterion.
\end{par} \vspace{1em}

\subsection*{Contents}

\begin{itemize}
\setlength{\itemsep}{-1ex}
   \item Usage
   \item Description
   \item Example
\end{itemize}


\subsection*{Usage}

\begin{par}
u=bic\_penv(X1,X2,Y)
\end{par} \vspace{1em}
\begin{par}
Input
\end{par} \vspace{1em}
\begin{itemize}
\setlength{\itemsep}{-1ex}
   \item X1: Predictors of main interst. An n by p1 matrix, n is the number of observations, and p1 is the number of main predictors. The predictors can be univariate or multivariate, discrete or continuous.
   \item X2: Covariates, or predictors not of main interest.  An n by p2 matrix, p2 is the number of covariates.
   \item Y: Multivariate responses. An n by r matrix, r is the number of responses and n is number of observations. The responses must be continuous variables.
\end{itemize}
\begin{par}
Output
\end{par} \vspace{1em}
\begin{itemize}
\setlength{\itemsep}{-1ex}
   \item u: Dimension of the envelope. An integer between 0 and r.
\end{itemize}


\subsection*{Description}

\begin{par}
This function implements the Bayesian information criteria (BIC) to select the dimension of the partial envelope subspace.
\end{par} \vspace{1em}


\subsection*{Example}

\begin{par}
load T7-7.dat \\
Y=T7\_7(:,1:4); \\
X=T7\_7(:,5:7); \\
X1=X(:,3); \\
X2=X(:,1:2); \\
u=bic\_penv(X1,X2,Y)
\end{par} \vspace{1em}



\newpage




    
\section{bstrp\_penv}

\begin{par}
Compute bootstrap standard error for the partial envelope model.
\end{par} \vspace{1em}

\subsection*{Contents}

\begin{itemize}
\setlength{\itemsep}{-1ex}
   \item Usage
   \item Description
   \item Example
\end{itemize}


\subsection*{Usage}

\begin{par}
bootse=bstrp\_penv(X1,X2,Y,B,u)
\end{par} \vspace{1em}
\begin{par}
Input
\end{par} \vspace{1em}
\begin{itemize}
\setlength{\itemsep}{-1ex}
   \item X1: Predictors of main interst. An n by p1 matrix, n is the number of observations, and p1 is the number of main predictors. The predictors can be univariate or multivariate, discrete or continuous.
   \item X2: Covariates, or predictors not of main interest.  An n by p2 matrix, p2 is the number of covariates.  The covariates can be univariate or multivariate, discrete or continuous.
   \item Y: Multivariate responses, an n by r matrix, r is the number of responses and n is number of observations.  The responses must be continuous variables.
   \item B: Number of boostrap samples.  A positive integer.
   \item u: Dimension of the partial envelope subspace.  A positive integer between 0 and r.
\end{itemize}
\begin{par}
Output
\end{par} \vspace{1em}
\begin{itemize}
\setlength{\itemsep}{-1ex}
   \item bootse: The standard error for elements in $\beta_1$ computed by bootstrap.  An r by p1 matrix.
\end{itemize}


\subsection*{Description}

\begin{par}
This function computes the bootstrap standard errors for the regression coefficients in the partial envelope model by bootstrapping the residuals.
\end{par} \vspace{1em}


\subsection*{Example}

\begin{par}
load T7-7.dat \\
Y=T7\_7(:,1:4); \\
X=T7\_7(:,5:7); \\
X1=X(:,3); \\
X2=X(:,1:2); \\
alpha=0.01; \\
u=lrt\_penv(X1,X2,Y,alpha) \\
B=100; \\
bootse=bstrp\_penv(X1,X2,Y,B,u)
\end{par} \vspace{1em}



\newpage






 
    
\section{lrt\_penv}

\begin{par}
Select the dimension of the partial envelope subspace using likelihood ratio testing.
\end{par} \vspace{1em}

\subsection*{Contents}

\begin{itemize}
\setlength{\itemsep}{-1ex}
   \item Usage
   \item Description
   \item Example
\end{itemize}


\subsection*{Usage}

\begin{par}
u=lrt\_penv(X1,X2,Y,alpha)
\end{par} \vspace{1em}
\begin{par}
Input
\end{par} \vspace{1em}
\begin{itemize}
\setlength{\itemsep}{-1ex}
   \item X1: Predictors of main interst. An n by p1 matrix, n is the number of observations, and p1 is the number of main predictors. The predictors can be univariate or multivariate, discrete or continuous.
   \item X2: Covariates, or predictors not of main interest.  An n by p2 matrix, p2 is the number of covariates.
   \item Y: Multivariate responses. An n by r matrix, r is the number of responses. The responses must be continuous variables.
   \item alpha: Significance level for testing.  A real number between 0 and 1, often taken at 0.05 or 0.01.
\end{itemize}
\begin{par}
Output
\end{par} \vspace{1em}
\begin{itemize}
\setlength{\itemsep}{-1ex}
   \item u: Dimension of the partial envelope subspace. An integer between 0 and r.
\end{itemize}


\subsection*{Description}

\begin{par}
This function implements the likelihood ratio testing procedure to select the dimension of the partial envelope subspace, with prespecified significance level $\alpha$.
\end{par} \vspace{1em}


\subsection*{Example}

\begin{par}
load T7-7.dat \\
Y=T7\_7(:,1:4); \\
X=T7\_7(:,5:7); \\
X1=X(:,3);\\ 
X2=X(:,1:2); \\
alpha=0.01; \\
u=lrt\_penv(X1,X2,Y,alpha)
\end{par} \vspace{1em}




\newpage






    
\section{penv}

\begin{par}
Fit the partial envelope model.
\end{par} \vspace{1em}

\subsection*{Contents}

\begin{itemize}
\setlength{\itemsep}{-1ex}
   \item Usage
   \item Description
   \item References
   \item Example
\end{itemize}


\subsection*{Usage}

\begin{par}
stat=penv(X1,X2,Y,u)
\end{par} \vspace{1em}
\begin{par}
Input
\end{par} \vspace{1em}
\begin{itemize}
\setlength{\itemsep}{-1ex}
   \item X1: Predictors of main interst. An n by p1 matrix, n is the number of observations, and p1 is the number of main predictors. The predictors can be univariate or multivariate, discrete or continuous.
   \item X2: Covariates, or predictors not of main interest.  An n by p2 matrix, p2 is the number of covariates.  The covariates can be univariate or multivariate, discrete or continuous.
   \item Y: Multivariate responses. An n by r matrix, r is the number of responses and n is number of observations. The responses must be continuous variables, and r should be strictly greater than p1.
   \item u: Dimension of the partial envelope. An integer between 0 and r.
\end{itemize}
\begin{par}
Output
\end{par} \vspace{1em}
\begin{par}
stat: A list that contains the maximum likelihood estimators and some statistics.
\end{par} \vspace{1em}
\begin{itemize}
\setlength{\itemsep}{-1ex}
   \item stat.beta1: The partial envelope estimator of $\beta_1$, which is the regression coefficients for X1. An r by p1 matrix.
   \item stat.beta2: The partial envelope estimator of $\beta_2$, which is the regression coefficients for X2. An r by p2 matrix.
   \item stat.Sigma: The partial envelope estimator of the error covariance matrix.  An r by r matrix.
   \item stat.Gamma: The orthogonal basis of the partial envelope subspace. An r by u semi-orthogonal matrix.
   \item stat.Gamma0: The orthogonal basis of the complement of the partial envelope subspace.  An r by r-u semi-orthogonal matrix.
   \item stat.eta: The coordinates of $\beta_1$ with respect to Gamma. An u by p1 matrix.
   \item stat.Omega: The coordinates of Sigma with respect to Gamma. An u by u matrix.
   \item stat.Omega0: The coordinates of Sigma with respect to Gamma0. An r-u by r-u matrix.
   \item stat.alpha: The estimated intercept in the partial envelope model.  An r by 1 vector.
   \item stat.l: The maximized log likelihood function.  A real number.
   \item stat.asyPenv: Asymptotic standard error for elements in $\beta$ under the partial envelope model.  An r by p1 matrix.  The standard errors returned are asymptotic, for actual standard errors, multiply by 1/sqrt(n).
   \item stat.ratio: The asymptotic standard error ratio of the stanard multivariate linear regression estimator over the partial envelope estimator, for each element in $\beta_1$.  An r by p1 matrix.
   \item stat.np: The number of parameters in the envelope model.  A positive integer.
\end{itemize}


\subsection*{Description}

\begin{par}
This function fits the partial envelope model to the responses Y and predictors X1 and X2, using the maximum likehood estimation.  When the dimension of the envelope is between 1 and r-1, we implemented the algorithm in Su and Cook (2011).  When the dimension is r, then the partial envelope model degenerates to the standard multivariate linear regression with Y as the responses and both X1 and X2 as predictors.  When the dimension is 0, X1 and Y are uncorrelated, and the fitting is the standard multivariate linear regression with Y as the responses and X2 as the predictors.
\end{par} \vspace{1em}


\subsection*{References}

\begin{itemize}
\setlength{\itemsep}{-1ex}
   \item The codes is implemented based on the algorithm in Section 3.2 of Su and Cook (2012).
   \item The Grassmann manifold optimization step calls the package sg\_min 2.4.1 by Ross Lippert (http://web.mit.edu/$\sim$ripper/www.sgmin.html).
\end{itemize}


\subsection*{Example}

\begin{par}
The following codes reconstruct the results of the paper and fiber example in Su and Cook (2012).
\end{par} \vspace{1em}
\begin{par}
load T7-7.dat \\
Y=T7\_7(:,1:4); \\
X=T7\_7(:,5:7); \\
X1=X(:,3); \\
X2=X(:,1:2); \\
alpha=0.01; \\
u=lrt\_penv(X1,X2,Y,alpha) \\
stat=penv(X1,X2,Y,u) \\
stat.Omega \\
eig(stat.Omega0) \\
stat.ratio
\end{par} \vspace{1em}



\newpage




\part{senv}
    
    
\section{aic\_senv}

\begin{par}
Select the dimension of the scaled envelope subspace using Akaike information criterion.
\end{par} \vspace{1em}

\subsection*{Contents}

\begin{itemize}
\setlength{\itemsep}{-1ex}
   \item Usage
   \item Description
   \item Example
\end{itemize}


\subsection*{Usage}

\begin{par}
u=aic\_senv(X,Y)
\end{par} \vspace{1em}
\begin{par}
Input
\end{par} \vspace{1em}
\begin{itemize}
\setlength{\itemsep}{-1ex}
   \item X: Predictors. An n by p matrix, p is the number of predictors and n is the number of observations. The predictors can be univariate or multivariate, discrete or continuous.
   \item Y: Multivariate responses. An n by r matrix, r is the number of responses. The responses must be continuous variables.
\end{itemize}
\begin{par}
Output
\end{par} \vspace{1em}
\begin{itemize}
\setlength{\itemsep}{-1ex}
   \item u: Dimension of the inner envelope. An integer between 0 and r.
\end{itemize}


\subsection*{Description}

\begin{par}
This function implements the Akaike information criteria (AIC) to select the dimension of the scaled envelope subspace.
\end{par} \vspace{1em}


\subsection*{Example}

\begin{par}
load('T9-12.txt')\\
Y=T9\_12(:,4:7);\\
X=T9\_12(:,1:3);\\
u=aic\_senv(X,Y)

\end{par} \vspace{1em}


\newpage





    
\section{bic\_senv}

\begin{par}
Select the dimension of the scaled envelope subspace using Bayesian information criterion.
\end{par} \vspace{1em}

\subsection*{Contents}

\begin{itemize}
\setlength{\itemsep}{-1ex}
   \item Usage
   \item Description
   \item Example
\end{itemize}


\subsection*{Usage}

\begin{par}
u=bic\_senv(X,Y)
\end{par} \vspace{1em}
\begin{par}
Input
\end{par} \vspace{1em}
\begin{itemize}
\setlength{\itemsep}{-1ex}
   \item X: Predictors. An n by p matrix, p is the number of predictors and n is the number of observations. The predictors can be univariate or multivariate, discrete or continuous.
   \item Y: Multivariate responses. An n by r matrix, r is the number of responses. The responses must be continuous variables.
\end{itemize}
\begin{par}
Output
\end{par} \vspace{1em}
\begin{itemize}
\setlength{\itemsep}{-1ex}
   \item u: Dimension of the inner envelope. An integer between 0 and r.
\end{itemize}


\subsection*{Description}

\begin{par}
This function implements the Bayesian information criteria (BIC) to select the dimension of the scaled envelope subspace.
\end{par} \vspace{1em}


\subsection*{Example}

\begin{par}
load('T9-12.txt')\\
Y=T9\_12(:,4:7);\\
X=T9\_12(:,1:3);\\
u=bic\_senv(X,Y)

\end{par} \vspace{1em}

\newpage





\section{bstrp\_senv}

\begin{par}
Compute bootstrap standard error for the scaled envelope model.
\end{par} \vspace{1em}

\subsection*{Contents}

\begin{itemize}
\setlength{\itemsep}{-1ex}
   \item Usage
   \item Description
   \item Example
\end{itemize}


\subsection*{Usage}

\begin{par}
bootse=bstrp\_senv(X,Y,B,u)
\end{par} \vspace{1em}
\begin{par}
Input
\end{par} \vspace{1em}
\begin{itemize}
\setlength{\itemsep}{-1ex}
   \item X: Predictors, an n by p matrix, p is the number of predictors.  The predictors can be univariate or multivariate, discrete or continuous.
   \item Y: Multivariate responses, an n by r matrix, r is the number of responses and n is number of observations.  The responses must be continuous variables.
   \item B: Number of boostrap samples.  A positive integer.
   \item u: Dimension of the envelope subspace.  A positive integer between 0 and r.
\end{itemize}
\begin{par}
Output
\end{par} \vspace{1em}
\begin{itemize}
\setlength{\itemsep}{-1ex}
   \item bootse: The standard error for elements in $\beta$ computed by bootstrap.  An r by p matrix.
\end{itemize}


\subsection*{Description}

\begin{par}
This function computes the bootstrap standard errors for the regression coefficients in the scaled envelope model by bootstrapping the residuals.
\end{par} \vspace{1em}

\subsection*{Example}

\begin{par}
load('T9-12.txt')\\
Y=T9\_12(:,4:7);\\
X=T9\_12(:,1:3);\\
u=bic\_ienv(X,Y)\\
B=20;\\
bootse=bstrp\_senv(X,Y,B,u)

\end{par} \vspace{1em}

\newpage




    
\section{dF4senv}

\begin{par}
First derivative of the objective funtion for computing the envelope subspace in the scaled envelope model.
\end{par} \vspace{1em}

\subsection*{Contents}

\begin{itemize}
\setlength{\itemsep}{-1ex}
   \item Usage
   \item Description
\end{itemize}


\subsection*{Usage}

\begin{par}
f = dF4senv(R,dataParameter)
\end{par} \vspace{1em}
\begin{par}
Input
\end{par} \vspace{1em}
\begin{itemize}
\setlength{\itemsep}{-1ex}
   \item R: An r by u semi-orthogonal matrix, 0\ensuremath{<}u\ensuremath{<}=p.
   \item dataParameter: A structure that contains the statistics calculated form the data.
\end{itemize}
\begin{par}
Output
\end{par} \vspace{1em}
\begin{itemize}
\setlength{\itemsep}{-1ex}
   \item dF: The first derivative of the objective function for computing the  envelope subspace.  An r by u matrix.
\end{itemize}


\subsection*{Description}

\begin{par}
This first derivative of F4senv obtained by matrix calculus calculations.
\end{par} \vspace{1em}


\newpage




    
\section{F4senv}

\begin{par}
Objective funtion for computing the envelope subspace in scaled envelope model.
\end{par} \vspace{1em}

\subsection*{Contents}

\begin{itemize}
\setlength{\itemsep}{-1ex}
   \item Usage
   \item Description
\end{itemize}


\subsection*{Usage}

\begin{par}
f = F4senv(R,dataParameter)
\end{par} \vspace{1em}
\begin{par}
Input
\end{par} \vspace{1em}
\begin{itemize}
\setlength{\itemsep}{-1ex}
   \item R: An r by u semi orthogonal matrix, 0\ensuremath{<}u\ensuremath{<}r.
   \item dataParameter: A structure that contains the statistics calculated form the data.
\end{itemize}
\begin{par}
Output
\end{par} \vspace{1em}
\begin{itemize}
\setlength{\itemsep}{-1ex}
   \item f: A scalar containing the value of the objective function evaluated at R.
\end{itemize}


\subsection*{Description}

\begin{par}
The objective function is derived in Section 4.1 in Cook and Su (2012) using maximum likelihood estimation. The columns of the semi-orthogonal matrix that minimizes this function span the estimated envelope subspace.
\end{par} \vspace{1em}


\newpage





    
\section{objfun}

\begin{par}
Objective funtion for computing the scales in the scaled envelope model.
\end{par} \vspace{1em}

\subsection*{Contents}

\begin{itemize}
\setlength{\itemsep}{-1ex}
   \item Usage
   \item Description
\end{itemize}


\subsection*{Usage}

\begin{par}
f = objfun(d,Gamma,dataParameter)
\end{par} \vspace{1em}
\begin{par}
Input
\end{par} \vspace{1em}
\begin{itemize}
\setlength{\itemsep}{-1ex}
   \item d: An r-1 dimensional column vector containing the scales for the 2nd to the rth responses.  All the entries in d are positive.
   \item Gamma: A r by u semi-orthogomal matrix that spans the envelope subspace or the estimated envelope subspace.
   \item dataParameter: A structure that contains the statistics calculated form the data.
\end{itemize}
\begin{par}
Output
\end{par} \vspace{1em}
\begin{itemize}
\setlength{\itemsep}{-1ex}
   \item f: A scalar containing the value of the objective function evaluated at d.
\end{itemize}


\subsection*{Description}

\begin{par}
The objective function is derived in Section 4.1 of Su and Cook (2012)  using maximum likelihood estimation.
\end{par} \vspace{1em}



\newpage




\section{senv}

\begin{par}
Fit the scaled envelope model.
\end{par} \vspace{1em}

\subsection*{Contents}

\begin{itemize}
\setlength{\itemsep}{-1ex}
   \item Usage
   \item Description
   \item References
   \item Example
\end{itemize}


\subsection*{Usage}

\begin{par}
stat=senv(X,Y,u)
\end{par} \vspace{1em}
\begin{par}
Input
\end{par} \vspace{1em}
\begin{itemize}
\setlength{\itemsep}{-1ex}
   \item X: Predictors. An n by p matrix, p is the number of predictors. The predictors can be univariate or multivariate, discrete or continuous.
   \item Y: Multivariate responses. An n by r matrix, r is the number of responses and n is number of observations. The responses must be continuous variables, and r should be strictly greater than p.
   \item u: Dimension of the envelope. An integer between 0 and r.
\end{itemize}
\begin{par}
Output
\end{par} \vspace{1em}
\begin{par}
stat: A list that contains the maximum likelihood estimators and some statistics.
\end{par} \vspace{1em}
\begin{itemize}
\setlength{\itemsep}{-1ex}
   \item stat.beta: The scaled envelope estimator of the regression coefficients $\beta$. An r by p matrix.
   \item stat.Sigma: The scaled envelope estimator of the error covariance matrix.  An r by r matrix.
   \item stat.Lambda: The matrix of estimated scales. An r by r diagonal matrix with the first diagonal element equal to 1 and other diagonal elements being positive.
   \item stat.Gamma: The orthogonal basis of the envelope subspace. An r by u semi-orthogonal matrix.
   \item stat.Gamma0: The orthogonal basis of the complement of the envelope subspace.  An r by r-u semi-orthogonal matrix.
   \item stat.eta: The coordinates of $\beta$ with respect to Gamma. An u by p matrix.
   \item stat.Omega: The coordinates of Sigma with respect to Gamma. An u by u matrix.
   \item stat.Omega0: The coordinates of Sigma with respect to Gamma0. An r-u by r-u matrix.
   \item stat.alpha: The estimated intercept in the scaled envelope model.  An r by 1 vector.
   \item stat.l: The maximized log likelihood function.  A real number.
   \item stat.asySenv: Asymptotic standard error for elements in $\beta$ under the scaled envelope model.  An r by p matrix.  The standard errors returned are asymptotic, for actual standard errors, multiply by 1/sqrt(n).
   \item stat.ratio: The asymptotic standard error ratio of the standard multivariate linear regression estimator over the scaled envelope estimator, for each element in $\beta$.  An r by p matrix.
   \item stat.np: The number of parameters in the scaled envelope model.  A positive integer.
\end{itemize}


\subsection*{Description}

\begin{par}
This function fits the scaled envelope model to the responses and predictors, using the maximum likehood estimation.  When the dimension of the envelope is between 1 and r-1, we implemented the algorithm in Cook and Su (2012).  When the dimension is r, then the scaled envelope model degenerates to the standard multivariate linear regression.  When the dimension is 0, it means that X and Y are uncorrelated, and the fitting is different.
\end{par} \vspace{1em}


\subsection*{References}

\begin{itemize}
\setlength{\itemsep}{-1ex}
   \item The codes is implemented based on the algorithm in Section 4.1 of Cook and Su (2012).
   \item The Grassmann manifold optimization step calls the package sg\_min 2.4.1 by Ross Lippert (http://web.mit.edu/$\sim$ripper/www.sgmin.html).
\end{itemize}




\subsection*{Example}

\begin{par}
The following codes produce the results of the test and performance
example in Cook and Su (2012). \vspace{1em}

load('T9-12.txt')\\
Y=T9\_12(:,4:7);\\
X=T9\_12(:,1:3);\\
u=bic\_env(X,Y)\\
stat=env(X,Y,u);\\
1-1./stat.ratio\\
u=bic\_senv(X,Y)\\
stat=senv(X,Y,u);\\
stat.Lambda\\
1-1./stat.ratio
\end{par} \vspace{1em}








\end{document}
    
